\documentclass[a4paper]{article}
\usepackage{amsmath, amssymb, amsfonts}
\usepackage[italian]{babel}
\usepackage[utf8]{inputenc}
\usepackage{graphicx}
\usepackage{hyperref}
\usepackage[a4paper, total={6in, 8in}]{geometry}
\usepackage{minted}
\usepackage{mathpazo}
\usepackage{fancyhdr}
\usepackage{amsthm}
\usepackage{tikz}
\usetikzlibrary{automata,positioning}

% Definizione degli ambienti
\newtheorem{theorem}{Teorema}
\newtheorem{definition}{Definizione}
\newtheorem{example}{Esempio}
\newtheorem{lemma}{Lemma}
\newtheorem{proposition}{Proposizione}
\newcommand{\blankS}{\ensuremath{\square}}
\newtheorem{proof_sketch}{Sketch di Dimostrazione} % Adding a sketch environment for complex proofs

\hypersetup{
    pdftitle={Lezione di Informatica Teorica: Riduzioni e Indecidibilità del PCP},
    pdfauthor={Appunti da Trascrizione AI}
}

\pagestyle{fancy}
\fancyhf{}
\fancyhead[L]{\textit{Lezione di Informatica Teorica}}
\fancyfoot[C]{\thepage}

\title{Lezione di Informatica Teorica: Riduzioni e Indecidibilità del PCP}
\author{Appunti da Trascrizione Automatica}
\date{\today}

\begin{document}
\maketitle
\tableofcontents
\newpage

\section{Introduzione alle Riduzioni}

Una \emph{riduzione} è uno strumento fondamentale in Teoria della Computazione per dimostrare la difficoltà computazionale di problemi.

\begin{definition}[Riduzione Formale]
Siano $A$ e $B$ due linguaggi (problemi di decisione). Una funzione $f: \Sigma^* \to \Sigma^*$ è una \textbf{riduzione calcolabile} di $A$ a $B$ (indicata $A \le_m B$) se:
\begin{enumerate}
    \item $f$ è calcolabile (esiste una Macchina di Turing che calcola $f$ in tempo finito per ogni input).
    \item Per ogni stringa $w \in \Sigma^*$, si ha che $w \in A \iff f(w) \in B$.
\end{enumerate}
Il significato di $w \in A \iff f(w) \in B$ è duplice:
\begin{itemize}
    \item Se $w \in A$, allora $f(w) \in B$.
    \item Se $w \notin A$, allora $f(w) \notin B$.
\end{itemize}
\end{definition}

\subsection{Scopo delle Riduzioni}

Le riduzioni sono utilizzate per dimostrare l'indecidibilità di problemi.
\begin{theorem}
Se $A \le_m B$ e $A$ è un problema indecidibile, allora anche $B$ è un problema indecidibile.
\end{theorem}
\begin{proof}
Supponiamo per contraddizione che $B$ sia decidibile. Allora esisterebbe una Macchina di Turing $M_B$ che decide $B$.
Poiché $f$ è calcolabile, possiamo costruire una Macchina di Turing $M_A$ che decide $A$ come segue:
\begin{enumerate}
    \item Input $w$.
    \item Calcola $f(w)$.
    \item Simula $M_B$ su $f(w)$.
    \item Accetta se $M_B$ accetta, rifiuta se $M_B$ rifiuta.
\end{enumerate}
Per la definizione di riduzione, $w \in A \iff f(w) \in B$. Quindi $M_A$ decide $A$. Ma questo contraddice l'ipotesi che $A$ sia indecidibile. Dunque, $B$ deve essere indecidibile.
\end{proof}

\begin{theorem}[Transitività delle Riduzioni]
Se $A \le_m B$ e $B \le_m C$, allora $A \le_m C$.
\end{theorem}
\begin{proof}
Siano $f$ la riduzione da $A$ a $B$ e $g$ la riduzione da $B$ a $C$.
Entrambe $f$ e $g$ sono calcolabili. Allora la composizione $g \circ f$ è anch'essa calcolabile.
Inoltre, $w \in A \iff f(w) \in B \iff g(f(w)) \in C$.
Quindi $g \circ f$ è una riduzione calcolabile da $A$ a $C$.
\end{proof}

\section{Problema di Corrispondenza di Post (PCP)}

Il Problema di Corrispondenza di Post (PCP) è un problema di decisione sulle stringhe.

\begin{definition}[PCP]
Un'istanza del PCP è data da due liste non-vuote di stringhe, $A = (A_1, A_2, \dots, A_k)$ e $B = (B_1, B_2, \dots, B_k)$, definite su un alfabeto $\Sigma$.
Il problema chiede se esista una sequenza di indici non-vuota $I = (i_1, i_2, \dots, i_m)$ con $m \ge 1$ e $1 \le i_j \le k$ per ogni $j$, tale che:
$$A_{i_1} A_{i_2} \dots A_{i_m} = B_{i_1} B_{i_2} \dots B_{i_m}$$
La risposta è "sì" se una tale sequenza esiste, "no" altrimenti.
\end{definition}

\begin{example}
Consideriamo le liste:
$A = (A_1 = 1, A_2 = 10111, A_3 = 10)$
$B = (B_1 = 111, B_2 = 10, B_3 = 0)$

Una possibile sequenza di indici è $I = (2, 1, 1, 3)$:
$A_2 A_1 A_1 A_3 = (10111)(1)(1)(10) = 101111110$
$B_2 B_1 B_1 B_3 = (10)(111)(111)(0) = 101111110$
Poiché le stringhe risultanti sono uguali, questa istanza ha una soluzione ("sì").
\end{example}

\subsection{Proprietà del PCP}
\begin{itemize}
    \item Il PCP è un problema \textbf{ricorsivamente enumerabile} (in R). Possiamo ideare un algoritmo che, data un'istanza, cerca tutte le possibili sequenze di indici di lunghezza crescente. Se trova una soluzione, accetta. Se non trova una soluzione in un tempo finito, potrebbe continuare a cercare indefinitamente.
    \item Il PCP è \textbf{indecidibile} (non in R). Non esiste un algoritmo che, per ogni istanza del PCP, termina e fornisce la risposta corretta ("sì" o "no"). Dimostreremo questo usando le riduzioni.
\end{itemize}

\section{Problema di Corrispondenza di Post Modificato (MPCP)}

Per dimostrare l'indecidibilità del PCP, useremo un problema intermedio, il MPCP.

\begin{definition}[MPCP]
Un'istanza del MPCP è data da due liste non-vuote di stringhe, $A = (A_1, A_2, \dots, A_k)$ e $B = (B_1, B_2, \dots, B_k)$, definite su un alfabeto $\Sigma$.
Il problema chiede se esista una sequenza di indici $I = (i_1, i_2, \dots, i_m)$ con $m \ge 1$ e $1 \le i_j \le k$ per ogni $j$, tale che:
\begin{enumerate}
    \item La sequenza deve iniziare con l'indice $1$: $i_1 = 1$.
    \item $A_{i_1} A_{i_2} \dots A_{i_m} = B_{i_1} B_{i_2} \dots B_{i_m}$
\end{enumerate}
La risposta è "sì" se una tale sequenza esiste, "no" altrimenti.
\end{definition}

L'unica differenza tra PCP e MPCP è la condizione che la sequenza di indici debba iniziare per $1$.

\section{Riduzione 1: $LU \le_m MPCP$}

Dimostriamo che il Linguaggio Universale ($LU$) si riduce al MPCP. Dato che $LU$ è indecidibile, questo implicherà che MPCP è indecidibile.

\subsection{Idea della Riduzione}
Un'istanza di $LU$ è una coppia $(M, w)$, dove $M$ è una Macchina di Turing e $w$ è una stringa.
Un'istanza di MPCP è una coppia di liste di stringhe $(A, B)$.
La funzione di riduzione $f$ deve trasformare $(M, w)$ in $(A, B)$.
L'idea centrale è simulare la computazione di $M$ su $w$ usando le stringhe delle liste $A$ e $B$. La computazione di una TM può essere rappresentata come una sequenza di \emph{descrizioni istantanee} (ID) separate da un simbolo speciale (e.g., $\#$).
$ID_1 \# ID_2 \# ID_3 \# \dots \# ID_k$
Vogliamo costruire le liste $A$ e $B$ in modo tale che, se $M$ accetta $w$, allora esista una sequenza di indici per MPCP che costruisce la stessa stringa.
Le stringhe di $B$ saranno sempre "un passo avanti" rispetto alle stringhe di $A$, simulando la prossima ID. Quando $M$ accetta, le stringhe di $A$ avranno la possibilità di "recuperare" e allinearsi con quelle di $B$.

\subsection{Costruzione delle Liste $A$ e $B$}
L'alfabeto del MPCP sarà $\Gamma \cup Q \cup \{\#, \$, *\}$, dove $\Gamma$ è l'alfabeto del nastro di $M$, $Q$ è l'insieme degli stati di $M$, e $\#, \$, *$ sono nuovi simboli.
Assumiamo che $M$ non scriva mai il simbolo blank e che il suo nastro sia semi-infinito a destra (non si muova mai a sinistra della posizione iniziale). Queste sono assunzioni standard che non limitano la generalità delle TM.

Le liste $A$ e $B$ sono costruite con le seguenti classi di coppie di stringhe $(A_i, B_i)$:

\begin{enumerate}
    \item \textbf{Coppia Iniziale (obbligatoria per MPCP):}
    Questa coppia inizia la simulazione e garantisce che $B$ sia un passo avanti.
    \begin{itemize}
        \item $A_1 = \#$
        \item $B_1 = \# q_0 w \#$ (dove $q_0$ è lo stato iniziale di $M$, $w$ è la stringa d'input, e $\#$ è un simbolo di confine).
    \end{itemize}

    \item \textbf{Coppie di Copia:}
    Permettono di copiare simboli di configurazione che non sono sotto la testina.
    Per ogni simbolo $x \in \Gamma \cup \{\#\}$:
    \begin{itemize}
        \item $A_i = x$
        \item $B_i = x$
    \end{itemize}

    \item \textbf{Coppie di Transizione:}
    Simulano il movimento della testina e la modifica del nastro secondo le regole di transizione di $M$. Queste regole sono generate per ogni $q \in Q \setminus F$ (stato non finale) e ogni $X \in \Gamma \cup \{\#\}$.
    \begin{itemize}
        \item \textbf{Spostamento a destra ($R$):} Se $\delta(q, X) = (p, Y, R)$:
            \begin{itemize}
                \item $A_i = qX$
                \item $B_i = Yp$
            \end{itemize}
            (Esempio: se $M$ legge $X$ nello stato $q$, scrive $Y$ e va nello stato $p$ muovendosi a destra, allora $qX$ in $A$ corrisponde a $Yp$ in $B$).
            \emph{Caso speciale: $X$ è un blank (simbolo di bordo \texttt{\#}).} Se $\delta(q, \#) = (p, Y, R)$:
            \begin{itemize}
                \item $A_i = q\#$
                \item $B_i = Yp\#$
            \end{itemize}
        \item \textbf{Spostamento a sinistra ($L$):} Se $\delta(q, X) = (p, Y, L)$:
            \begin{itemize}
                \item $A_i = ZqX$ (per ogni $Z \in \Gamma \cup \{\#\}$)
                \item $B_i = pZY$
            \end{itemize}
            (Esempio: se $M$ legge $X$ nello stato $q$, scrive $Y$ e va nello stato $p$ muovendosi a sinistra, allora la sequenza $ZqX$ in $A$ (dove $Z$ è il simbolo a sinistra di $q$) corrisponde a $pZY$ in $B$).
    \end{itemize}

    \item \textbf{Coppie di Accettazione/Recupero:}
    Queste coppie sono utilizzate solo quando $M$ entra in uno stato accettante $q_f \in F$. Permettono alla stringa concatenata di $A$ di "recuperare" la lunghezza della stringa concatenata di $B$.
    Per ogni $q_f \in F$ e ogni $X, Y \in \Gamma$:
    \begin{itemize}
        \item $A_i = Xq_f Y$
        \item $B_i = q_f Y$
    \end{itemize}
    Questo riduce la differenza di lunghezza tra $A_i$ e $B_i$ per $q_f$ e $Y$.
    \begin{itemize}
        \item $A_i = Xq_f \#$
        \item $B_i = q_f \#$
    \end{itemize}
    \begin{itemize}
        \item $A_i = q_f Y \#$
        \item $B_i = q_f \#$
    \end{itemize}
    \begin{itemize}
        \item $A_i = q_f \# \#$
        \item $B_i = q_f \#$
    \end{itemize}
    (Il professore ha menzionato $XQY \to QY$ come regola generale per $Q \in F$. Il principio è che la stringa di A diventa più lunga o uguale, permettendo di chiudere la partita.)

    \item \textbf{Coppia Finale:}
    Permette di completare il match una volta che la TM è in uno stato accettante e la differenza di lunghezza è stata recuperata.
    Per ogni $q_f \in F$:
    \begin{itemize}
        \item $A_i = q_f \# \#$ (due simboli \texttt{\#} per A)
        \item $B_i = \# \#$ (due simboli \texttt{\#} per B)
    \end{itemize}
\end{enumerate}

\subsection{Esempio di Simulazione}
Sia $M$ la Macchina di Turing definita come segue (dal diagramma della lezione):
Stati: $Q = \{q_1, q_2, q_3\}$ (dove $q_1$ è iniziale, $q_3$ è finale).
Alfabeto del nastro: $\Gamma = \{0, 1, \beta\}$ ($\beta$ è il simbolo blank).
Funzione di transizione (interpretata dal diagramma):
$\delta(q_1, 0) = (q_2, 1, R)$
$\delta(q_1, 1) = (q_1, 0, L)$
$\delta(q_2, 1) = (q_1, 0, R)$
$\delta(q_1, \beta) = (q_2, 1, L)$ (loop dal blank, torna indietro e cambia stato)
$\delta(q_2, \beta) = (q_3, 0, R)$ (accetta e si sposta a destra)

Sia la stringa d'input $w = 01$.
La computazione di $M$ su $w=01$ è:
1. $\#q_1 01\#$ (Configurazione iniziale)
2. $\#1q_2 1\#$ (Da $q_1$ legge $0$, scrive $1$, va in $q_2$, si sposta a $R$)
3. $\#10q_1 \#$ (Da $q_2$ legge $1$, scrive $0$, va in $q_1$, si sposta a $R$)
4. $\#1q_2 1\#$ (Da $q_1$ legge $\beta$, scrive $1$, va in $q_2$, si sposta a $L$)
5. $\#10q_3 \#$ (Da $q_2$ legge $\beta$, scrive $0$, va in $q_3$, si sposta a $R$) - Stato accettante $q_3$.

\paragraph{Come la riduzione costruisce la soluzione MPCP:}
\begin{itemize}
    \item Si inizia con la coppia iniziale: $A_1 = \#$, $B_1 = \#q_1 01\#$.
        La stringa concatenata di A (finora $\#$) è "dietro" quella di B (finora $\#q_1 01\#$).
    \item Per simulare $\#q_1 01\# \to \#1q_2 1\#$:
        \begin{itemize}
            \item Usiamo coppie di copia per $\#$ iniziale: $(A_i = \#, B_i = \#)$.
            \item Per la transizione $q_1 0 \to 1q_2$: usiamo la coppia $(A_i = q_1 0, B_i = 1q_2)$.
            \item Usiamo coppie di copia per $1\#$: $(A_i=1, B_i=1)$ e $(A_i=\#, B_i=\#)$.
        \end{itemize}
        A questo punto, la stringa concatenata di A sarà $\#q_1 01\#$, mentre quella di B sarà $\#q_1 01\# \#1q_2 1\#$. B è ancora un passo avanti.
    \item Questo processo continua. Le regole di copia e transizione assicurano che la stringa costruita da $B$ contenga la sequenza di ID, con la stringa costruita da $A$ che la segue con un ID di ritardo.
    \item Quando la computazione raggiunge uno stato finale (es. $q_3$ nell'esempio, $ID_5 = \#10q_3\#$), le coppie di accettazione (Classe 4) permettono alla stringa di $A$ di "catturare" la stringa di $B$. Ad esempio, per $Xq_f Y \to q_f Y$, $A$ contribuisce con 3 simboli e $B$ con 2, riducendo il gap.
    \item Infine, la coppia finale (Classe 5) permette di pareggiare completamente le stringhe. Se $A$ arriva a $\#ID_1\#ID_2\# \dots \#ID_k \#q_f \# \#$ e $B$ arriva a $\#ID_1\#ID_2\# \dots \#ID_k \# \# \#$, allora la coppia finale farà sì che $A$ e $B$ diventino uguali.
\end{itemize}

\subsection{Dimostrazione}
\begin{proof_sketch}
Si deve dimostrare che $(M, w)$ è un'istanza "sì" di $LU$ se e solo se l'istanza $(A, B)$ generata dalla riduzione è un'istanza "sì" di MPCP.

\paragraph{Parte 1: Se $M$ accetta $w$, allora l'istanza MPCP ha soluzione.}
Se $M$ accetta $w$, significa che esiste una sequenza di configurazioni $ID_1, ID_2, \dots, ID_k$ dove $ID_1 = q_0 w$, e $ID_k$ è una configurazione accettante.
Costruiamo la sequenza di indici per MPCP come segue:
Iniziamo sempre con l'indice $1$ (la coppia iniziale). Questo produce $A_{concat} = \#$ e $B_{concat} = \#ID_1\#$.
Poi, si scelgono le coppie di stringhe $(A_i, B_i)$ dalle classi 2 e 3 che simulano la transizione da $ID_j$ a $ID_{j+1}$. Ogni volta che si concatena una coppia di transizione, la stringa di $A$ "completa" la $ID_j$ e la stringa di $B$ "inizia" la $ID_{j+1}$.
Questo fa sì che $A_{concat}$ diventi $\#ID_1\#ID_2\# \dots \#ID_{k-1}\#$ e $B_{concat}$ diventi $\#ID_1\#ID_2\# \dots \#ID_k\#$.
Poiché $ID_k$ è una configurazione accettante, possiamo usare le coppie di Classe 4 per "rimuovere" i simboli vicino allo stato finale in $A$ e $B$ in modo disuguale (o meglio, $A$ apporta più simboli di $B$ per la stessa parte di configurazione), riducendo il divario di lunghezza. Infine, la coppia di Classe 5 permette di far terminare le stringhe con esattamente gli stessi simboli, facendo sì che $A_{concat} = B_{concat}$.
Dunque, esiste una soluzione per l'istanza MPCP.

\paragraph{Parte 2: Se l'istanza MPCP ha soluzione, allora $M$ accetta $w$.}
Supponiamo che l'istanza MPCP $(A, B)$ generata dalla riduzione abbia una soluzione.
Per la definizione di MPCP, questa soluzione deve iniziare con la coppia $(A_1, B_1)$, che è $(\#, \#q_0 w\#)$.
L'unico modo per le stringhe concatenate di $A$ e $B$ di rimanere allineate e infine eguagliarsi è che le coppie usate simulino correttamente le transizioni di $M$.
Qualsiasi sequenza di indici che non rispetti la simulazione delle transizioni (ad esempio, usando una coppia $(X,Y)$ senza che $Y$ sia il risultato della transizione di $X$) non permetterebbe mai alle stringhe di allinearsi correttamente a causa dell'alternanza di simboli di nastro e stato.
Poiché le stringhe di $B$ sono sempre un passo avanti (contengono $\#ID_j\#ID_{j+1}\#$ mentre $A$ ha solo $\#ID_j\#$), l'unico modo per $A$ di "recuperare" e far sì che le stringhe concatenate di $A$ e $B$ diventino uguali è tramite l'uso delle coppie di Classe 4 (Accettazione/Recupero) e Classe 5 (Finale).
Le coppie di Classe 4 e 5 possono essere utilizzate solo se la configurazione attuale di $M$ (simulata) contiene uno stato accettante.
Pertanto, se esiste una soluzione MPCP, la simulazione deve aver raggiunto una configurazione accettante, il che significa che $M$ accetta $w$.
\end{proof_sketch}

Poiché $LU$ è indecidibile e $LU \le_m MPCP$, concludiamo che $MPCP$ è \textbf{indecidibile}.

\section{Riduzione 2: $MPCP \le_m PCP$}

Ora dimostriamo che $MPCP$ si riduce al $PCP$. Dato che $MPCP$ è indecidibile, questo implicherà che $PCP$ è indecidibile.

\subsection{Idea della Riduzione}
Un'istanza di MPCP è una coppia di liste $(A, B)$. Un'istanza di PCP è una coppia di liste $(C, D)$.
La funzione di riduzione $f$ deve trasformare $(A, B)$ in $(C, D)$.
La sfida è che PCP non richiede che la soluzione inizi con un indice specifico, mentre MPCP sì (indice 1). Dobbiamo modificare le liste $(A, B)$ in $(C, D)$ in modo che qualsiasi soluzione PCP debba iniziare con la coppia modificata dall'indice 1 dell'MPCP.

\subsection{Costruzione delle Liste $C$ e $D$}
Introduciamo due nuovi simboli non presenti nell'alfabeto originale: $*$ (asterisco) e $\$$ (dollaro).

Le liste $C$ e $D$ sono costruite come segue:

\begin{enumerate}
    \item \textbf{Trasformazione delle Coppie Originali:}
    Per ogni coppia $(A_i, B_i)$ con $i \in \{1, \dots, k\}$:
    \begin{itemize}
        \item $C_i$: Ogni simbolo di $A_i$ è seguito da un asterisco.
              Esempio: se $A_i = s_1 s_2 \dots s_m$, allora $C_i = s_1 * s_2 * \dots s_m *$.
        \item $D_i$: Ogni simbolo di $B_i$ è preceduto da un asterisco.
              Esempio: se $B_i = t_1 t_2 \dots t_n$, allora $D_i = * t_1 * t_2 \dots * t_n$.
    \end{itemize}

    \item \textbf{Coppia Iniziale Forzata:}
    Per forzare la sequenza a iniziare con l'equivalente dell'indice 1 di MPCP:
    \begin{itemize}
        \item $C_0 = * C_1$ (la stringa $C_1$ (ottenuta dal $A_1$ originale) con un asterisco aggiunto all'inizio).
              Esempio: se $A_1 = 1$, $C_1 = 1*$. Allora $C_0 = *1*$.
        \item $D_0 = D_1$ (la stringa $D_1$ (ottenuta dal $B_1$ originale)).
              Esempio: se $B_1 = 111$, $D_1 = *1*1*1$. Allora $D_0 = *1*1*1$.
    \end{itemize}
    Nota: Tutte le stringhe in $C_i$ (per $i>0$) iniziano con un simbolo dell'alfabeto originale seguito da $^*$, mentre tutte le stringhe in $D_i$ iniziano con $^*$. L'unica eccezione è $C_0$ che inizia con $^*$. Questo forza la scelta di $C_0$ e $D_0$ come prima coppia della soluzione PCP.

    \item \textbf{Coppia Finale Forzata:}
    Per garantire che le stringhe concatenate possano eguagliarsi alla fine:
    \begin{itemize}
        \item $C_{k+1} = \$$ (un singolo simbolo dollaro)
        \item $D_{k+1} = *\$$ (asterisco seguito da dollaro)
    \end{itemize}
    Questo assicura che se le stringhe si allineano in modo corretto con gli asterischi e poi matchano con il dollaro, allora la soluzione è stata trovata.
\end{enumerate}

\subsection{Esempio di Costruzione}
Partiamo dall'istanza PCP dell'Esempio iniziale, ma trattandola come un'istanza MPCP (dove l'indice 1 deve essere il primo):
$A = (A_1 = 1, A_2 = 10111, A_3 = 10)$
$B = (B_1 = 111, B_2 = 10, B_3 = 0)$

Applichiamo la riduzione per ottenere $(C, D)$:
\begin{itemize}
    \item \textbf{Coppie trasformate (da indice 1 a 3):}
    $C_1 = 1*$
    $D_1 = *1*1*1$

    $C_2 = 1*0*1*1*1*$
    $D_2 = *1*0$

    $C_3 = 1*0*$
    $D_3 = *0$

    \item \textbf{Coppia Iniziale Forzata (indice 0):}
    $C_0 = *1*$
    $D_0 = *1*1*1$

    \item \textbf{Coppia Finale Forzata (indice 4):}
    $C_4 = \$$
    $D_4 = *\$$
\end{itemize}
L'istanza PCP risultante è $C=(C_0, C_1, C_2, C_3, C_4)$ e $D=(D_0, D_1, D_2, D_3, D_4)$.

\subsection{Dimostrazione}
\begin{proof_sketch}
Si deve dimostrare che l'istanza $(A, B)$ è "sì" per MPCP se e solo se l'istanza $(C, D)$ generata dalla riduzione è "sì" per PCP.

\paragraph{Parte 1: Se $(A, B)$ ha soluzione MPCP, allora $(C, D)$ ha soluzione PCP.}
Se $(A, B)$ ha una soluzione MPCP, esiste una sequenza di indici $I = (1, i_2, \dots, i_m)$ tale che $A_1 A_{i_2} \dots A_{i_m} = B_1 B_{i_2} \dots B_{i_m}$.
Costruiamo la soluzione PCP come segue: $I' = (0, 1, i_2, \dots, i_m, k+1)$.
La stringa $C_{concat}$ sarà $C_0 C_1 C_{i_2} \dots C_{i_m} C_{k+1}$.
La stringa $D_{concat}$ sarà $D_0 D_1 D_{i_2} \dots D_{i_m} D_{k+1}$.

Sostituendo le definizioni:
$C_{concat} = (* A_1') (A_1' \text{ con } * \text{dopo}) (A_{i_2}' \text{ con } * \text{dopo}) \dots (A_{i_m}' \text{ con } * \text{dopo}) (\$)$
$D_{concat} = (B_1' \text{ con } * \text{prima}) (B_1' \text{ con } * \text{prima}) (B_{i_2}' \text{ con } * \text{prima}) \dots (B_{i_m}' \text{ con } * \text{prima}) (*\$)$

Se $A_1 A_{i_2} \dots A_{i_m} = B_1 B_{i_2} \dots B_{i_m}$, allora è facile vedere che:
$A_1' A_{i_2}' \dots A_{i_m}' \text{ con asterischi finali}$
sarà uguale a
$B_1' B_{i_2}' \dots B_{i_m}' \text{ con asterischi iniziali}$.
L'aggiunta iniziale di $C_0 = *A_1'$ e $D_0 = B_1'$ (dove $B_1'$ ha l'asterisco iniziale) fa sì che il primo asterisco di $C_0$ si allinei con il primo asterisco di $D_0$. Poi, la corrispondenza simbolo-asterisco e asterisco-simbolo si mantiene. L'aggiunta finale di $\$$ e $*\$$ permette di concludere il match.

\paragraph{Parte 2: Se $(C, D)$ ha soluzione PCP, allora $(A, B)$ ha soluzione MPCP.}
Supponiamo che $(C, D)$ abbia una soluzione PCP, ovvero una sequenza di indici $I' = (j_1, j_2, \dots, j_p)$ tale che $C_{j_1} C_{j_2} \dots C_{j_p} = D_{j_1} D_{j_2} \dots D_{j_p}$.
\begin{itemize}
    \item \textbf{Forzatura dell'inizio:} L'unica stringa in $C$ che inizia con un asterisco è $C_0$. Tutte le stringhe in $D$ (tranne $D_{k+1}=*\$$) iniziano con un asterisco. Pertanto, per un match, la prima coppia usata deve essere $(C_0, D_0)$. Questo garantisce che $j_1 = 0$.
    \item \textbf{Allineamento degli asterischi:} Dopo l'inizio con $(C_0, D_0)$, tutte le stringhe di $C_i$ (per $i>0$) terminano con un asterisco, e tutte le stringhe di $D_i$ iniziano con un asterisco. Questo significa che un $C_i$ e un $D_j$ possono iniziare a corrispondere solo se l'ultimo carattere del precedente segmento di $C$ è un simbolo non-asterisco, e il primo carattere del successivo segmento di $D$ è un asterisco. Questo meccanismo di "passo-passo" impone che la sequenza di indici debba essere tale da mantenere una perfetta alternanza di simboli normali e asterischi per tutta la stringa.
    \item \textbf{Forzatura della fine:} Per terminare la stringa con un match, deve essere usata la coppia $(C_{k+1}, D_{k+1}) = (\$, *\$)$. Questo assicura che il match si concluda con un dollaro.
\end{itemize}
Data la struttura della riduzione, se le stringhe concatenate di $C$ e $D$ sono uguali, significa che la sequenza originale di $A$ (senza asterischi e dollari) deve essere uguale alla sequenza originale di $B$ (senza asterischi e dollari).
Poiché la soluzione PCP deve iniziare con $(C_0, D_0)$, questo implica che la soluzione MPCP inizia con la coppia $(A_1, B_1)$, soddisfacendo il requisito di MPCP.
Dunque, esiste una soluzione per l'istanza MPCP.
\end{proof_sketch}

Poiché $MPCP$ è indecidibile e $MPCP \le_m PCP$, concludiamo che $PCP$ è \textbf{indecidibile}.

\end{document}