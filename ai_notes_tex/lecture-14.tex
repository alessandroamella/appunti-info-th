\documentclass[a4paper]{article}
\usepackage{amsmath, amssymb, amsfonts}
\usepackage[italian]{babel}
\usepackage[utf8]{inputenc}
\usepackage{graphicx}
\usepackage{hyperref}
\usepackage[scaled]{helvet}
\renewcommand{\familydefault}{\sfdefault}
\usepackage[T1]{fontenc}
\usepackage[a4paper, total={6in, 8in}]{geometry}
\usepackage{minted}
\usepackage{fancyhdr}
\usepackage{amsthm}
\usepackage{tikz}
\usetikzlibrary{
    automata,
    positioning,
    arrows,
    arrows.meta,
    calc,
    fit,
    patterns,
    graphs,
    graphdrawing,
    matrix,
    quotes,
    shapes.geometric,
    snakes,
    shapes,
    decorations.pathreplacing,
    calligraphy,
    chains,
}
% Definizione degli ambienti
\theoremstyle{definition} % Use definition style for normal (non-italic) text
\newtheorem{theorem}{Teorema}
\newtheorem{definition}{Definizione}
\newtheorem{example}{Esempio}
\newtheorem{lemma}{Lemma}
\newtheorem{proposition}{Proposizione}
\newcommand{\blankS}{\ensuremath{\raisebox{-0.15ex}{\scalebox{1.3}[0.7]{$\sqcup$}}\mkern2mu}}
\newtheorem{proof_sketch}{Sketch di Dimostrazione} % Adding a sketch environment for complex proofs
\theoremstyle{remark} % Cambio lo stile per le osservazioni
\newtheorem{remark}{Osservazione}

\hypersetup{
    pdftitle={Lezione di Informatica Teorica: Riduzioni e Indecidibilità del PCP},
    pdfauthor={Appunti da Trascrizione AI}
}

\pagestyle{fancy}
\fancyhf{}
\fancyhead[L]{\textit{Lezione di Informatica Teorica}}
\fancyfoot[C]{\thepage}

\title{Lezione di Informatica Teorica: Riduzioni e Indecidibilità del PCP}
\author{Appunti da Trascrizione Automatica}
\date{\today}

\begin{document}
\maketitle
\tableofcontents
\newpage

\section{Recap sulle Riduzioni}

\begin{example}[Esempio di Riduzione]
Consideriamo due problemi semplici per illustrare il concetto:
\begin{itemize}
    \item \textbf{Problema A (EVEN):} Data una stringa binaria $w$, decidere se il numero di simboli $1$ in $w$ è pari.
    \item \textbf{Problema B (ODD):} Data una stringa binaria $w$, decidere se il numero di simboli $1$ in $w$ è dispari.
\end{itemize}

Possiamo definire una riduzione $f$ da EVEN a ODD come segue:
$$f(w) = w1$$
cioè $f$ aggiunge un simbolo $1$ alla fine della stringa $w$.

\textbf{Verifichiamo che sia una riduzione valida:}
\begin{enumerate}
    \item $f$ è chiaramente calcolabile: basta concatenare $1$ alla fine di $w$.
    \item Dimostriamo che $w \in \text{EVEN} \iff f(w) \in \text{ODD}$:
    \begin{itemize}
        \item Se $w$ ha un numero pari di $1$, allora $f(w) = w1$ ha un numero dispari di $1$ (pari + 1 = dispari). Quindi $w \in \text{EVEN} \implies f(w) \in \text{ODD}$.
        \item Se $w$ ha un numero dispari di $1$, allora $f(w) = w1$ ha un numero pari di $1$ (dispari + 1 = pari). Quindi $w \notin \text{EVEN} \implies f(w) \notin \text{ODD}$.
    \end{itemize}
\end{enumerate}

Quindi $\text{EVEN} \le_m \text{ODD}$. Se sapessimo che EVEN è indecidibile (ipotesi), allora per il teorema che seguirà, anche ODD sarebbe indecidibile.
\end{example}

\begin{remark}
Il significato di $w \in A \iff f(w) \in B$ è duplice:
\begin{itemize}
    \item Se $w \in A$, allora $f(w) \in B$.
    \item Se $w \notin A$, allora $f(w) \notin B$.
\end{itemize}
\end{remark}

\subsection{Scopo delle Riduzioni}

Le riduzioni sono utilizzate per dimostrare l'indecidibilità di problemi.
\begin{theorem}
Se $A \le_m B$ e $A$ è un problema indecidibile, allora anche $B$ è un problema indecidibile.
\end{theorem}
\begin{proof}
Supponiamo per contraddizione che $B$ sia decidibile. Allora esisterebbe una Macchina di Turing $M_B$ che decide $B$.
Poiché $f$ è calcolabile, possiamo costruire una Macchina di Turing $M_A$ che decide $A$ come segue:
\begin{enumerate}
    \item Input $w$.
    \item Calcola $f(w)$.
    \item Simula $M_B$ su $f(w)$.
    \item Accetta se $M_B$ accetta, rifiuta se $M_B$ rifiuta.
\end{enumerate}
Per la definizione di riduzione, $w \in A \iff f(w) \in B$. Quindi $M_A$ decide $A$. Ma questo contraddice l'ipotesi che $A$ sia indecidibile. Dunque, $B$ deve essere indecidibile.
\end{proof}

\begin{theorem}[Transitività delle Riduzioni]
Se $A \le_m B$ e $B \le_m C$, allora $A \le_m C$.
\end{theorem}
\begin{proof}
Siano $f$ la riduzione da $A$ a $B$ e $g$ la riduzione da $B$ a $C$.
Entrambe $f$ e $g$ sono calcolabili. Allora la composizione $g \circ f$ è anch'essa calcolabile.
Inoltre, $w \in A \iff f(w) \in B \iff g(f(w)) \in C$.
Quindi $g \circ f$ è una riduzione calcolabile da $A$ a $C$.
\end{proof}

\section{Problema di Corrispondenza di Post (PCP)}

Il Problema di Corrispondenza di Post (PCP) è un problema di decisione sulle stringhe.

\begin{definition}[PCP]
Un'istanza del PCP è data da due liste non-vuote di stringhe, $A = (A_1, A_2, \dots, A_k)$ e $B = (B_1, B_2, \dots, B_k)$, definite su un alfabeto $\Sigma$.
Il problema chiede se esista una sequenza di indici non-vuota $I = (i_1, i_2, \dots, i_m)$ con $m \ge 1$ e $1 \le i_j \le k$ per ogni $j$, tale che:
$$A_{i_1} A_{i_2} \dots A_{i_m} = B_{i_1} B_{i_2} \dots B_{i_m}$$
La risposta è "sì" se una tale sequenza esiste, "no" altrimenti.
\end{definition}

\vspace{0.5em}

\begin{example}

\begin{align*}
A &= (A_1 = 1,\; A_2 = 10111,\; A_3 = 10) \\
B &= (B_1 = 111,\; B_2 = 10,\; B_3 = 0)
\end{align*}

Possiamo rappresentare un'istanza del PCP utilizzando tessere (domini).

\begin{center}
% Domino representation (TikZ)
\begin{tikzpicture}[x=2.5cm, y=1.2cm, font=\sffamily]
  % Dominoes
  % Domino 1
  \draw (0,0) rectangle (1,1);
  \draw (0,0.5) -- (1,0.5);
  \node at (0.5,0.75) {$1$};
  \node at (0.5,0.25) {$111$};
  \node at (0.5,-0.3) {$A_1 \mid B_1$};
  % Domino 2
  \draw (1.2,0) rectangle (2.2,1);
  \draw (1.2,0.5) -- (2.2,0.5);
  \node at (1.7,0.75) {$10111$};
  \node at (1.7,0.25) {$10$};
  \node at (1.7,-0.3) {$A_2 \mid B_2$};
  % Domino 3
  \draw (2.4,0) rectangle (3.4,1);
  \draw (2.4,0.5) -- (3.4,0.5);
  \node at (2.9,0.75) {$10$};
  \node at (2.9,0.25) {$0$};
  \node at (2.9,-0.3) {$A_3 \mid B_3$};
\end{tikzpicture}
\end{center}

Ogni tessera ha una parte superiore ($A_i$) e una inferiore ($B_i$).

\textbf{Domanda:} Esiste una sequenza di indici $I = (i_1, i_2, \dots, i_m)$ tale che, concatenando le parti superiori e inferiori delle tessere scelte, si ottengano due stringhe uguali?

\textbf{Esempio di soluzione:}
Scegliamo la sequenza di indici $I = (2, 1, 1, 3)$.

\begin{center}
% Visual domino sequence (TikZ)
\begin{tikzpicture}[x=1.7cm, y=1.2cm, font=\sffamily]
  % Sequence: 2, 1, 1, 3
  % Domino 2
  \draw[thick] (0,0) rectangle (1,1);
  \draw (0,0.5) -- (1,0.5);
  \node at (0.5,0.75) {$10111$};
  \node at (0.5,0.25) {$10$};
  \node at (0.5,-0.3) {$A_2 \mid B_2$};
  % Domino 1
  \draw[thick] (1.1,0) rectangle (2.1,1);
  \draw (1.1,0.5) -- (2.1,0.5);
  \node at (1.6,0.75) {$1$};
  \node at (1.6,0.25) {$111$};
  \node at (1.6,-0.3) {$A_1 \mid B_1$};
  % Domino 1 again
  \draw[thick] (2.2,0) rectangle (3.2,1);
  \draw (2.2,0.5) -- (3.2,0.5);
  \node at (2.7,0.75) {$1$};
  \node at (2.7,0.25) {$111$};
  \node at (2.7,-0.3) {$A_1 \mid B_1$};
  % Domino 3
  \draw[thick] (3.3,0) rectangle (4.3,1);
  \draw (3.3,0.5) -- (4.3,0.5);
  \node at (3.8,0.75) {$10$};
  \node at (3.8,0.25) {$0$};
  \node at (3.8,-0.3) {$A_3 \mid B_3$};
\end{tikzpicture}
\end{center}

\textbf{Costruzione delle stringhe:}

\begin{align*}
  \text{Parte superiore (A):}\quad & 10111\ 1\ 1\ 10 = \textbf{101111110} \\
  \text{Parte inferiore (B):}\quad & 10\ 111\ 111\ 0 = \textbf{101111110}
\end{align*}

Poiché le due stringhe coincidono, questa sequenza è una soluzione per l'istanza PCP.

\textbf{Nota visiva:} Puoi pensare a ciascuna tessera come a un ``domino'' con la parte superiore e inferiore, e l'obiettivo è allineare le tessere in modo che la concatenazione delle parti superiori e inferiori dia la stessa stringa. 
\textbf{Le tessere possono essere usate anche più volte, cioè con ripetizione.}

\end{example}

\subsection{Proprietà del PCP}
\begin{itemize}
\item Il PCP è un problema \textbf{riconoscibile} (in RE). Possiamo ideare un algoritmo che, data un'istanza, cerca tutte le possibili sequenze di indici di lunghezza crescente. Se trova una soluzione, accetta. Se non trova una soluzione in un tempo finito, potrebbe continuare a cercare indefinitamente.
\item Il PCP è \textbf{indecidibile} (non in R). Non esiste un algoritmo che, per ogni istanza del PCP, termina e fornisce la risposta corretta ("sì" o "no"). Dimostreremo questo usando le riduzioni.
\end{itemize}

\textbf{Nota terminologica:}
\begin{itemize}
\item \textbf{R} (ricorsivo) = insieme dei linguaggi \textbf{decidibili}
\item \textbf{RE} (ricorsivamente enumerabile) = insieme dei linguaggi \textbf{riconoscibili} o \textbf{semi-decidibili}
\end{itemize}

Il PCP appartiene alla classe RE $\setminus$ R, ovvero è riconoscibile ma non decidibile.

\section{Problema di Corrispondenza di Post Modificato (MPCP)}

Per dimostrare l'indecidibilità del PCP, useremo un problema intermedio, il MPCP.

\begin{definition}[MPCP]
Un'istanza del MPCP è data da due liste non-vuote di stringhe, $A = (A_1, A_2, \dots, A_k)$ e $B = (B_1, B_2, \dots, B_k)$, definite su un alfabeto $\Sigma$.
Il problema chiede se esista una sequenza di indici $I = (i_1, i_2, \dots, i_m)$ con $m \ge 1$ e $1 \le i_j \le k$ per ogni $j$, tale che:
\begin{enumerate}
    \item La sequenza deve iniziare con l'indice $1$: $i_1 = 1$.
    \item $A_{i_1} A_{i_2} \dots A_{i_m} = B_{i_1} B_{i_2} \dots B_{i_m}$
\end{enumerate}
La risposta è "sì" se una tale sequenza esiste, "no" altrimenti.
\end{definition}

\begin{remark}
L'unica differenza tra PCP e MPCP è la condizione che la sequenza di indici debba iniziare per $1$.
\end{remark}

\section{Riduzione 1: $LU \le_m MPCP$}

Dimostriamo che il Linguaggio Universale ($LU$) si riduce al MPCP. Dato che $LU$ è indecidibile, questo implicherà che MPCP è indecidibile.

% --- Recap: Linguaggio Universale (LU) ---
\noindent\rule{\textwidth}{0.4pt}
\textbf{Recap: Linguaggio Universale (LU)}

Il linguaggio universale $LU$ è l'insieme delle coppie $\langle M, w \rangle$, dove $M$ è una Macchina di Turing e $w$ è una stringa, tali che $M$ accetta $w$ (cioè si arresta in uno stato accettante su $w$):
\[
LU = \{ \langle M, w \rangle \mid M \text{ accetta } w \}
\]
$LU$ è \textbf{riconoscibile} ($LU \in RE$), perché esiste una Macchina di Turing Universale che simula $M$ su $w$ e accetta se $M$ accetta $w$. Tuttavia, $LU$ \textbf{non è decidibile} ($LU \notin R$), perché non esiste un algoritmo che, per ogni $\langle M, w \rangle$, stabilisca sempre in tempo finito se $M$ accetta $w$ (problema dell'arresto).
\noindent\rule{\textwidth}{0.4pt}

\subsection{Idea della Riduzione}

\paragraph{Setup della Riduzione}
\begin{itemize}
    \item \textbf{Input di $LU$:} Una coppia $(M, w)$, dove $M$ è una Macchina di Turing e $w$ è una stringa.
    \item \textbf{Output di MPCP:} Una coppia di liste di stringhe $(A, B)$.
    \item \textbf{Funzione di riduzione:} $f$ deve trasformare $(M, w)$ in $(A, B)$.
\end{itemize}

\paragraph{Strategia di Simulazione}
L'idea centrale è \textbf{simulare la computazione di $M$ su $w$} usando le stringhe delle liste $A$ e $B$. 

La computazione di una Macchina di Turing può essere rappresentata come una sequenza di \emph{descrizioni istantanee} (ID) separate da un simbolo speciale:
$$ID_1 \# ID_2 \# ID_3 \# \dots \# ID_k$$

dove ogni $ID_i$ rappresenta lo stato della macchina, la posizione della testina e il contenuto del nastro in un determinato momento.

\begin{example}[Esempio di ID]
Una Macchina di Turing con nastro $011$, testina sul primo $1$ e stato $q_2$ ha ID:
\[
0\,q_2\,11
\]
\vspace{-1em}
\begin{center}
\begin{tikzpicture}[
    tape/.style={rectangle, draw, minimum width=1cm, minimum height=1.2cm, font=\large},
    head/.style={draw, thick, fill=yellow!30, minimum width=1cm, minimum height=1.2cm},
    state/.style={font=\sffamily\bfseries}
]
    \node[tape] (cell1) at (0,0) {$0$};
    \node[head] (cell2) at (1,0) {\shortstack{$\mathbf{q_2}$\\$1$}};
    \node[tape] (cell3) at (2,0) {$1$};
    \node[tape] (cell4) at (3,0) {$\blankS$};
    \draw[thick,->] (1,1) -- (1,0.6);
    \node[state] at (1,1.3) {Testina};
    \node at (-1,0) {$\cdots$};
    \node at (4,0) {$\cdots$};
\end{tikzpicture}
\end{center}

Se la computazione produce le seguenti configurazioni:
\[
ID_1 = q_0 01,\quad
ID_2 = 1q_2 1,\quad
ID_3 = 10q_1,\quad
ID_4 = 1q_2 1,\quad
ID_5 = 10q_3
\]
allora la sequenza delle ID separate da \# è:
\[
q_0 01 \# 1q_2 1 \# 10q_1 \# 1q_2 1 \# 10q_3
\]
\end{example}

\paragraph{Meccanismo di Sincronizzazione}
Vogliamo costruire le liste $A$ e $B$ in modo tale che:
\begin{enumerate}
    \item Se $M$ accetta $w$, allora esista una sequenza di indici per MPCP che costruisce la stessa stringa.
    \item Le stringhe di $B$ saranno sempre \textbf{"un passo avanti"} rispetto alle stringhe di $A$, simulando la prossima configurazione istantanea.
    \item Quando $M$ raggiunge uno stato accettante, le stringhe di $A$ avranno la possibilità di \textbf{"recuperare"} e allinearsi con quelle di $B$.
\end{enumerate}

\subsection{Costruzione delle Liste $A$ e $B$}
L'alfabeto del MPCP sarà $\Gamma \cup Q \cup \{\#, \$, *\}$, dove $\Gamma$ è l'alfabeto del nastro di $M$, $Q$ è l'insieme degli stati di $M$, e $\#, \$, *$ sono nuovi simboli.
Assumiamo che $M$ non scriva mai il simbolo blank e che il suo nastro sia semi-infinito a destra (non si muova mai a sinistra della posizione iniziale). Queste sono assunzioni standard che non limitano la generalità delle TM.

Le liste $A$ e $B$ sono costruite con le seguenti classi di coppie di stringhe $(A_i, B_i)$:

\begin{enumerate}
    \item \textbf{Coppia Iniziale (obbligatoria per MPCP):}
    Questa coppia inizia la simulazione e garantisce che $B$ sia un passo avanti.
    \begin{itemize}
        \item $A_1 = \#$
        \item $B_1 = \# q_0 w \#$ (dove $q_0$ è lo stato iniziale di $M$, $w$ è la stringa d'input, e $\#$ è un simbolo di confine).
    \end{itemize}

    \item \textbf{Coppie di Copia:}
    Permettono di copiare simboli di configurazione che non sono sotto la testina.
    Per ogni simbolo $x \in \Gamma \cup \{\#\}$:
    \begin{itemize}
        \item $A_i = x$
        \item $B_i = x$
    \end{itemize}

    \item \textbf{Coppie di Transizione:}
    Simulano il movimento della testina e la modifica del nastro secondo le regole di transizione di $M$. Queste regole sono generate per ogni $q \in Q \setminus F$ (stato non finale) e ogni $X \in \Gamma \cup \{\#\}$.
    \begin{itemize}
        \item \textbf{Spostamento a destra ($R$):} Se $\delta(q, X) = (p, Y, R)$:
            \begin{itemize}
                \item $A_i = qX$
                \item $B_i = Yp$
            \end{itemize}
            (Esempio: se $M$ legge $X$ nello stato $q$, scrive $Y$ e va nello stato $p$ muovendosi a destra, allora $qX$ in $A$ corrisponde a $Yp$ in $B$).
            \emph{Caso speciale: $X$ è un blank (simbolo di bordo \texttt{\#}).} Se $\delta(q, \#) = (p, Y, R)$:
            \begin{itemize}
                \item $A_i = q\#$
                \item $B_i = Yp\#$
            \end{itemize}
        \item \textbf{Spostamento a sinistra ($L$):} Se $\delta(q, X) = (p, Y, L)$:
            \begin{itemize}
                \item $A_i = ZqX$ (per ogni $Z \in \Gamma \cup \{\#\}$)
                \item $B_i = pZY$
            \end{itemize}
            (Esempio: se $M$ legge $X$ nello stato $q$, scrive $Y$ e va nello stato $p$ muovendosi a sinistra, allora la sequenza $ZqX$ in $A$ (dove $Z$ è il simbolo a sinistra di $q$) corrisponde a $pZY$ in $B$).
    \end{itemize}

    \item \textbf{Coppie di Accettazione/Recupero:}
    Queste coppie sono utilizzate solo quando $M$ entra in uno stato accettante $q_f \in F$. Permettono alla stringa concatenata di $A$ di "recuperare" la stringa concatenata di $B$.
    Per ogni $q_f \in F$ e ogni $X, Y \in \Gamma$:
    \begin{itemize}
        \item $A_i = Xq_f Y$
        \item $B_i = q_f Y$
    \end{itemize}
    Questo riduce la differenza di lunghezza tra $A_i$ e $B_i$ per $q_f$ e $Y$.
    \begin{itemize}
        \item $A_i = Xq_f \#$
        \item $B_i = q_f \#$
    \end{itemize}
    \begin{itemize}
        \item $A_i = q_f Y \#$
        \item $B_i = q_f \#$
    \end{itemize}
    \begin{itemize}
        \item $A_i = q_f \# \#$
        \item $B_i = q_f \#$
    \end{itemize}
    % (Il professore ha menzionato $XQY \to QY$ come regola generale per $Q \in F$. Il principio è che la stringa di A diventa più lunga o uguale, permettendo di chiudere la partita.)

    \item \textbf{Coppia Finale:}
    Permette di completare il match una volta che la TM è in uno stato accettante e la differenza di lunghezza è stata recuperata.
    Per ogni $q_f \in F$:
    \begin{itemize}
        \item $A_i = q_f \# \#$ (due simboli \texttt{\#} per A)
        \item $B_i = \# \#$ (due simboli \texttt{\#} per B)
    \end{itemize}
\end{enumerate}

\subsection{Esempio di Simulazione}

Consideriamo la Macchina di Turing $M$ definita come segue:

\begin{itemize}
    \item \textbf{Stati:} $Q = \{q_1, q_2, q_3\}$ (dove $q_1$ è lo stato iniziale e $q_3$ è lo stato finale)
    \item \textbf{Alfabeto del nastro:} $\Gamma = \{0, 1, \beta\}$ (dove $\beta$ è il simbolo blank)
    \item \textbf{Funzione di transizione:}
    \begin{align}
        \delta(q_1, 0) &= (q_2, 1, R) \\
        \delta(q_1, 1) &= (q_1, 0, L) \\
        \delta(q_2, 1) &= (q_1, 0, R) \\
        \delta(q_1, \beta) &= (q_2, 1, L) \\
        \delta(q_2, \beta) &= (q_3, 0, R)
    \end{align}
\end{itemize}

\begin{figure}[h]
\centering
\begin{tikzpicture}[node distance=5cm, auto, thick]
    % Stati
    \node[state, initial] (q1) {$q_1$};
    \node[state, right of=q1] (q2) {$q_2$};
    \node[state, accepting, right of=q2] (q3) {$q_3$};
    
    % Transizioni
    \path[->]
        (q1) edge[bend left] node[above] {$\begin{array}{c}0 / 1, R \\ \beta / 1, L\end{array}$} (q2)
        (q1) edge[loop above] node[above] {$1 / 0, L$} (q1)
        (q2) edge[bend left] node[below] {$1 / 0, R$} (q1)
        (q2) edge node[above] {$\beta / 0, R$} (q3);
\end{tikzpicture}
\caption{Diagramma della Macchina di Turing $M$}
\end{figure}

Consideriamo la stringa d'input $w = 01$. La computazione di $M$ su $w$ procede come segue:

\begin{enumerate}
    \item $\#q_1 01\#$ \quad (Configurazione iniziale)
    \item $\#1q_2 1\#$ \quad (Da $q_1$ legge $0$, scrive $1$, va in $q_2$, si sposta a destra)
    \item $\#10q_1 \#$ \quad (Da $q_2$ legge $1$, scrive $0$, va in $q_1$, si sposta a destra)
    \item $\#1q_2 1\#$ \quad (Da $q_1$ legge $\beta$, scrive $1$, va in $q_2$, si sposta a sinistra)
    \item $\#10q_3 \#$ \quad (Da $q_2$ legge $\beta$, scrive $0$, va in $q_3$, si sposta a destra)
\end{enumerate}

La computazione termina nello stato accettante $q_3$, quindi $M$ accetta la stringa $w = 01$.

\paragraph{Come la riduzione costruisce la soluzione MPCP:}
\begin{itemize}
    \item Si inizia con la coppia iniziale: $A_1 = \#$, $B_1 = \#q_1 01\#$.
        La stringa concatenata di A (finora $\#$) è "dietro" quella di B (finora $\#q_1 01\#$).
    \item Per simulare $\#q_1 01\# \to \#1q_2 1\#$:
        \begin{itemize}
            \item Usiamo coppie di copia per $\#$ iniziale: $(A_i = \#, B_i = \#)$.
            \item Per la transizione $q_1 0 \to 1q_2$: usiamo la coppia $(A_i = q_1 0, B_i = 1q_2)$.
            \item Usiamo coppie di copia per $1\#$: $(A_i=1, B_i=1)$ e $(A_i=\#, B_i=\#)$.
        \end{itemize}
        A questo punto, la stringa concatenata di A sarà $\#q_1 01\#$, mentre quella di B sarà $\#q_1 01\# \#1q_2 1\#$. B è ancora un passo avanti.
    \item Questo processo continua. Le regole di copia e transizione assicurano che la stringa costruita da $B$ contenga la sequenza di ID, con la stringa costruita da $A$ che la segue con un ID di ritardo.
    \item Quando la computazione raggiunge uno stato finale (es. $q_3$ nell'esempio, $ID_5 = \#10q_3\#$), le coppie di accettazione (Classe 4) permettono alla stringa di $A$ di "catturare" la stringa di $B$. Ad esempio, per $Xq_f Y \to QY$, $A$ contribuisce con 3 simboli e $B$ con 2, riducendo il gap.
    \item Infine, la coppia finale (Classe 5) permette di pareggiare completamente le stringhe. Se $A$ arriva a $\#ID_1\#ID_2\# \dots \#ID_k \#q_f \# \#$ e $B$ arriva a $\#ID_1\#ID_2\# \dots \#ID_k \# \# \#$, allora la coppia finale farà sì che $A$ e $B$ diventino uguali.
\end{itemize}

\subsection{Dimostrazione}
\begin{proof_sketch}
Si deve dimostrare che $(M, w)$ è un'istanza "sì" di $LU$ se e solo se l'istanza $(A, B)$ generata dalla riduzione è un'istanza "sì" di MPCP.

\paragraph{Parte 1: Se $M$ accetta $w$, allora l'istanza MPCP ha soluzione.}

\begin{itemize}
    \item \emph{Premessa:} Se $M$ accetta $w$, significa che esiste una sequenza di configurazioni $ID_1, ID_2, \dots, ID_k$ dove $ID_1 = q_0 w$, e $ID_k$ è una configurazione accettante.
    
    \item \emph{Costruzione della sequenza di indici:}
    \begin{enumerate}
        \item Iniziamo sempre con l'indice $1$ (la coppia iniziale). Questo produce $A_{concat} = \#$ e $B_{concat} = \#ID_1\#$.
        \item Poi, si scelgono le coppie di stringhe $(A_i, B_i)$ dalle classi 2 e 3 che simulano la transizione da $ID_j$ a $ID_{j+1}$. 
        \item Ogni volta che si concatena una coppia di transizione, la stringa di $A$ "completa" la $ID_j$ e la stringa di $B$ "inizia" la $ID_{j+1}$.
    \end{enumerate}
    
    \item \emph{Risultato:} Questo fa sì che $A_{concat}$ diventi:
    \[
        \#ID_1\#ID_2\# \dots \#ID_{k-1}\#
    \]
    e $B_{concat}$ diventi:
    \[
        \#ID_1\#ID_2\# \dots \#ID_k\#.
    \]

    \item \emph{Fase di recupero:} Poiché $ID_k$ è una configurazione accettante, possiamo usare le coppie di Classe 4 per "rimuovere" i simboli vicino allo stato finale in $A$ e $B$ in modo disuguale, riducendo il divario di lunghezza. Infine, la coppia di Classe 5 permette di far terminare le stringhe con esattamente gli stessi simboli, facendo sì che $A_{concat} = B_{concat}$.
    
    \item \emph{Conclusione:} Esiste una soluzione per l'istanza MPCP.
\end{itemize}

\paragraph{Parte 2: Se l'istanza MPCP ha soluzione, allora $M$ accetta $w$.}

\begin{itemize}
    \item \emph{Premessa:} Supponiamo che l'istanza MPCP $(A, B)$ generata dalla riduzione abbia una soluzione.
    
    \item \emph{Vincolo iniziale:} Per la definizione di MPCP, questa soluzione deve iniziare con la coppia di indici $(1, i_2, \dots, i_m)$, dove $i_2, \dots, i_m$ sono indici arbitrari.
    
    \item \emph{Necessità di simulazione corretta:} L'unico modo per le stringhe concatenate di $A$ e $B$ di rimanere allineate e infine eguagliarsi è che le coppie usate simulino correttamente le transizioni di $M$. Qualsiasi sequenza di indici che non rispetti la simulazione delle transizioni non permetterebbe mai alle stringhe di allinearsi correttamente.
    
    \item \emph{Meccanismo di recupero:} Poiché le stringhe di $B$ sono sempre un passo avanti (contengono $\#ID_j\#ID_{j+1}\#$ mentre $A$ ha solo $\#ID_j\#$), l'unico modo per $A$ di "recuperare" e far sì che le stringhe concatenate di $A$ e $B$ diventino uguali è tramite l'uso delle coppie di Classe 4 (Accettazione/Recupero) e Classe 5 (Finale).
    
    \item \emph{Conclusione:} Le coppie di Classe 4 e 5 possono essere utilizzate solo se la configurazione attuale di $M$ (simulata) contiene uno stato accettante. Pertanto, se esiste una soluzione MPCP, la simulazione deve aver raggiunto una configurazione accettante, il che significa che $M$ accetta $w$.
\end{itemize}
\end{proof_sketch}

Poiché $LU$ è indecidibile e $LU \le_m MPCP$, concludiamo che $MPCP$ è \textbf{indecidibile}.

\section{Riduzione 2: $MPCP \le_m PCP$}

Ora dimostriamo che $MPCP$ si riduce al $PCP$. Dato che $MPCP$ è indecidibile, questo implicherà che $PCP$ è indecidibile.

\subsection{Idea della Riduzione}
Un'istanza di MPCP è una coppia di liste $(A, B)$. Un'istanza di PCP è una coppia di liste $(C, D)$.
La funzione di riduzione $f$ deve trasformare $(A, B)$ in $(C, D)$.
La sfida è che PCP non richiede che la soluzione inizi con un indice specifico, mentre MPCP sì (indice 1). Dobbiamo modificare le liste $(A, B)$ in $(C, D)$ in modo che qualsiasi soluzione PCP debba iniziare con la coppia modificata dall'indice 1 dell'MPCP.

\subsection{Costruzione delle Liste $C$ e $D$}
Introduciamo due nuovi simboli non presenti nell'alfabeto originale: $*$ (asterisco) e $\$$ (dollaro).

Le liste $C$ e $D$ sono costruite come segue:

\begin{enumerate}
    \item \textbf{Trasformazione delle Coppie Originali:}
    Per ogni coppia $(A_i, B_i)$ con $i \in \{1, \dots, k\}$:
    \begin{itemize}
        \item $C_i$: Ogni simbolo di $A_i$ è seguito da un asterisco.
              Esempio: se $A_i = s_1 s_2 \dots s_m$, allora $C_i = s_1 * s_2 * \dots s_m *$.
        \item $D_i$: Ogni simbolo di $B_i$ è preceduto da un asterisco.
              Esempio: se $B_i = t_1 t_2 \dots t_n$, allora $D_i = * t_1 * t_2 \dots * t_n$.
    \end{itemize}

    \item \textbf{Coppia Iniziale Forzata:}
    Per forzare la sequenza a iniziare con l'equivalente dell'indice 1 di MPCP:
    \begin{itemize}
        \item $C_0 = * C_1$ (la stringa $C_1$ (ottenuta dal $A_1$ originale) con un asterisco aggiunto all'inizio).
              Esempio: se $A_1 = 1$, $C_1 = 1*$. Allora $C_0 = *1*$.
        \item $D_0 = D_1$ (la stringa $D_1$ (ottenuta dal $B_1$ originale)).
              Esempio: se $B_1 = 111$, $D_1 = *1*1*1$. Allora $D_0 = *1*1*1$.
    \end{itemize}
    Nota: Tutte le stringhe in $C_i$ (per $i>0$) iniziano con un simbolo dell'alfabeto originale seguito da $^*$, mentre tutte le stringhe in $D_i$ iniziano con $^*$. L'unica eccezione è $C_0$ che inizia con $^*$. Questo forza la scelta di $C_0$ e $D_0$ come prima coppia della soluzione PCP.

    \item \textbf{Coppia Finale Forzata:}
    Per garantire che le stringhe concatenate possano eguagliarsi alla fine:
    \begin{itemize}
        \item $C_{k+1} = \$$ (un singolo simbolo dollaro)
        \item $D_{k+1} = *\$$ (asterisco seguito da dollaro)
    \end{itemize}
    Questo assicura che se le stringhe si allineano in modo corretto con gli asterischi e poi matchano con il dollaro, allora la soluzione è stata trovata.
\end{enumerate}

\subsection{Esempio di Costruzione}
Partiamo dall'istanza PCP dell'Esempio iniziale, ma trattandola come un'istanza MPCP (dove l'indice 1 deve essere il primo):
$A = (A_1 = 1, A_2 = 10111, A_3 = 10)$
$B = (B_1 = 111, B_2 = 10, B_3 = 0)$

Applichiamo la riduzione per ottenere $(C, D)$:
\begin{itemize}
    \item \textbf{Coppie trasformate (da indice 1 a 3):}
    $C_1 = 1*$
    $D_1 = *1*1*1$

    $C_2 = 1*0*1*1*1*$
    $D_2 = *1*0$

    $C_3 = 1*0*$
    $D_3 = *0$

    \item \textbf{Coppia Iniziale Forzata (indice 0):}
    $C_0 = *1*$
    $D_0 = *1*1*1$

    \item \textbf{Coppia Finale Forzata (indice 4):}
    $C_4 = \$$
    $D_4 = *\$$
\end{itemize}
L'istanza PCP risultante è $C=(C_0, C_1, C_2, C_3, C_4)$ e $D=(D_0, D_1, D_2, D_3, D_4)$.

\subsection{Dimostrazione}
\begin{proof_sketch}
Si deve dimostrare che l'istanza $(A, B)$ è "sì" per MPCP se e solo se l'istanza $(C, D)$ generata dalla riduzione è "sì" per PCP.

\paragraph{Parte 1: Se $(A, B)$ ha soluzione MPCP, allora $(C, D)$ ha soluzione PCP.}

\begin{itemize}
    \item \emph{Premessa:} Se $(A, B)$ ha una soluzione MPCP, esiste una sequenza di indici $I = (1, i_2, \dots, i_m)$ tale che $A_1 A_{i_2} \dots A_{i_m} = B_1 B_{i_2} \dots B_{i_m}$.
    
    \item \emph{Costruzione della soluzione PCP:} Costruiamo la soluzione PCP come segue: $I' = (0, 1, i_2, \dots, i_m, k+1)$.
    
    \item \emph{Stringhe concatenate:} Le stringhe risultanti saranno:
    \begin{align}
        C_{concat} &= C_0 C_1 C_{i_2} \dots C_{i_m} C_{k+1} \\
        D_{concat} &= D_0 D_1 D_{i_2} \dots D_{i_m} D_{k+1}
    \end{align}
    
    \item \emph{Sostituzioni delle definizioni:}
    \begin{align}
        C_{concat} &= (* A_1') (A_1' \text{ con } * \text{dopo}) (A_{i_2}' \text{ con } * \text{dopo}) \dots (A_{i_m}' \text{ con } * \text{dopo}) (\$) \\
        D_{concat} &= (B_1' \text{ con } * \text{prima}) (B_1' \text{ con } * \text{prima}) (B_{i_2}' \text{ con } * \text{prima}) \dots (B_{i_m}' \text{ con } * \text{prima}) (*\$)
    \end{align}
    
    \item \emph{Correttezza dell'allineamento:} Se $A_1 A_{i_2} \dots A_{i_m} = B_1 B_{i_2} \dots B_{i_m}$, allora:
    \begin{itemize}
        \item $A_1' A_{i_2}' \dots A_{i_m}'$ con asterischi finali sarà uguale a $B_1' B_{i_2}' \dots B_{i_m}'$ con asterischi iniziali
        \item L'aggiunta iniziale di $C_0 = *A_1'$ e $D_0 = B_1'$ fa sì che il primo asterisco di $C_0$ si allinei con il primo asterisco di $D_0$
        \item La corrispondenza simbolo-asterisco e asterisco-simbolo si mantiene per tutta la sequenza
        \item L'aggiunta finale di $\$$ e $*\$$ permette di concludere il match
    \end{itemize}
\end{itemize}

\paragraph{Parte 2: Se $(C, D)$ ha soluzione PCP, allora $(A, B)$ ha soluzione MPCP.}

\begin{itemize}
    \item \emph{Premessa:} Supponiamo che $(C, D)$ abbia una soluzione PCP, ovvero una sequenza di indici $I' = (j_1, j_2, \dots, j_p)$ tale che $C_{j_1} C_{j_2} \dots C_{j_p} = D_{j_1} D_{j_2} \dots D_{j_p}$.
    
    \item \emph{Forzatura dell'inizio:} L'unica stringa in $C$ che inizia con un asterisco è $C_0$. Tutte le stringhe in $D$ (tranne $D_{k+1}=*\$$) iniziano con un asterisco. Pertanto, per un match, la prima coppia usata deve essere $(C_0, D_0)$. Questo garantisce che $j_1 = 0$.
    
    \item \emph{Allineamento degli asterischi:} Dopo l'inizio con $(C_0, D_0)$, tutte le stringhe di $C_i$ (per $i>0$) terminano con un asterisco, e tutte le stringhe di $D_i$ iniziano con un asterisco. Questo meccanismo di "passo-passo" impone che la sequenza di indici debba mantenere una perfetta alternanza di simboli normali e asterischi per tutta la stringa.
    
    \item \emph{Forzatura della fine:} Per terminare la stringa con un match, deve essere usata la coppia
    $$(C_{k+1}, D_{k+1}) = (\$, *\$).$$
    Questo assicura che il match si concluda con un dollaro.
    
    \item \emph{Corrispondenza con MPCP:} Data la struttura della riduzione, se le stringhe concatenate di $C$ e $D$ sono uguali, significa che la sequenza originale di $A$ (senza asterischi e dollari) deve essere uguale alla sequenza originale di $B$ (senza asterischi e dollari).
    
    \item \emph{Conclusione:} Poiché la soluzione PCP deve iniziare con $(C_0, D_0)$, questo implica che la soluzione MPCP inizia con la coppia $(A_1, B_1)$, soddisfacendo il requisito di MPCP. Dunque, esiste una soluzione per l'istanza MPCP.
\end{itemize}
\end{proof_sketch}

Poiché $MPCP$ è indecidibile e $MPCP \le_m PCP$, concludiamo che $PCP$ è \textbf{indecidibile}.

\end{document}