\documentclass[a4paper]{article}
\usepackage{amsmath, amssymb, amsfonts}
\usepackage[italian]{babel}
\usepackage[utf8]{inputenc}
\usepackage{graphicx}
\usepackage{hyperref}
\usepackage[scaled]{helvet}
\renewcommand{\familydefault}{\sfdefault}
\usepackage[T1]{fontenc}
\usepackage[a4paper, total={6.5in, 9in}]{geometry} % Aumentato leggermente lo spazio
\usepackage{minted}
\usepackage{mathpazo}
\usepackage{fancyhdr}
\usepackage{amsthm}
\usepackage{tikz}
\usetikzlibrary{automata,positioning,arrows.meta,decorations.pathreplacing,calligraphy}

% Definizione degli ambienti
\theoremstyle{definition} % Use definition style for normal (non-italic) text
\newtheorem{theorem}{Teorema}
[section]
\newtheorem{definition}{Definizione}[section]
\newtheorem{example}{Esempio}[section]
\newtheorem{lemma}{Lemma}[section]
\newtheorem{proposition}{Proposizione}[section]

\hypersetup{
    pdftitle={Lezione di Informatica Teorica: Macchine di Turing Non Deterministiche},
    pdfauthor={Appunti da Trascrizione AI}
}

\pagestyle{fancy}
\fancyhf{}
\fancyhead[L]{\textit{Lezione di Informatica Teorica}}
\fancyfoot[C]{\thepage}


\title{Lezione di Informatica Teorica: Macchine di Turing Non Deterministiche}
\author{Appunti da Trascrizione Automatica}
\date{\today}

\begin{document}


% Stile per i diagrammi TikZ
\tikzset{
    ->, % Stile delle frecce
    >=stealth, % Punta delle frecce
    node distance=3cm, % Distanza tra i nodi
    every state/.style={thick, fill=gray!10}, % Stile per gli stati
    initial text={}, % Nessun testo 'start'
    every edge/.style={draw, text=black, sloped},
    transition_label/.style={text width=3cm, align=center}
}

\maketitle
\tableofcontents
\newpage

\section{Introduzione e Richiami sulle Macchine di Turing}

Nella lezione precedente abbiamo introdotto il modello non deterministico delle Macchine di Turing (NTM). Oggi ci concentreremo su esercizi pratici per comprendere meglio la progettazione e il funzionamento delle NTM, apprezzandone la semplificazione rispetto alle Macchine di Turing Deterministiche (DTM).

\subsection{Linguaggi Ricorsivi e Ricorsivamente Enumerabili}

Rivediamo brevemente le classi di linguaggi:
\begin{itemize}
    \item \textbf{Linguaggi Ricorsivamente Enumerabili (RE)}: Sono quei linguaggi per i quali esistono Macchine di Turing (deterministica o non deterministica, dato che hanno lo stesso potere computazionale) che li \emph{accettano}. Se la stringa di input appartiene al linguaggio, la macchina si ferma e accetta. Se la stringa non appartiene, la macchina può fermarsi e rifiutare, oppure non fermarsi affatto (loop).
    \item \textbf{Linguaggi Ricorsivi (R)}: Sono quei linguaggi per i quali esistono Macchine di Turing (deterministica o non deterministica) che li \emph{decidono}. Questo significa che per ogni stringa di input, la macchina si ferma sempre, fornendo una risposta corretta (accetta o rifiuta).
\end{itemize}

\begin{definition}{Condizione di Accettazione NTM}{condizione-di-accett}
Una Macchina di Turing non deterministica $M$ accetta una stringa di input $w$ se e solo se esiste almeno un ramo di computazione nel suo albero di computazione che porta ad uno stato di accettazione.
\end{definition}

È fondamentale comprendere che:
\begin{itemize}
    \item La macchina di Turing non può dare risposte sbagliate.
    \item Se la stringa è un'istanza "sì" del linguaggio, la macchina deve arrestarsi e accettare.
    \item Se la stringa è un'istanza "no", la macchina può fermarsi e rifiutare, oppure non fermarsi affatto (loop).
    \item Ogni volta che la macchina si arresta, fornisce la risposta corretta.
\end{itemize}

È importante sottolineare che le NTM sono un \textbf{modello astratto di calcolo} e non esistono fisicamente. La possibilità di simulare una NTM con una DTM implica un costo computazionale molto elevato (catastrofe di tempo, un problema aperto noto come P vs NP), ma il potere di riconoscimento dei linguaggi è il medesimo.

\section{Tecnica di Progettazione: "Guess and Check"}

Il principio fondamentale per la progettazione di NTM è la tecnica "Guess and Check" (Indovina e Controlla).

\begin{definition}{Tecnica Guess and Check}{tecnica-guess-and-ch}
Una NTM può essere concettualizzata come una macchina che "indovina" una parte della soluzione (fase di \emph{guess}) e poi "controlla" (verifica, fase di \emph{check}) che l'indovinello sia corretto.
\end{definition}

Questa metafora è estremamente utile per la progettazione, poiché una NTM non ha un vero potere di "intuizione" o "preveggenza". La sua capacità deriva dalla definizione formale di accettazione: se esiste un percorso di computazione che porta all'accettazione, allora la stringa è accettata. La fase di "check" è fondamentale per \textbf{filtrare} i percorsi di computazione che corrisponderebbero a "indovinelli" sbagliati, assicurando che solo i guess corretti possano portare a un'accettazione.

\paragraph{L'Albero di Computazione:}
Consideriamo l'albero di computazione di una NTM:
\begin{itemize}
    \item Ogni punto in cui la macchina compie una scelta non deterministica genera un "ramo" nell'albero.
    \item Un "guess" corrisponde a un punto di ramificazione dove la macchina "prova" diverse opzioni.
    \item Il "check" è una sequenza di transizioni che verifica la validità del guess. Se il guess è sbagliato, il ramo di computazione corrispondente deve terminare in uno stato non accettante o entrare in un loop.
    \item Solo i rami che seguono un guess corretto (e superano il check) devono portare a uno stato di accettazione.
\end{itemize}
Questo assicura che, per la definizione di accettazione NTM, la stringa venga accettata solo se esiste un guess corretto che supera il check. La metafora del "guess and check" ci permette di ragionare come se la macchina "trovasse" la soluzione ottimale, sapendo che gli altri rami verranno scartati dal processo di check.

\section{Esempi Pratici di NTM}

\subsection{Esempio 1: Riconoscimento di Stringhe Duplicate}

Il linguaggio da riconoscere è:
\begin{equation}
L = \{ww \mid w \in \{0,1\}^+\}
\end{equation}

Questo linguaggio richiede che la stringa sia composta da due copie identiche di una sottostringa $w$. Con una NTM, possiamo "indovinare" dove finisce la prima $w$ e poi verificare.

\subsubsection{Architettura e Diagramma dell'NTM}
La macchina userà due nastri: Nastro 1 (input), Nastro 2 (copia di $w$).
La logica è:
\begin{enumerate}
    \item \textbf{Copia (q0, q1)}: Copia una porzione iniziale dell'input dal Nastro 1 al Nastro 2.
    \item \textbf{Guess (da q1 a q2)}: Ad ogni passo, la macchina può non deterministicamente decidere che la porzione copiata è $w$. Questa è la transizione cruciale.
    \item \textbf{Check (da q2 a q4)}:
    \begin{itemize}
        \item Riavvolge il Nastro 2.
        \item Confronta carattere per carattere la restante parte dell'input (su Nastro 1) con la $w$ copiata (su Nastro 2).
        \item Se il confronto ha successo (entrambi i nastri finiscono contemporaneamente), accetta. Altrimenti, questo ramo di calcolo fallisce.
    \end{itemize}
\end{enumerate}

\begin{figure}[h!]
\centering
\begin{tikzpicture}
    % Definizione degli stati
    \node[state, initial] (q0) {$q_0$};
    \node[state, right=of q0] (q1) {$q_1$};
    \node[state, right=of q1, yshift=1cm] (q2) {$q_2$};
    \node[state, right=of q2] (q3) {$q_3$};
    \node[state, accepting, right=of q3] (q4) {$q_4$};

    % Definizione delle transizioni
    \path[->]
        (q0) edge[above] node[transition_label] {$\alpha, \_ \to \alpha, \alpha, R, R$} (q1)
        (q1) edge[loop above] node[transition_label] {$\alpha, \_ \to \alpha, \alpha, R, R$} ()
        (q1) edge[above, bend left=15] node[transition_label, text=red] {\textbf{GUESS della metà}\\ $\alpha, \_ \to \alpha, \_, S, L$} (q2)
        (q2) edge[loop above] node[transition_label] {Riavvolgi Nastro 2 \\ $*, \alpha \to *, \alpha, S, L$} ()
        (q2) edge[above] node[transition_label] {Fine riavvolgimento\\ $*, \_ \to *, \_, S, R$} (q3)
        (q3) edge[loop above] node[transition_label] {\textbf{CHECK} \\ $\alpha, \alpha \to \alpha, \alpha, R, R$} ()
        (q3) edge[above] node[transition_label] {Successo \\ $\_, \_ \to \_, \_, S, S$} (q4);

    % Note esplicative
    \draw [decorate, decoration={brace, mirror, amplitude=10pt}, line width=1.2pt]
    (q0.south west) -- (q1.south east) node [midway, below=12pt] {\textbf{Fase di Copia}};
    \draw [decorate, decoration={brace, mirror, amplitude=10pt}, line width=1.2pt]
    (q2.south west) -- (q4.south east) node [midway, below=12pt] {\textbf{Fase di Check}};
\end{tikzpicture}
\caption{NTM per $L = \{ww \mid w \in \{0,1\}^+\}$. La notazione è $(\text{N1}_{\text{in}}, \text{N2}_{\text{in}}) \to (\text{N1}_{\text{out}}, \text{N2}_{\text{out}}, \text{Mossa N1}, \text{Mossa N2})$. $\alpha \in \{0,1\}$, $*$ indica qualsiasi simbolo.}
\end{figure}

\subsection{Esempio 2: Riconoscimento di Sottostringhe Dirette o Inverse}

Il linguaggio da riconoscere è:
\begin{equation}
L = \{A\#B \mid A,B \in \{0,1\}^+, A \subseteq B \lor A^R \subseteq B \}
\end{equation}

La macchina deve "gessare" due cose:
\begin{enumerate}
    \item \textbf{Quale ricerca fare}: se di $A$ o di $A^R$.
    \item \textbf{Dove inizia} la sottostringa ($A$ o $A^R$) all'interno di $B$.
\end{enumerate}

\subsubsection{Architettura e Diagramma dell'NTM}
La macchina usa due nastri: Nastro 1 (input), Nastro 2 (copia di $A$).
\begin{itemize}
    \item \textbf{Fase 1 (Copia di A)}: In $q_0 \to q_1$, copia $A$ sul Nastro 2.
    \item \textbf{Fase 2 (Guess del tipo di ricerca)}: In $q_1$, quando legge il simbolo $\#$, la macchina si biforca non deterministicamente.
        \begin{itemize}
            \item \textbf{Ramo A (verso $q_2$)}: Si prepara a cercare $A$. Riavvolge il Nastro 2.
            \item \textbf{Ramo $A^R$ (verso $q_7$)}: Si prepara a cercare $A^R$. Lascia la testina del Nastro 2 alla fine di $A$.
        \end{itemize}
    \item \textbf{Fase 3 (Ricerca e Check)}:
        \begin{itemize}
            \item Entrambi i rami eseguono un "guess" della posizione di inizio in $B$ (saltando caratteri in $q_3$ e $q_7$) e poi eseguono il "check" (in $q_4$ e $q_8$).
            \item Se il check ha successo, entrambi i rami convergono verso uno stato finale comune ($q_5 \to q_6$) per verificare che non ci sia altro input anomalo.
        \end{itemize}
\end{itemize}

\begin{figure}[h!]
\centering
\scalebox{0.7}{
\begin{tikzpicture}[node distance=2cm and 2.2cm]
    % Stati iniziali
    \node[state, initial] (q0) {$q_0$};
    \node[state, right=of q0] (q1) {$q_1$};

    % Biforcazione
    \node[right=of q1] (fork) {};

    % Ramo SUPERIORE (Check A)
    \node[state, above right=of fork] (q2) {$q_2$};
    \node[state, right=of q2] (q3) {$q_3$};
    \node[state, right=of q3] (q4) {$q_4$};

    % Ramo INFERIORE (Check A^R)
    \node[state, below right=of fork] (q7) {$q_7$};
    \node[state, right=of q7] (q8) {$q_8$};

    % Stati finali comuni
    \node[state, right=of q4] (q5) {$q_5$};
    \node[state, accepting, right=of q5] (q6) {$q_6$};
    
    % Transizioni
    \path[->]
        (q0) edge[above] node[transition_label, text width=2.5cm] {Copia A su N2 \\ $\alpha, \_ \to \alpha, \alpha, R, R$} (q1)
        (q1) edge[loop above] node[transition_label, text width=2.5cm] {$\alpha, \_ \to \alpha, \alpha, R, R$} ()
        (q1) edge[above] node[transition_label, text=red, text width=2.5cm] {\textbf{GUESS A / A$^R$} \\ $\#, \_ \to \#, \_, R, L$} (q2)
        (q1) edge[below] node[transition_label, text=red, text width=2.5cm] {$\#, \_ \to \#, \_, R, S$} (q7)
        
        % Ramo A
        (q2) edge[below] node[transition_label, text width=2.5cm] {Riavvolgi N2 \\ $\to q_3$ dopo blank} (q3)
        (q3) edge[loop above] node[transition_label, text width=2.5cm] {Guess Inizio A \\ $\alpha, * \to \alpha, *, R, S$} ()
        (q3) edge[above] node[transition_label, text width=2.5cm] {Inizio Check A} (q4)
        (q4) edge[loop above] node[transition_label, text width=2.5cm] {Check A \\ $\alpha, \alpha \to \alpha, \alpha, R, R$} ()
        (q4) edge[above] node[transition_label, text width=2.5cm] {Fine Check A\\ $*, \_ \to *, \_, S, S$} (q5)

        % Ramo A^R
        (q7) edge[loop below] node[transition_label, text width=2.5cm] {Guess Inizio A$^R$ \\ $\alpha, * \to \alpha, *, R, S$} ()
        (q7) edge[below] node[transition_label, text width=2.5cm] {Inizio Check A$^R$} (q8)
        (q8) edge[loop below] node[transition_label, text width=2.5cm] {Check A$^R$ \\ $\alpha, \alpha \to \alpha, \alpha, R, L$} ()
        (q8) edge[below] node[transition_label, text width=2.5cm] {Fine Check A$^R$ \\ $*, \_ \to *, \_, S, S$} (q5)
        
        % Uscita
        (q5) edge[above] node[transition_label, text width=2.5cm] {Vai a fine input} (q6);

    % Note
    \node[above=0.2cm of q2, text width=4cm, align=center, text=blue] {\textbf{Ramo per $A \subseteq B$}};
    \node[below=0.2cm of q7, text width=4cm, align=center, text=blue] {\textbf{Ramo per $A^R \subseteq B$}};
\end{tikzpicture}
}
\caption{NTM per $L = \{A\#B \mid A \subseteq B \lor A^R \subseteq B \}$. La biforcazione dopo $q_1$ è la chiave.}
\end{figure}

\subsubsection{Errore Comune di Progettazione: Check Ambigui}
Un errore frequente nella progettazione delle NTM si verifica quando il "check" diventa ambiguo. Se la scelta non deterministica tra la verifica di $A$ e $A^R$ avviene \emph{durante} il confronto, la macchina può accettare stringhe che non dovrebbe.

\textbf{Esempio di Errore:} Si usano tre nastri: N1 (Input), N2 (copia di $A$), N3 (copia di $A$ letta al contrario). Dopo aver "gessato" l'inizio della sottostringa in $B$, si entra in un unico stato di check, $q_{\text{check}}$, che ha due loop non deterministici:
\begin{itemize}
    \item Un loop confronta N1 con N2 (per $A$).
    \item Un altro loop confronta N1 con N3 (per $A^R$).
\end{itemize}
\textbf{Perché è un errore?} La macchina potrebbe confrontare i primi caratteri di $B$ usando il primo loop (match con $A$) e i successivi caratteri usando il secondo loop (match con $A^R$). In questo modo, potrebbe accettare una stringa che non è né $A$ né $A^R$, ma un "misto" delle due. Il check non è robusto. La scelta va fatta \textbf{prima} del check, separando nettamente i rami di computazione, come nel diagramma corretto.

\begin{figure}[h!]
\centering
\begin{tikzpicture}
    \node[state] (qc) {$q_{\text{check}}$};
    \path[->] (qc) edge[loop above] node[transition_label, text=red] {Confronta N1 con N2 \\ (Check A)} ()
                   edge[loop below] node[transition_label, text=red] {Confronta N1 con N3 \\ (Check A$^R$)} ();
\end{tikzpicture}
\caption{\textbf{Design Errato}: Un check ambiguo che permette di "mischiare" i controlli di $A$ e $A^R$.}
\end{figure}

\newpage
\subsection{Esempio 3: Riconoscimento di un Linguaggio con Palindromi di Lunghezza Crescente}

Questo è un problema più complesso, che richiede un'attenta gestione iterativa di "guess" e "check".
\begin{align}
L = \{X^n \# W_1 \# \dots \# W_n \mid &\; n>0, \; W_i \in \{a,b,c,d\}^+, \\
&\; \forall i \in [1,n], \exists S_i \subseteq W_i, \text{ t.c. } |S_i|=i \text{ e } S_i = S_i^R \} \nonumber
\end{align}

\subsubsection{Discussione della Strategia}
La macchina dovrà usare più nastri e iterare su ogni $W_i$, verificando la condizione richiesta.

\textbf{Nastri suggeriti:}
\begin{itemize}
    \item \textbf{Nastro 1}: Nastro di Input.
    \item \textbf{Nastro 2}: Contatore per $i$. Inizialmente vuoto, si aggiunge una 'X' ad ogni iterazione per tenere traccia della lunghezza della palindroma da cercare.
    \item \textbf{Nastro 3}: Nastro ausiliario per la copia e la verifica della palindroma $S_i$.
\end{itemize}

\textbf{Fasi della computazione (strategia "Guess and Check"):}

\begin{enumerate}
    \item \textbf{Fase 1: Validazione iniziale e conteggio di $n$}
        \begin{itemize}
            \item La macchina conta il numero di 'X' iniziali ($n$) e il numero di stringhe $W_i$ (contando i $\#$). Se i conteggi non corrispondono, o se $n=0$, rifiuta. Questo può essere fatto in una fase deterministica preliminare.
        \end{itemize}

    \item \textbf{Fase 2: Ciclo iterativo su ogni $W_i$ (da $i=1$ a $n$)}
        \begin{itemize}
            \item La macchina entra in un grande ciclo per processare $W_1, W_2, \dots, W_n$.
            \item Ad ogni iterazione $i$, aggiunge una 'X' al Nastro 2. Il numero di 'X' su Nastro 2 rappresenta la lunghezza $i$ della palindroma da cercare in $W_i$.
            \item \textbf{Guess \& Check per $S_i$ in $W_i$}: Per ogni $W_i$ e il contatore $i$ corrente (su Nastro 2):
                \begin{enumerate}
                    \item \textbf{Guess della posizione di $S_i$}: La macchina scansiona $W_i$ sul Nastro 1. Non deterministicamente, salta un numero arbitrario di caratteri, "indovinando" il punto di inizio di $S_i$.
                    
                    \item \textbf{Copia e Check}: Una volta "gessato" l'inizio, la macchina copia sul Nastro 3 un numero di caratteri da $W_i$ pari alla lunghezza $i$ (letta dal Nastro 2).
                    
                    \item \textbf{Check di palindromia e lunghezza}: La macchina esegue due controlli sul Nastro 3:
                    \begin{itemize}
                        \item Verifica che la stringa copiata sia effettivamente lunga $i$ (confrontando con le 'X' su Nastro 2).
                        \item Verifica che la stringa copiata sia palindroma (leggendo il Nastro 3 dai due estremi verso il centro).
                    \end{itemize}
                    Se uno di questi check fallisce, questo ramo di computazione si blocca.
                \end{enumerate}
            \item Se per un dato $W_i$ un ramo di computazione supera con successo il check, la macchina si sposta al successivo $\#$ per iniziare a processare $W_{i+1}$.
        \end{itemize}

    \item \textbf{Fase Finale: Accettazione}
        \begin{itemize}
            \item Se il ciclo termina con successo per tutti gli $n$ blocchi $W_i$, la macchina accetta.
        \end{itemize}
\end{enumerate}

\paragraph{Strategia Alternativa (più potente):} Come discusso in aula, una NTM può anche "indovinare" la palindroma $S_i$ e poi verificare che esista in $W_i$.
\begin{enumerate}
    \item \textbf{Guess $S_i$}: Non deterministicamente, la macchina "scrive" una stringa palindroma di lunghezza $i$ sul Nastro 3.
    \item \textbf{Check $S_i \subseteq W_i$}: La macchina verifica se la stringa appena "inventata" sul Nastro 3 è una sottostringa di $W_i$ (su Nastro 1).
\end{enumerate}
Entrambe le strategie sono valide e dimostrano la potenza del modello "Guess and Check". La prima è forse più intuitiva ("trova e controlla"), mentre la seconda è più astratta ("inventa e controlla").

\end{document}