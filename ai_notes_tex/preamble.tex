% =========================================================================
% ===                COMMON PREAMBLE FOR ALL LECTURES                   ===
% =========================================================================

\documentclass[a4paper, 11pt, openany]{book} % Use 'book' class for chapters, openany allows chapters to start on any page

% --- Core Packages ---
\usepackage[italian]{babel}
\usepackage[utf8]{inputenc}
\usepackage{amsmath, amssymb, amsfonts}
\usepackage{graphicx}
\usepackage{dot2texi}
\usepackage[a4paper, total={6.3in, 9.2in}]{geometry}

% --- Fonts and Appearance ---
% \usepackage{mathpazo} % For a nicer font
\usepackage[scaled]{helvet}
\renewcommand{\familydefault}{\sfdefault}
\usepackage[T1]{fontenc}
\usepackage{fancyhdr} % For custom headers and footers

% --- License Information ---
\usepackage[
  type={CC},
  modifier={by-sa},
  version={4.0},
]{doclicense}

% --- Special Content Packages ---
\usepackage{minted} % For pseudocode. Requires --shell-escape compiler flag
\usepackage{amsthm}   % For theorem environments
\usepackage{tikz}     % For diagrams
\usepackage[shortlabels]{enumitem} % For customized lists with backward compatibility
\usepackage{subcaption} % For subfigures
\usepackage{float}    % For figure placement [H]

% --- Enumitem Configuration ---
% Set up enumitem to handle standard enumerate environments properly
\setlist[enumerate]{label=\arabic*.}

% --- Hyperlinking ---
\usepackage{hyperref}
\usepackage[scaled]{helvet}
\renewcommand{\familydefault}{\sfdefault}
\usepackage[T1]{fontenc}

% --- TikZ Libraries (superset of all used libraries) ---
\usetikzlibrary{
    automata,
    positioning,
    arrows,
    arrows.meta,
    calc,
    fit,
    patterns,
    graphs,
    graphdrawing,
    matrix,
    quotes,
    shapes.geometric,
    snakes,
    shapes,
    decorations.pathreplacing,
    calligraphy,
    chains,
}

% Add this to your preamble
\usepackage{tcolorbox}
\tcbuselibrary{theorems,breakable,skins}
\usepackage{tikz} % Make sure TikZ is loaded

% Define colors for different theorem types
\definecolor{theoremcolor}{RGB}{0,100,150}
\definecolor{definitioncolor}{RGB}{0,150,100}
\definecolor{examplecolor}{RGB}{150,100,0}
\definecolor{problemcolor}{RGB}{150,0,100}
\definecolor{lemmacolor}{RGB}{100,0,150}
\definecolor{remarkcolor}{RGB}{100,100,100}

% --- Theorem-like Environments with boxes ---
\theoremstyle{definition}

% Theorem
\newtcbtheorem[number within=section]{theorem}{Teorema}%
{colback=theoremcolor!5,colframe=theoremcolor!35!black,fonttitle=\bfseries,
 breakable,enhanced}{th}

% Definition
\newtcbtheorem[use counter from=theorem]{definition}{Definizione}%
{colback=definitioncolor!5,colframe=definitioncolor!35!black,fonttitle=\bfseries,
 breakable,enhanced}{def}

% Example
\newtcbtheorem[use counter from=theorem]{example}{Esempio}%
{colback=examplecolor!5,colframe=examplecolor!35!black,fonttitle=\bfseries,
 breakable,enhanced}{ex}

% Problem
\newtcbtheorem[number within=section]{problem}{Problema}%
{colback=problemcolor!5,colframe=problemcolor!35!black,fonttitle=\bfseries,
 breakable,enhanced}{prob}

% Lemma
\newtcbtheorem[use counter from=theorem]{lemma}{Lemma}%
{colback=lemmacolor!5,colframe=lemmacolor!35!black,fonttitle=\bfseries,
 breakable,enhanced}{lem}

% Proposition
\newtcbtheorem[use counter from=theorem]{proposition}{Proposizione}%
{colback=theoremcolor!5,colframe=theoremcolor!35!black,fonttitle=\bfseries,
 breakable,enhanced}{prop}

% Corollary
\newtcbtheorem[use counter from=theorem]{corollary}{Corollario}%
{colback=theoremcolor!5,colframe=theoremcolor!35!black,fonttitle=\bfseries,
 breakable,enhanced}{cor}

% Remark
\newtcbtheorem[use counter from=theorem]{remark}{Osservazione}%
{colback=remarkcolor!5,colframe=remarkcolor!35!black,fonttitle=\bfseries,
 breakable,enhanced}{rem}

% Proof Sketch
\newtcbtheorem[use counter from=theorem]{proof_sketch}{Sketch di Dimostrazione}%
{colback=gray!5,colframe=gray!35!black,fonttitle=\bfseries,
 breakable,enhanced}{psketch}

% Usage examples:
% \begin{theorem}{Title}{label}
%   Content of theorem
% \end{theorem}
%
% \begin{definition}{Title}{label}
%   Content of definition
% \end{definition}
%
% Example with TikZ inside a theorem:
% \begin{theorem}{Pythagorean Theorem}{pythagoras}
%   In a right triangle, $a^2 + b^2 = c^2$.
%   \begin{center}
%   \begin{tikzpicture}[scale=0.8]
%     \draw (0,0) -- (3,0) -- (3,4) -- cycle;
%     \draw (0,0) -- (0.5,0) -- (0.5,0.5) -- (0,0.5) -- cycle;
%     \node at (1.5,-0.3) {$a$};
%     \node at (3.3,2) {$b$};
%     \node at (1.3,2.2) {$c$};
%   \end{tikzpicture}
%   \end{center}
% \end{theorem}

% --- Custom Commands (from various lectures) ---
% For Turing Machine symbols (from lecture 9/10)
\newcommand{\B}{\text{B}} % Blank symbol
\newcommand{\SigmaI}{\Sigma_I} % Input Alphabet
\newcommand{\GammaT}{\Gamma} % Tape Alphabet
\newcommand{\alphaSym}{\alpha} % Generic symbol from input alphabet (non-#)

% Halt epsilon complement
\newcommand{\haltcomplement}{\ensuremath{\overline{\text{HALT}_\epsilon}}}


% Comandi personalizzati per simboli TM
% \newcommand{\B}{\texttt{B}} % Blank symbol
\newcommand{\Sh}{\texttt{\#}} % Sharp symbol
\newcommand{\X}{\texttt{X}} % Counter symbol
% \newcommand{\alphaSym}{\alpha} % Generic symbol from input alphabet (non-#)
\newcommand{\betaSym}{\beta}  % Another generic symbol
\newcommand{\any}{\_} % Wildcard symbol

% \newcommand{\blankS}{\ensuremath{\square}}


\newcommand{\blankS}{\ensuremath{\raisebox{-0.15ex}{\scalebox{1.3}[0.7]{$\sqcup$}}\mkern2mu}}

% guida all'esame
% --- COMANDI PERSONALIZZATI ---
\newcommand{\lang}[1]{\mathcal{L}(#1)}
\newcommand{\set}[1]{\{#1\}}
\newcommand{\tuple}[1]{\langle#1\rangle}
\newcommand{\bigO}[1]{\mathcal{O}(#1)}
\newcommand{\reducep}{\le_p}
\newcommand{\reducelog}{\le_L}


% For TikZ diagrams (from lecture 27)
\tikzset{
    green_arrow/.style={-Latex, thick, draw=green!70!black},
    red_arrow/.style={-Latex, thick, draw=red!70!black},
    black_arrow/.style={-Latex, thick, draw=black},
    double_black_arrow/.style={-Latex, thick, draw=black, double},
    node_style/.style={circle, draw, fill=white, inner sep=1pt},
    small_node_style/.style={circle, draw, fill=lightgray, inner sep=0.5pt},
    diamond_node_style/.style={diamond, draw, fill=white, inner sep=1pt},
    clause_node_style/.style={rectangle, draw, fill=white, inner sep=2pt},
    transition_label/.style={text width=3cm, align=center},
}

% --- PDF Metadata ---
\hypersetup{
    pdftitle={Appunti Completi di Informatica Teorica},
    pdfauthor={Appunti da Trascrizione AI}
}

% --- Header and Footer Configuration ---
\pagestyle{fancy}
\fancyhf{}
\fancyhead[LE,RO]{\textit{Informatica Teorica}} % Right on Odd, Left on Even
\fancyhead[RE,LO]{\nouppercase{\leftmark}} % Chapter name
\fancyfoot[C]{\thepage}

% =========================================================================
% ===                  END OF COMMON PREAMBLE                           ===
% =========================================================================
