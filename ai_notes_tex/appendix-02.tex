\documentclass[a4paper]{article}
\usepackage{amsmath, amssymb, amsfonts}
\usepackage[italian]{babel}
\usepackage[utf8]{inputenc} % Corrected this line
\usepackage{graphicx}
\usepackage{hyperref}
\usepackage[scaled]{helvet}
\renewcommand{\familydefault}{\sfdefault}
\usepackage[T1]{fontenc}
\usepackage[a4paper, total={6in, 8in}]{geometry}
\usepackage{minted}
\usepackage{mathpazo}
\usepackage{fancyhdr}
\usepackage{amsthm}
\usepackage{tikz}
\usetikzlibrary{automata,positioning}

% Definizione degli ambienti
\theoremstyle{definition} % Use definition style for normal (non-italic) text
\newtheorem{theorem}{Teorema}
\newtheorem{definition}{Definizione}
\newtheorem{example}{Esempio}
\newtheorem{lemma}{Lemma}
\newtheorem{problem}{Problema}[subsection]
\newtheorem{proposition}{Proposizione}
\newcommand{\blankS}{\ensuremath{\raisebox{-0.15ex}{\scalebox{1.3}[0.7]{$\sqcup$}}\mkern2mu}}

\hypersetup{
    pdftitle={Lezione di Informatica Teorica - How to Esame},
    pdfauthor={Appunti da Trascrizione AI}
}

\pagestyle{fancy}
\fancyhf{}
\fancyhead[L]{\textit{How to Esame}}
\fancyfoot[C]{\thepage}

\title{How to Esame}
\author{Appunti da Trascrizione Automatica}
\date{\today}


% --- COMANDI PERSONALIZZATI ---
\newcommand{\lang}[1]{\mathcal{L}(#1)}
\newcommand{\set}[1]{\{#1\}}
\newcommand{\tuple}[1]{\langle#1\rangle}
\newcommand{\bigO}[1]{\mathcal{O}(#1)}
\newcommand{\reducep}{\le_p}
\newcommand{\reducelog}{\le_L}

\begin{document}
\maketitle
\tableofcontents
\newpage

\section{Teoria della Calcolabilità}

\subsection{Introduzione e Modelli Formali}
Il corso si concentra sulla distinzione tra la complessità degli algoritmi e la complessità intrinseca dei problemi. L'obiettivo è capire quali problemi sono risolvibili da un computer (calcolabili/decidibili) e quali no, e tra quelli risolvibili, quali sono "facili" e quali "difficili".

\subsubsection{Problemi di Ricerca e di Decisione}
\begin{itemize}
    \item \textbf{Problema di Ricerca}: Un problema la cui soluzione può essere un output complesso (es. calcolare il percorso minimo in un grafo, trovare il prodotto di due matrici). Formalmente, una relazione tra stringhe di input e stringhe di output.
    \item \textbf{Problema di Decisione}: Un problema la cui risposta è solo \textbf{sì} o \textbf{no} (un booleano). È un caso speciale dei problemi di ricerca.
\end{itemize}
Ci si concentra sui problemi di decisione perché sono più semplici da analizzare formalmente.

\subsubsection{Linguaggi Formali}
Per semplificare ulteriormente, i problemi di decisione vengono ricondotti al problema di decidere l'appartenenza di una stringa a un linguaggio.
\begin{definition}{Linguaggio}{linguaggio}
Dato un alfabeto $\Sigma$ (un insieme finito di simboli), un \textbf{linguaggio} $L$ è un qualsiasi sottoinsieme di $\Sigma^*$, dove $\Sigma^*$ è l'insieme di tutte le stringhe di lunghezza finita costruibili con i simboli di $\Sigma$.
\end{definition}
Decidere un linguaggio $L$ significa, data una stringa $w$, determinare se $w \in L$.

\subsubsection{Macchine di Turing (TM)}
La Macchina di Turing è il modello formale di calcolo universale. È un automa a stati finiti con capacità aggiuntive.
\begin{definition}{Macchina di Turing}{macchina-di-turing}
Una TM è una tupla $M = (Q, \Sigma, \Gamma, \delta, q_0, B, F)$ dove:
\begin{itemize}
    \item $Q$ è un insieme finito di stati.
    \item $\Sigma$ è l'alfabeto di input.
    \item $\Gamma$ è l'alfabeto del nastro ($\Sigma \subset \Gamma$).
    \item $\delta: Q \times \Gamma \to Q \times \Gamma \times \{L, R\}$ è la funzione di transizione.
    \item $q_0 \in Q$ è lo stato iniziale.
    \item $B \in \Gamma \setminus \Sigma$ è il simbolo speciale \emph{blank}.
    \item $F \subseteq Q$ è l'insieme degli stati finali (o di accettazione).
\end{itemize}
\end{definition}
Caratteristiche chiave:
\begin{enumerate}
    \item Il nastro è infinito in entrambe le direzioni.
    \item La testina può leggere, scrivere e muoversi a sinistra (L) o a destra (R).
\end{enumerate}
\textbf{Varianti}: Le TM multi-nastro e le TM non-deterministiche (NDTM) hanno lo stesso potere computazionale di una TM standard, sebbene una NDTM possa essere esponenzialmente più veloce. La simulazione di una NDTM su una DTM richiede tempo esponenziale.

\subsection{Decidibilità e Indecidibilità}
\subsubsection{Classi R e RE}
\begin{definition}{Linguaggi Ricorsivi / Decidibili - Classe R}{linguaggi-ricorsivi-}
Un linguaggio $L$ è \textbf{ricorsivo} (o \textbf{decidibile}) se esiste una TM $M$ che lo \emph{decide}. Questo significa che $M$ termina su ogni input $w$:
\begin{itemize}
    \item Se $w \in L$, $M$ accetta.
    \item Se $w \notin L$, $M$ rifiuta (terminando in uno stato non finale).
\end{itemize}
La classe di tutti i linguaggi ricorsivi è denotata con $\mathbf{R}$.
\end{definition}

\begin{definition}{Linguaggi Ricorsivamente Enumerabili - Classe RE}{linguaggi-ricorsivam}
Un linguaggio $L$ è \textbf{ricorsivamente enumerabile} (o \textbf{riconoscibile}) se esiste una TM $M$ che lo \emph{accetta}.
\begin{itemize}
    \item Se $w \in L$, $M$ termina e accetta.
    \item Se $w \notin L$, $M$ può rifiutare o non terminare mai (loop).
\end{itemize}
La classe di tutti i linguaggi R.E. è denotata con $\mathbf{RE}$.
\end{definition}
È chiaro che $\mathbf{R} \subseteq \mathbf{RE}$.

\subsubsection{Problemi Indecidibili Famosi}
\begin{problem}{Linguaggio Universale $L_u$}{linguaggio-universal}
$L_u = \{ \tuple{M, w} \mid M \text{ è una TM e } M \text{ accetta } w \}$.
\begin{itemize}
    \item $L_u \in \mathbf{RE}$: Si può costruire una Macchina di Turing Universale (UTM) che simula $M$ su $w$. Se $M$ accetta, la UTM accetta.
    \item $L_u \notin \mathbf{R}$: Non è decidibile. La dimostrazione usa la diagonalizzazione.
\end{itemize}
\end{problem}
\begin{problem}{Halting Problem $H_{TM}$}{halting-problem}
$H_{TM} = \{ \tuple{M, w} \mid M \text{ è una TM e } M \text{ termina su } w \}$.
\begin{itemize}
    \item $H_{TM} \in \mathbf{RE}$.
    \item $H_{TM} \notin \mathbf{R}$.
\end{itemize}
\end{problem}
\textbf{Co-RE}: Un linguaggio $L$ è in Co-RE se il suo complemento $\bar{L}$ è in RE. Abbiamo dimostrato che $L \in \mathbf{R} \iff L \in \mathbf{RE} \cap \mathbf{Co-RE}$.

\subsubsection{Teorema di Rice}
È un teorema potente che generalizza l'indecidibilità a intere classi di problemi.
\begin{definition}{Proprietà Semantica}{proprieta-semantica}
Una proprietà $P$ dei linguaggi R.E. è una \textbf{proprietà semantica} se per ogni coppia di TM $M_1, M_2$ tali che $\lang{M_1} = \lang{M_2}$, vale che $M_1 \in P \iff M_2 \in P$. La proprietà dipende solo dal linguaggio, non dalla struttura della macchina.
\end{definition}
\begin{theorem}{Teorema di Rice}{teorema-di-rice}
Ogni proprietà non banale dei linguaggi ricorsivamente enumerabili è indecidibile.
\end{theorem}
Una proprietà è "non banale" se non è né vuota (nessun linguaggio la possiede) né universale (tutti i linguaggi R.E. la possiedono).
\begin{example}{}{{ example-appendix-02-1 }}
Il problema "il linguaggio di una TM è vuoto?" ($L_e$) è indecidibile per il Teorema di Rice. È una proprietà semantica (riguarda il linguaggio) e non è banale (ci sono TM con linguaggio vuoto e TM con linguaggio non vuoto).
\end{example}

\section{Teoria della Complessità}

\subsection{Classi di Complessità Temporale}
All'interno della classe $\mathbf{R}$, analizziamo i problemi in base alle risorse (tempo, spazio) necessarie per risolverli.
\begin{definition}{Complessità Temporale}{complessita-temporal}
Sia $M$ una TM che decide un linguaggio. Il \textbf{running time} di $M$ è una funzione $T(n)$ tale che per ogni input $w$ di lunghezza $n$, il numero di passi di $M$ su $w$ è al massimo $T(n)$.
\end{definition}

\subsubsection{La Classe P}
\begin{definition}{Classe P}{classe-p}
$\mathbf{P}$ (Polynomial Time) è la classe di tutti i linguaggi $L$ per cui esiste una TM \textbf{deterministica} $M$ che decide $L$ in tempo polinomiale.
$$ \mathbf{P} = \bigcup_{k \ge 1} \text{DTIME}(\bigO{n^k}) $$
I problemi in $\mathbf{P}$ sono considerati \textbf{trattabili} o "facili".
\end{definition}

\subsubsection{La Classe NP}
\begin{definition}{Classe NP - via NDTM}{classe-np-via-ndtm}
$\mathbf{NP}$ (Non-deterministic Polynomial Time) è la classe di tutti i linguaggi $L$ per cui esiste una TM \textbf{non-deterministica} $M$ che decide $L$ in tempo polinomiale.
$$ \mathbf{NP} = \bigcup_{k \ge 1} \text{NTIME}(\bigO{n^k}) $$
\end{definition}
\begin{definition}{Classe NP - via Verificatore}{classe-np-via-verifi}
Un linguaggio $L$ è in $\mathbf{NP}$ se esiste un algoritmo \textbf{verificatore} deterministico $V$, che lavora in tempo polinomiale, e una costante $k$ tali che:
$$ w \in L \iff \exists c, |c| = \bigO{|w|^k} \text{ tale che } V(w, c) = \text{accetta} $$
La stringa $c$ è chiamata \textbf{certificato} o \textbf{testimone}. I problemi in $\mathbf{NP}$ sono quelli le cui soluzioni (istanze sì) sono facilmente \emph{verificabili} se ci viene dato un aiuto (il certificato).
\end{definition}
Chiaramente, $\mathbf{P} \subseteq \mathbf{NP}$. La domanda se $\mathbf{P} = \mathbf{NP}$ è il più grande problema aperto dell'informatica.

\subsection{NP-Completezza}
Per studiare la relazione P vs NP, si è introdotta la nozione di problemi "più difficili" in NP.
\begin{definition}{Riduzione Polinomiale}{riduzione-polinomial}
Un linguaggio $A$ è riducibile in tempo polinomiale a un linguaggio $B$, denotato $A \reducep B$, se esiste una funzione $f$ calcolabile in tempo polinomiale (da un trasduttore) tale che per ogni stringa $w$:
$$ w \in A \iff f(w) \in B $$
\end{definition}
\begin{definition}{NP-Hard e NP-Completo}{nphard-e-npcompleto}
\begin{itemize}
    \item Un linguaggio $L$ è \textbf{NP-hard} se ogni linguaggio $L' \in \mathbf{NP}$ è riducibile in tempo polinomiale a $L$ ($L' \reducep L$).
    \item Un linguaggio $L$ è \textbf{NP-completo} se è NP-hard e $L \in \mathbf{NP}$.
\end{itemize}
\end{definition}
I problemi NP-completi sono i problemi più difficili in NP. Se anche solo uno di essi fosse risolvibile in tempo polinomiale, allora $\mathbf{P} = \mathbf{NP}$.

\subsubsection{Principali Problemi NP-Completi}
La base di tutte le dimostrazioni di NP-completezza è il seguente teorema:
\begin{theorem}{Teorema di Cook-Levin}{teorema-di-cooklevin}
Il problema della soddisfacibilità booleana (\textbf{SAT}) è NP-completo.
\end{theorem}
Da qui, si dimostra l'NP-completezza di altri problemi tramite una catena di riduzioni:
\begin{itemize}
    \item \textbf{3-SAT}: Una variante di SAT dove ogni clausola ha esattamente (o al più) 3 letterali. Si dimostra tramite riduzione da SAT.
    \item \textbf{Independent Set (IS)}: Dato un grafo $G$ e un intero $k$, esiste un insieme di $k$ vertici a due a due non adiacenti? Riduzione da 3-SAT.
    \item \textbf{Vertex Cover (VC)}: Dato $G$ e $k$, esiste un insieme di $k$ vertici che "coprono" tutti gli archi? Riduzione da IS.
    \item \textbf{Clique}: Dato $G$ e $k$, esiste un sottografo completo (cricca) di $k$ vertici? Riduzione da IS (sul grafo complemento).
    \item \textbf{Napsack (decisionale)}: Problema dello zaino. Riduzione da Exact Cover.
    \item \textbf{Traveling Salesperson Problem (TSP, decisionale)}: Problema del commesso viaggiatore. Riduzione da Hamiltonian Cycle.
\end{itemize}

\subsection{Classi di Complessità Oltre NP}
\subsubsection{Co-NP}
\begin{definition}{Classe Co-NP}{classe-conp}
Un linguaggio $L$ è in $\mathbf{Co-NP}$ se il suo complemento, $\bar{L}$, è in $\mathbf{NP}$.
\end{definition}
I problemi in Co-NP sono quelli per cui esiste un certificato conciso per le istanze \textbf{no}. Un esempio è \textbf{UNSAT} (formule non soddisfacibili). La relazione tra NP e Co-NP è un altro problema aperto.

\subsubsection{Classi Spaziali e Gerarchia Polinomiale}
\begin{itemize}
    \item \textbf{L} e \textbf{NL}: Problemi risolvibili in spazio logaritmico deterministico e non-deterministico.
    \item \textbf{PSPACE}: Problemi risolvibili in spazio polinomiale.
    \item \textbf{EXPTIME}: Problemi risolvibili in tempo esponenziale.
\end{itemize}
Relazioni note:
$$ \mathbf{L} \subseteq \mathbf{NL} \subseteq \mathbf{P} \subseteq \mathbf{NP} \subseteq \mathbf{PSPACE} \subseteq \mathbf{EXPTIME} $$
\begin{theorem}{Teorema di Savitch}{teorema-di-savitch}
Per ogni funzione $S(n) \ge \log n$:
$$ \text{NSPACE}(S(n)) \subseteq \text{DSPACE}(S(n)^2) $$
Questo implica che $\mathbf{PSPACE} = \mathbf{NPSPACE}$.
\end{theorem}
La \textbf{Gerarchia Polinomiale (PH)} è una gerarchia di classi ($\Sigma_k^p, \Pi_k^p, \Delta_k^p$) definita tramite macchine a oracolo che si trova tra NP e PSPACE.

\section{Modalità d'Esame}
\subsection{Struttura e Valutazione}
\begin{itemize}
    \item \textbf{Durata}: 3.5 ore.
    \item \textbf{Punteggio Massimo Scritto}: 32 punti. Questo permette un margine di errore.
    \item \textbf{Struttura Prova Scritta}:
    \begin{enumerate}
        \item \textbf{Progettazione di una Macchina di Turing} (8 punti): Data la descrizione di un linguaggio, definire una TM (eventualmente non-deterministica e/o multi-nastro) che lo decide. L'enfasi è su problemi che richiedono una gestione non banale della computazione.
        \item \textbf{Domande di Teoria} (circa 4 domande, 8 punti totali): Definizioni formali e concetti visti a lezione (es. "Cos'è una riduzione?", "Definisci la classe NP", "Che differenza c'è tra P e R?").
        \item \textbf{Dimostrazioni viste a lezione} (1-2 domande, circa 6-8 punti): Dimostrare un teorema o una proprietà discussa durante il corso (es. transitività delle riduzioni, il teorema di Rice in forma generale, proprietà delle classi R e RE).
        \item \textbf{Esercizio di (In)decidibilità} (5-6 punti): Dato un nuovo linguaggio (proprietà di TM), discutere la sua decidibilità. Potrebbe richiedere l'applicazione del Teorema di Rice (verificando se la proprietà è semantica e non banale) o una riduzione ad-hoc da un problema noto indecidibile (es. $L_u$, $H_{TM}$).
        \item \textbf{Esercizio di NP-Completezza} (6-7 punti): Dato un nuovo problema, dimostrare la sua NP-completezza. Questo richiede:
        \begin{itemize}
            \item Dimostrare l'appartenenza a NP (guess \& check).
            \item Dimostrare l'NP-hardness tramite una riduzione da un problema noto NP-completo (il problema sorgente sarà suggerito).
        \end{itemize}
    \end{enumerate}
    \item \textbf{Orale}:
    \begin{itemize}
        \item L'orale è \textbf{obbligatorio} solo per chi ottiene un punteggio $\ge 29$ allo scritto e desidera un voto superiore a 28.
        \item Chi ottiene $\ge 29$ può scegliere di non fare l'orale e registrare il voto di \textbf{28}.
        \item L'orale può \textbf{aumentare o diminuire} il voto dello scritto.
        \item La \textbf{lode} si può ottenere solo tramite l'orale.
        \item Le domande verteranno su collegamenti tra concetti, ragionamenti e dimostrazioni più complesse.
    \end{itemize}
    \item \textbf{Registrazione}: Chi ottiene un voto $\ge 18$ e non è soggetto all'orale obbligatorio può registrare direttamente il voto dello scritto. Il silenzio-assenso dopo la pubblicazione dei risultati implica l'accettazione del voto.
\end{itemize}


\end{document}