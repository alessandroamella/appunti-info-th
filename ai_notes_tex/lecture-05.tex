\documentclass[a4paper]{article}
\usepackage{amsmath, amssymb, amsfonts}
\usepackage[italian]{babel}
\usepackage[utf8]{inputenc}
\usepackage{graphicx}
\usepackage{hyperref}
\usepackage[a4paper, total={6in, 8in}]{geometry}
\usepackage{minted}
\usepackage{mathpazo}
\usepackage{fancyhdr}
\usepackage{amsthm}
\usepackage{tikz}
\usetikzlibrary{automata,positioning}

% Definizione degli ambienti
\theoremstyle{definition} % Use definition style for normal (non-italic) text
\newtheorem{theorem}{Teorema}
\newtheorem{definition}{Definizione}
\newtheorem{example}{Esempio}
\newtheorem{lemma}{Lemma}
\newtheorem{proposition}{Proposizione}
\newcommand{\blankS}{\ensuremath{\raisebox{-0.15ex}{\scalebox{1.3}[0.7]{$\sqcup$}}\mkern2mu}}

\hypersetup{
    pdftitle={Lezione di Informatica Teorica: TM e Decidibilità},
    pdfauthor={Appunti da Trascrizione AI}
}

\pagestyle{fancy}
\fancyhf{}
\fancyhead[L]{\textit{Lezione di Informatica Teorica}}
\fancyfoot[C]{\thepage}

\title{Lezione di Informatica Teorica: Macchine di Turing e Decidibilità dei Linguaggi}
\author{Appunti da Trascrizione Automatica}
\date{\today}

\begin{document}
\maketitle
\tableofcontents
\newpage

\section{Introduzione alle Macchine di Turing}

Ripassiamo brevemente i concetti fondamentali delle Macchine di Turing (TM) introdotti nella lezione precedente.

\begin{definition}[Macchina di Turing]
Una Macchina di Turing è un automa a stati finiti con le seguenti peculiarità:
\begin{itemize}
    \item Possiede un \textbf{nastro di input infinito} che può essere letto e scritto. A differenza degli automi a stati finiti che possono solo leggere l'input, le TM possono modificare il loro nastro.
    \item La \textbf{testina} della macchina può spostarsi sia a destra che a sinistra sul nastro. Gli automi a stati finiti, invece, si muovono solo a destra.
    \item Il \textbf{programma} (funzione di transizione) di una TM è fissato e non può cambiare durante l'esecuzione. Una volta che la macchina è "accesa" con un determinato programma, esso rimane invariato. Questo, pur sembrando una limitazione rispetto ai computer moderni, vedremo che non lo è, poiché le TM sono modelli di calcolo universali.
\end{itemize}
Queste caratteristiche rendono le Macchine di Turing un modello di calcolo universale, capace di simulare qualsiasi algoritmo computabile.
\end{definition}

\subsection{Linguaggi: Decidibilità vs. Accettazione}

È fondamentale distinguere due concetti chiave relativi ai linguaggi che una Macchina di Turing può riconoscere.

\begin{definition}[Macchina che Decide un Linguaggio]
Una Macchina di Turing \textbf{decide} un linguaggio $L$ se per ogni stringa $w$ in input:
\begin{itemize}
    \item Se $w \in L$, la macchina si ferma in uno stato accettante.
    \item Se $w \notin L$, la macchina si ferma in uno stato non accettante.
\end{itemize}
In sintesi, una macchina che decide un linguaggio è garantita fermarsi su ogni input, fornendo una risposta "sì" (accettazione) o "no" (rifiuto).
\end{definition}

\begin{definition}[Macchina che Accetta un Linguaggio]
Una Macchina di Turing \textbf{accetta} un linguaggio $L$ se:
\begin{itemize}
    \item Per ogni stringa $w \in L$, la macchina si ferma in uno stato accettante.
    \item Per ogni stringa $w \notin L$, la macchina non accetta $w$. Questo significa che la macchina può fermarsi in uno stato non accettante oppure non fermarsi affatto (loop infinito).
\end{itemize}
Le macchine che accettano linguaggi non danno una garanzia di terminazione per input che non fanno parte del linguaggio.

Per gli esercizi odierni, progetteremo macchine che \textbf{decidono} i linguaggi specificati.
\end{definition}

\section{Tecniche di Programmazione delle Macchine di Turing}

La programmazione delle Macchine di Turing richiede una visualizzazione chiara del loro funzionamento. Il suggerimento principale è "vedere il filmato" del movimento della testina e delle modifiche sul nastro nella propria mente, e poi tradurre questo filmato in una sequenza di transizioni di stato.

Un'altra tecnica importante è \textbf{evitare di cancellare i simboli di input} (sostituendoli con blank) se non strettamente necessario, specialmente per stringhe complesse. Questo può creare "buchi" sul nastro, rendendo difficile tenere traccia della posizione relativa della testina e delle parti rimanenti della stringa. È preferibile \textbf{marcare i simboli di input} con simboli speciali dell'alfabeto del nastro ($\Gamma$), mantenendo l'input compatto.

\subsection{Esercizio 1: Linguaggio dei Palindromi}
Il linguaggio $L = \{ww^R \mid w \in \{0,1\}^*\}$ è l'insieme delle stringhe palindromiche composte da due parti $w$ e $w^R$ concatenate. L'alfabeto è $\{0,1\}$. Esempi: $\varepsilon$, $00$, $11$, $0110$, $1001$.

\begin{example}[Strategia]
La strategia consiste nel confrontare il primo carattere di $w$ con l'ultimo carattere di $w^R$, il secondo con il penultimo, e così via, fino a quando tutti i caratteri sono stati confrontati.
\begin{enumerate}
    \item Partire dal primo simbolo non marcato del nastro.
    \item Leggere il simbolo (es. $0$ o $1$), marcarlo (sostituirlo con un Blank $\blankS$).
    \item Spostare la testina fino alla fine della stringa (trovando il primo \blankS a destra), quindi tornare indietro di un passo per posizionarsi sull'ultimo simbolo.
    \item Verificare che l'ultimo simbolo sia uguale a quello letto inizialmente. Se sì, marcarlo (sostituirlo con \blankS).
    \item Spostare la testina all'inizio della parte non marcata della stringa.
    \item Ripetere il processo fino a quando tutti i simboli sono stati marcati.
    \item Se tutti i simboli sono stati marcati correttamente, la stringa è accettata.
\end{enumerate}
\end{example}

\begin{tikzpicture}[shorten >=1pt,node distance=3cm,on grid,auto]
    \node[state,initial] (q0) {$q_0$};
    \node[state] (q1) [right=of q0] {$q_1$};
    \node[state] (q2) [right=of q1] {$q_2$};
    \node[state] (q3) [right=of q2] {$q_3$};
    \node[state] (q4) [below=of q0] {$q_4$};
    \node[state] (q5) [right=4cm of q4] {$q_5$};
    \node[state,accepting] (q6) [above=2cm of q0] {$q_6$};

    \path[->]
    (q0) edge node {$0/\blankS, R$} (q1)
         edge [left] node {$1/\blankS, R$} (q4)
         edge [bend left] node {$\blankS/\blankS, R$} (q6)
    (q1) edge [loop below] node {\begin{tabular}{c}$0/0, R$\\$1/1, R$\end{tabular}} ()
         edge node {$\blankS/\blankS, L$} (q2)
    (q2) edge node {$0/\blankS, L$} (q3)
    (q3) edge [loop right] node {\begin{tabular}{c}$0/0, L$\\$1/1, L$\end{tabular}} ()
         edge [bend right=30] node[above] {$\blankS/\blankS, R$} (q0)
    (q4) edge [loop left] node {\begin{tabular}{c}$0/0, R$\\$1/1, R$\end{tabular}} ()
         edge node {$\blankS/\blankS, L$} (q5)
    (q5) edge [bend left=15] node[right=0.3cm] {$1/\blankS, L$} (q3);
\end{tikzpicture}

\vspace{0.5cm}
\noindent \textbf{Spiegazione delle transizioni:}
\begin{itemize}
    \item \textbf{Stato $q_0$ (Iniziale):}
    \begin{itemize}
        \item Se legge $0$: scrive \blankS (blank), sposta a destra ($R$), va in $q_1$.
        \item Se legge $1$: scrive \blankS, sposta a destra ($R$), va in $q_4$.
        \item Se legge \blankS: scrive \blankS, sposta a destra ($R$), va in $q_6$ (stato accettante). Questo accetta la stringa vuota.
    \end{itemize}
    \item \textbf{Stato $q_1$ (Cerca fine stringa dopo aver letto $0$):}
    \begin{itemize}
        \item Se legge $0$ o $1$: lascia il simbolo, sposta a destra ($R$), rimane in $q_1$. Salta tutti i caratteri.
        \item Se legge \blankS: lascia \blankS, sposta a sinistra ($L$), va in $q_2$. Questo indica che la testina ha raggiunto la fine della stringa e si posiziona sull'ultimo carattere.
    \end{itemize}
    \item \textbf{Stato $q_2$ (Confronta ultimo carattere per $0$):}
    \begin{itemize}
        \item Se legge $0$: scrive \blankS, sposta a sinistra ($L$), va in $q_3$. Il $0$ iniziale è stato confrontato con successo.
        \item Se legge $1$ o \blankS: la stringa non è del formato corretto (implica rifiuto non fermandosi in stato accettante).
    \end{itemize}
    \item \textbf{Stato $q_3$ (Torna all'inizio):}
    \begin{itemize}
        \item Se legge $0$ o $1$: lascia il simbolo, sposta a sinistra ($L$), rimane in $q_3$. Salta tutti i caratteri.
        \item Se legge \blankS: lascia \blankS, sposta a destra ($R$), va in $q_0$. Questo indica che la testina ha raggiunto l'inizio del nastro e si posiziona sul prossimo carattere non marcato.
    \end{itemize}
    \item \textbf{Stato $q_4$ (Cerca fine stringa dopo aver letto $1$):}
    \begin{itemize}
        \item Se legge $0$ o $1$: lascia il simbolo, sposta a destra ($R$), rimane in $q_4$. Salta tutti i caratteri.
        \item Se legge \blankS: lascia \blankS, sposta a sinistra ($L$), va in $q_5$. Raggiunto fine stringa, posiziona sull'ultimo.
    \end{itemize}
    \item \textbf{Stato $q_5$ (Confronta ultimo carattere per $1$):}
    \begin{itemize}
        \item Se legge $1$: scrive \blankS, sposta a sinistra ($L$), va in $q_3$. L'$1$ iniziale è stato confrontato con successo.
        \item Se legge $0$ o \blankS: la stringa non è del formato corretto (implica rifiuto).
    \end{itemize}
    \item \textbf{Stato $q_6$ (Accettante):} Se la macchina raggiunge questo stato, la stringa è accettata.
\end{itemize}
Questa TM decide il linguaggio, poiché su ogni input valido o meno, si fermerà in uno stato accettante o non accettante.

\subsection{Esercizio 2}
Il linguaggio $L = \{a^n b^m \mid m > n > 0\}$ è l'insieme di stringhe con sequenze di $a$ seguite da sequenze di $b$, dove il numero di $b$ è strettamente maggiore del numero di $a$, ed entrambi sono maggiori di $0$. Esempio: $aab^{3}$, $ab^{2}$.

\begin{example}[Strategia]
La strategia prevede di marcare un'$a$ e poi la prima $b$ corrispondente, ripetendo il processo. Alla fine, si verifica che tutte le $a$ siano state marcate e che ci siano $b$ non marcate rimaste.
\begin{enumerate}
    \item Marcare la prima $a$ non marcata (es. con $X$).
    \item Spostarsi a destra, saltando altre $a$ e $Y$ (b già marcate) fino a trovare la prima $b$ non marcata.
    \item Marcare questa $b$ (es. con $Y$).
    \item Spostarsi a sinistra, saltando $a$, $b$, $Y$ fino a trovare la $X$ più a destra.
    \item Spostarsi a destra di una posizione per trovare la prossima $a$ non marcata e ripetere.
    \item Una volta che tutte le $a$ sono state marcate, si controlla che ci siano $b$ non marcate rimaste. Se sì, la stringa è accettata.
\end{enumerate}
\end{example}

\noindent \textbf{Alfabeto del nastro $\Gamma = \{a, b, X, Y, B\}$.}
\noindent \textbf{Stati:} $q_0$ (iniziale), $q_1$ (marcata $a$, cerca $b$), $q_2$ (salta $Y$ per trovare $b$), $q_3$ (marcata $b$, torna indietro), $q_4$ (torna indietro per $X$), $q_5$ (accettazione $b$ residue), $q_6$ (accettante).

\noindent \textbf{Transizioni:}
\begin{itemize}
    \item \textbf{Stato $q_0$ (Iniziale / Cerca $a$):}
        \begin{itemize}
            \item $q_0 \xrightarrow{a / X, R} q_1$ (Marca la prima $a$ con $X$, si muove a destra)
            \item $q_0 \xrightarrow{Y / Y, R} q_5$ (Se trova una $Y$, significa che tutte le $a$ sono state marcate e si controllano le $b$ residue)
            \item Se $q_0$ legge \blankS (stringa vuota) o \blankS (stringa non valida, $n > 0$) $\to$ implicitamente rifiuta.
        \end{itemize}
    \item \textbf{Stato $q_1$ (Marca $a$, cerca $b$):}
        \begin{itemize}
            \item $q_1 \xrightarrow{a / a, R} q_1$ (Salta altre $a$)
            \item $q_1 \xrightarrow{Y / Y, R} q_2$ (Salta $b$ già marcate con $Y$)
            \item $q_1 \xrightarrow{b / Y, L} q_3$ (Marca la prima $b$ con $Y$, va a sinistra. Questo è il caso in cui non ci sono $Y$ tra $a$ e $b$ non marcate)
        \end{itemize}
    \item \textbf{Stato $q_2$ (Salto $Y$ intermedie):}
        \begin{itemize}
            \item $q_2 \xrightarrow{Y / Y, R} q_2$ (Continua a saltare $Y$ già marcate)
            \item $q_2 \xrightarrow{b / Y, L} q_3$ (Marca la $b$ con $Y$, va a sinistra)
        \end{itemize}
    \item \textbf{Stato $q_3$ (Marca $b$, torna indietro):}
        \begin{itemize}
            \item $q_3 \xrightarrow{a / a, L} q_3$ (Salta $a$ mentre torna indietro)
            \item $q_3 \xrightarrow{Y / Y, L} q_3$ (Salta $Y$ mentre torna indietro)
            \item $q_3 \xrightarrow{X / X, R} q_0$ (Ha trovato l'$X$ della $a$ iniziale. Sposta a destra per la prossima $a$ da marcare)
        \end{itemize}
    \item \textbf{Stato $q_5$ (Verifica $b$ residue):}
        \begin{itemize}
            \item $q_5 \xrightarrow{Y / Y, R} q_5$ (Salta $Y$ residue)
            \item $q_5 \xrightarrow{b / b, R} q_5$ (Salta $b$ residue non marcate - devono esistere per $m>n$)
            \item $q_5 \xrightarrow{B / B, R} q_6$ (Se trova \blankS, significa che ci sono più $b$ delle $a$ ed è un blank. Stringa accettata)
        \end{itemize}
    \item \textbf{Stato $q_6$ (Accettante):} Stringa accettata.
\end{itemize}
Questa TM decide il linguaggio, rifiutando stringhe che non soddisfano le condizioni (es. $n \ge m$, ordine sbagliato dei caratteri, ecc.).

\subsection{Esercizio 3}
Il linguaggio $L = \{a^n b^m \mid n > 0, m = 2n\}$ è l'insieme di stringhe con sequenze di $a$ seguite da sequenze di $b$, dove il numero di $b$ è il doppio del numero di $a$, ed $n$ è maggiore di $0$. Esempio: $aab^{4}$, $ab^{2}$.

\begin{example}[Strategia]
Molto simile al precedente, ma per ogni $a$ marcata, si devono marcare due $b$.
\begin{enumerate}
    \item Marcare la prima $a$ non marcata (con $X$).
    \item Spostarsi a destra, saltando $a$ e $Y$ fino a trovare la prima $b$ non marcata. Marcarla (con $Y$).
    \item Spostarsi ancora a destra, trovare la seconda $b$ non marcata. Marcarla (con $Y$).
    \item Spostarsi a sinistra, saltando tutti i simboli marcati e non, fino a trovare la $X$ più a destra.
    \item Spostarsi a destra di una posizione per trovare la prossima $a$ non marcata e ripetere.
    \item Una volta che tutte le $a$ sono state marcate, si controlla che tutte le $b$ siano state marcate. Se sì, la stringa è accettata.
\end{enumerate}
\end{example}

\noindent \textbf{Alfabeto del nastro $\Gamma = \{a, b, X, Y, B\}$.}
\noindent \textbf{Stati:} $q_0$ (iniziale), $q_1$ (marcata $a$, cerca prima $b$), $q_2$ (marcata prima $b$, cerca seconda $b$), $q_3$ (marcata seconda $b$, torna indietro), $q_4$ (torna indietro), $q_5$ (accettazione $Y$ residue), $q_6$ (accettante).

\noindent \textbf{Transizioni:}
\begin{itemize}
    \item \textbf{Stato $q_0$ (Iniziale / Cerca $a$):}
        \begin{itemize}
            \item $q_0 \xrightarrow{a / X, R} q_1$ (Marca la prima $a$ con $X$)
            \item $q_0 \xrightarrow{Y / Y, R} q_5$ (Se trova $Y$, tutte le $a$ sono state marcate, si controllano solo $b$ marcate)
        \end{itemize}
    \item \textbf{Stato $q_1$ (Marca $a$, cerca prima $b$):}
        \begin{itemize}
            \item $q_1 \xrightarrow{a / a, R} q_1$ (Salta $a$)
            \item $q_1 \xrightarrow{Y / Y, R} q_1$ (Salta $Y$ già marcate)
            \item $q_1 \xrightarrow{b / Y, R} q_2$ (Marca la prima $b$ con $Y$, si muove a destra per cercare la seconda $b$)
        \end{itemize}
    \item \textbf{Stato $q_2$ (Cerca seconda $b$):}
        \begin{itemize}
            \item $q_2 \xrightarrow{b / Y, L} q_3$ (Marca la seconda $b$ con $Y$, si muove a sinistra per tornare)
        \end{itemize}
    \item \textbf{Stato $q_3$ (Marca seconda $b$, torna indietro):}
        \begin{itemize}
            \item $q_3 \xrightarrow{a / a, L} q_3$ (Salta $a$)
            \item $q_3 \xrightarrow{Y / Y, L} q_3$ (Salta $Y$)
            \item $q_3 \xrightarrow{X / X, R} q_0$ (Trovato $X$, sposta a destra per la prossima $a$)
        \end{itemize}
    \item \textbf{Stato $q_5$ (Verifica $b$ residue):}
        \begin{itemize}
            \item $q_5 \xrightarrow{Y / Y, R} q_5$ (Salta $Y$ residue)
            \item $q_5 \xrightarrow{B / B, R} q_6$ (Se trova \blankS, tutte le $b$ sono state marcate correttamente. Accetta)
        \end{itemize}
    \item \textbf{Stato $q_6$ (Accettante):} Stringa accettata.
\end{itemize}
Condizioni di rifiuto implicite: se non si trovano due $b$ per ogni $a$, o se l'ordine dei simboli non è $a$ poi $b$.

\subsection{Esercizio 4}
Il linguaggio $L = \{w\#w \mid w \in \{a,b\}^+\}$ è l'insieme di stringhe composte da una sequenza $w$, seguita da un separatore $\#$, e poi dalla stessa sequenza $w$. $w$ deve avere almeno un simbolo. Esempi: $a\#a$, $ab\#ab$, $baba\#baba$.

\begin{example}[Strategia]
La strategia si basa sul confronto carattere per carattere delle due metà della stringa, usando il $\#$ come punto di riferimento.
\begin{enumerate}
    \item Marcare il primo simbolo non marcato della prima $w$ (es. $a$ o $b$ con $X$).
    \item Spostarsi a destra, saltando tutti i simboli intermedi, fino a trovare il $\#$.
    \item Superare il $\#$, e poi saltare i simboli già marcati ($X$) nella seconda $w$.
    \item Trovare il simbolo corrispondente nella seconda $w$ (deve essere uguale a quello marcato nel passo 1), marcarlo (con $X$).
    \item Spostarsi a sinistra, saltando tutti i simboli intermedi, fino a trovare il $\#$.
    \item Superare il $\#$, e poi saltare i simboli già marcati ($X$) nella prima $w$.
    \item Posizionarsi sul prossimo simbolo non marcato nella prima $w$ e ripetere il ciclo.
    \item Se tutti i simboli sono stati marcati correttamente, la stringa è accettata.
\end{enumerate}
\end{example}

\noindent \textbf{Alfabeto del nastro $\Gamma = \{a, b, \#, X, B\}$.}
\noindent \textbf{Stati:} $q_0$ (iniziale), $q_1$ (marcata $a$, cerca match), $q_2$ (dopo $\#$, salta $X$, cerca $a$), $q_3$ (match trovato, torna indietro), $q_4$ (dopo $\#$, torna indietro a $X$), $q_5$ (check finale $X$ prima di $\#$), $q_6$ (marcata $b$, cerca match), $q_7$ (dopo $\#$, salta $X$, cerca $b$), $q_8$ (dopo $\#$, check finale $X$ dopo $\#$), $q_9$ (accettante).

\noindent \textbf{Transizioni:}
\begin{itemize}
    \item \textbf{Stato $q_0$ (Iniziale / Cerca simbolo in $w_1$):}
        \begin{itemize}
            \item $q_0 \xrightarrow{a / X, R} q_1$ (Marca $a$ con $X$, va a destra)
            \item $q_0 \xrightarrow{b / X, R} q_6$ (Marca $b$ con $X$, va a destra)
            \item $q_0 \xrightarrow{X / X, R} q_5$ (Tutte le prime $w$ sono marcate, passa alla fase di controllo finale)
        \end{itemize}
    \item \textbf{Stato $q_1$ (Marcata $a$, cerca $a$ in $w_2$):}
        \begin{itemize}
            \item $q_1 \xrightarrow{a / a, R} q_1$ (Salta $a$)
            \item $q_1 \xrightarrow{b / b, R} q_1$ (Salta $b$)
            \item $q_1 \xrightarrow{\# / \#, R} q_2$ (Passa il separatore $\#$)
        \end{itemize}
    \item \textbf{Stato $q_2$ (Dopo $\#$, cerca $a$ in $w_2$):}
        \begin{itemize}
            \item $q_2 \xrightarrow{X / X, R} q_2$ (Salta $X$ già marcate)
            \item $q_2 \xrightarrow{a / X, L} q_3$ (Trova $a$, marca con $X$, va a sinistra per tornare)
        \end{itemize}
    \item \textbf{Stato $q_6$ (Marcata $b$, cerca $b$ in $w_2$):}
        \begin{itemize}
            \item $q_6 \xrightarrow{a / a, R} q_6$ (Salta $a$)
            \item $q_6 \xrightarrow{b / b, R} q_6$ (Salta $b$)
            \item $q_6 \xrightarrow{\# / \#, R} q_7$ (Passa il separatore $\#$)
        \end{itemize}
    \item \textbf{Stato $q_7$ (Dopo $\#$, cerca $b$ in $w_2$):}
        \begin{itemize}
            \item $q_7 \xrightarrow{X / X, R} q_7$ (Salta $X$ già marcate)
            \item $q_7 \xrightarrow{b / X, L} q_3$ (Trova $b$, marca con $X$, va a sinistra per tornare)
        \end{itemize}
    \item \textbf{Stato $q_3$ (Match trovato, torna indietro):}
        \begin{itemize}
            \item $q_3 \xrightarrow{X / X, L} q_3$ (Salta $X$ già marcate)
            \item $q_3 \xrightarrow{a / a, L} q_3$ (Salta $a$)
            \item $q_3 \xrightarrow{b / b, L} q_3$ (Salta $b$)
            \item $q_3 \xrightarrow{\# / \#, L} q_4$ (Passa il separatore $\#$)
        \end{itemize}
    \item \textbf{Stato $q_4$ (Dopo $\#$, torna all'inizio di $w_1$):}
        \begin{itemize}
            \item $q_4 \xrightarrow{a / a, L} q_4$ (Salta $a$)
            \item $q_4 \xrightarrow{b / b, L} q_4$ (Salta $b$)
            \item $q_4 \xrightarrow{X / X, R} q_0$ (Trovata $X$ della $a$ o $b$ iniziale, si posiziona a destra per la prossima iterazione)
        \end{itemize}
    \item \textbf{Stato $q_5$ (Controllo finale, parte $w_1$):}
        \begin{itemize}
            \item $q_5 \xrightarrow{X / X, R} q_5$ (Salta $X$ nella prima metà)
            \item $q_5 \xrightarrow{\# / \#, R} q_8$ (Passa il separatore $\#$, ora controlla seconda metà)
        \end{itemize}
    \item \textbf{Stato $q_8$ (Controllo finale, parte $w_2$):}
        \begin{itemize}
            \item $q_8 \xrightarrow{X / X, R} q_8$ (Salta $X$ nella seconda metà)
            \item $q_8 \xrightarrow{B / B, R} q_9$ (Se trova \blankS, tutte le parti sono state marcate, stringa accettata)
        \end{itemize}
    \item \textbf{Stato $q_9$ (Accettante):} Stringa accettata.
\end{itemize}
Questa TM decide il linguaggio, rifiutando stringhe come $a\#b$ o $aa\#a$ o $a\#bb$.

\subsection{Esercizio 5}
Il linguaggio $L = \{a^n b^n c^n \mid n > 0\}$ è l'insieme di stringhe con un numero uguale di $a$, $b$, e $c$, in sequenza $a^*b^*c^*$. $n$ deve essere maggiore di $0$. Esempio: $abc$, $aabbcc$.

\begin{example}[Strategia]
La strategia è marcare un'$a$, poi una $b$, poi una $c$, e ripetere fino a quando tutti i caratteri sono stati marcati.
\begin{enumerate}
    \item Marcare la prima $a$ non marcata (con $X$).
    \item Spostarsi a destra, saltando altre $a$ e $Y$ (b già marcate), fino a trovare la prima $b$ non marcata. Marcarla (con $Y$).
    \item Spostarsi a destra, saltando altre $b$ e $Z$ (c già marcate), fino a trovare la prima $c$ non marcata. Marcarla (con $Z$).
    \item Spostarsi a sinistra, saltando tutti i simboli marcati e non, fino a trovare la $X$ più a destra.
    \item Spostarsi a destra di una posizione per trovare la prossima $a$ non marcata e ripetere.
    \item Una volta che tutte le $a$ sono state marcate, si controlla che tutte le $b$ e $c$ siano state marcate anch'esse.
\end{enumerate}
\end{example}

\noindent \textbf{Alfabeto del nastro $\Gamma = \{a, b, c, X, Y, Z, B\}$.}
\noindent \textbf{Stati:} $q_0$ (iniziale), $q_1$ (marcata $a$, cerca $b$), $q_2$ (salta $Y$, cerca $b$), $q_3$ (marcata $b$, cerca $c$), $q_4$ (salta $Z$, cerca $c$), $q_5$ (marcata $c$, torna indietro), $q_6$ (accettazione $Y$ residue), $q_7$ (accettazione $Z$ residue), $q_8$ (accettante).

\noindent \textbf{Transizioni:}
\begin{itemize}
    \item \textbf{Stato $q_0$ (Iniziale / Cerca $a$):}
        \begin{itemize}
            \item $q_0 \xrightarrow{a / X, R} q_1$ (Marca $a$ con $X$)
            \item $q_0 \xrightarrow{Y / Y, R} q_6$ (Se trova $Y$, tutte le $a$ sono state marcate, si controllano $b$ e $c$ marcate)
        \end{itemize}
    \item \textbf{Stato $q_1$ (Marca $a$, cerca $b$):}
        \begin{itemize}
            \item $q_1 \xrightarrow{a / a, R} q_1$ (Salta $a$)
            \item $q_1 \xrightarrow{Y / Y, R} q_2$ (Salta $Y$ già marcate)
            \item $q_1 \xrightarrow{b / Y, R} q_3$ (Marca la prima $b$ con $Y$. Caso iniziale senza $Y$ intermedie)
        \end{itemize}
    \item \textbf{Stato $q_2$ (Salto $Y$ intermedie):}
        \begin{itemize}
            \item $q_2 \xrightarrow{Y / Y, R} q_2$ (Continua a saltare $Y$ già marcate)
            \item $q_2 \xrightarrow{b / Y, R} q_3$ (Marca $b$ con $Y$)
        \end{itemize}
    \item \textbf{Stato $q_3$ (Marca $b$, cerca $c$):}
        \begin{itemize}
            \item $q_3 \xrightarrow{b / b, R} q_3$ (Salta $b$)
            \item $q_3 \xrightarrow{Z / Z, R} q_4$ (Salta $Z$ già marcate)
            \item $q_3 \xrightarrow{c / Z, L} q_5$ (Marca la prima $c$ con $Z$. Caso iniziale senza $Z$ intermedie)
        \end{itemize}
    \item \textbf{Stato $q_4$ (Salto $Z$ intermedie):}
        \begin{itemize}
            \item $q_4 \xrightarrow{Z / Z, R} q_4$ (Continua a saltare $Z$ già marcate)
            \item $q_4 \xrightarrow{c / Z, L} q_5$ (Marca $c$ con $Z$)
        \end{itemize}
    \item \textbf{Stato $q_5$ (Marca $c$, torna indietro):}
        \begin{itemize}
            \item $q_5 \xrightarrow{Z / Z, L} q_5$ (Salta $Z$)
            \item $q_5 \xrightarrow{c / c, L} q_5$ (Salta $c$)
            \item $q_5 \xrightarrow{Y / Y, L} q_5$ (Salta $Y$)
            \item $q_5 \xrightarrow{b / b, L} q_5$ (Salta $b$)
            \item $q_5 \xrightarrow{a / a, L} q_5$ (Salta $a$)
            \item $q_5 \xrightarrow{X / X, R} q_0$ (Trovato $X$, sposta a destra per la prossima $a$)
        \end{itemize}
    \item \textbf{Stato $q_6$ (Verifica $Y$ residue):}
        \begin{itemize}
            \item $q_6 \xrightarrow{Y / Y, R} q_6$ (Salta $Y$ residue)
            \item $q_6 \xrightarrow{Z / Z, R} q_7$ (Se trova $Z$, passa a controllare le $c$)
        \end{itemize}
    \item \textbf{Stato $q_7$ (Verifica $Z$ residue):}
        \begin{itemize}
            \item $q_7 \xrightarrow{Z / Z, R} q_7$ (Salta $Z$ residue)
            \item $q_7 \xrightarrow{B / B, R} q_8$ (Se trova \blankS, tutte le $a$, $b$, $c$ sono state marcate correttamente. Accetta)
        \end{itemize}
    \item \textbf{Stato $q_8$ (Accettante):} Stringa accettata.
\end{itemize}
Questa TM decide il linguaggio. Implicitamente rifiuta stringhe con ordine errato dei caratteri (es. $acb$) o conteggi non corrispondenti (es. $aabc$).

\end{document}