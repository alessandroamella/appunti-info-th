\documentclass[a4paper]{article}
\usepackage{amsmath, amssymb, amsfonts}
\usepackage[italian]{babel}
\usepackage[utf8]{inputenc}
\usepackage{graphicx}
\usepackage{hyperref}
\usepackage[scaled]{helvet}
\renewcommand{\familydefault}{\sfdefault}
\usepackage[T1]{fontenc}
\usepackage[a4paper, total={6in, 8in}]{geometry}
\usepackage{minted}
\usepackage{mathpazo}
\usepackage{fancyhdr}
\usepackage{amsthm}
\usepackage{tikz}
\usetikzlibrary{
    automata,
    positioning,
    arrows,
    arrows.meta,
    calc,
    fit,
    patterns,
    graphs,
    graphdrawing,
    matrix,
    quotes,
    shapes.geometric,
    snakes,
    shapes,
    decorations.pathreplacing,
    calligraphy,
}
% Definizione degli ambienti
\theoremstyle{definition} % Use definition style for normal (non-italic) text
\newtheorem{theorem}{Teorema}
[section]
\newtheorem{definition}{Definizione}[section]
\newtheorem{example}{Esempio}[section]
\newtheorem{lemma}{Lemma}
\newtheorem{remark}[theorem]{Osservazione}
\newtheorem{proposition}{Proposizione}
\newcommand{\blankS}{\ensuremath{\raisebox{-0.15ex}{\scalebox{1.3}[0.7]{$\sqcup$}}\mkern2mu}}

\hypersetup{
    pdftitle={Lezione di Informatica Teorica: Automi a Stati Finiti e Non-Deterministici},
    pdfauthor={Appunti da Trascrizione AI}
}

\pagestyle{fancy}
\fancyhf{}
\fancyhead[L]{\textit{Lezione di Informatica Teorica}}
\fancyfoot[C]{\thepage}

\title{Lezione di Informatica Teorica: Automi a Stati Finiti}
\author{Appunti da Trascrizione Automatica}
\date{\today}

\begin{document}
\maketitle
\tableofcontents
\newpage

\section{Automi a Stati Finiti Deterministici (DFA)}
Come visto nella lezione precedente, gli automi a stati finiti deterministici (DFA) sono modelli di calcolo che accettano o rifiutano stringhe di input basandosi su un insieme di stati e transizioni definite.

\subsection{Computazione di un DFA}
Per capire il funzionamento di un automa, modelliamo i suoi passi attraverso il concetto di "configurazione".

\begin{definition}{Configurazione}{configurazione}
Una \emph{configurazione} di un DFA è una coppia $(q, w)$, dove:
\begin{itemize}
    \item $q \in Q$ è lo stato corrente in cui si trova l'automa.
    \item $w \in \Sigma^*$ è la porzione di stringa di input che deve ancora essere letta.
\end{itemize}
Una configurazione rappresenta uno "snapshot" dello stato di avanzamento della computazione.
\end{definition}

\begin{definition}{Computazione Parziale}{computazione-parzial}
Una \emph{computazione parziale} di un DFA $D$ su una stringa $w = c_1 c_2 \dots c_n$ è una sequenza di $m+1$ configurazioni:
\[ (R_0, c_1 \dots c_n) \xrightarrow{} (R_1, c_2 \dots c_n) \xrightarrow{} \dots \xrightarrow{} (R_m, c_{m+1} \dots c_n) \]
tale che:
\begin{itemize}
    \item $R_0 = q_0$ (la computazione inizia dallo stato iniziale con l'intera stringa in input).
    \item Per ogni $0 \leq i < m$, si ha $R_{i+1} = \delta(R_i, c_{i+1})$.
\end{itemize}

\begin{remark}{}{{ remark-lecture-03-1 }}
La mossa $\delta(R_i, c_{i+1})$ determina un \emph{unico} stato successivo.
\end{remark}

Il simbolo $c_{i+1}$ viene letto e consumato, e l'automa si sposta nello stato $R_{i+1}$.
\end{definition}

\begin{example}{Traccia di Computazione per $001$}{traccia-di-computazi}
Usiamo il DFA per numeri dispari e la stringa $w = 001$.
\begin{itemize}
    \item Passo 0: $(Q_0, 001)$ \quad (Configurazione iniziale)
    \item Passo 1: $(Q_0, 01)$ \quad ($\delta(Q_0, 0) = Q_0$)
    \item Passo 2: $(Q_0, 1)$ \quad ($\delta(Q_0, 0) = Q_0$)
    \item Passo 3: $(Q_1, \epsilon)$ \quad ($\delta(Q_0, 1) = Q_1$)
\end{itemize}
L'ultima configurazione è $(Q_1, \epsilon)$.
\end{example}

\begin{definition}{Computazione Completa}{computazione-complet}
Una \emph{computazione completa} (o semplicemente \emph{computazione}) è una computazione parziale che è massimale, cioè non può essere estesa ulteriormente. Questo si verifica in due casi:
\begin{itemize}
    \item La stringa di input è stata completamente consumata (la porzione residua è $\epsilon$).
    \item Non esiste una transizione definita per lo stato corrente e il simbolo da leggere (l'automa si "blocca").
\end{itemize}
\end{definition}

\begin{definition}{Accettazione di una Stringa (DFA)}{accettazione-di-una-}
Un DFA $D$ \emph{accetta} una stringa $w$ se la sua unica computazione completa, che inizia dalla configurazione $(q_0, w)$, termina in una configurazione $(q_f, \epsilon)$ tale che $q_f \in F$.
In altre parole, due condizioni devono essere soddisfatte:
\begin{enumerate}
    \item L'intera stringa di input $w$ deve essere consumata.
    \item Lo stato finale raggiunto dall'automa deve essere uno stato accettante ($q_f \in F$).
\end{enumerate}
\end{definition}

\begin{definition}{Rifiuto di una Stringa (DFA)}{rifiuto-di-una-strin}
Un DFA $D$ \emph{rifiuta} una stringa $w$ se la computazione su $w$ non è accettante. Ciò può avvenire se:
\begin{itemize}
    \item L'automa termina in uno stato $q \notin F$ pur avendo consumato tutto l'input.
    \item L'automa si blocca (non ha transizioni definite) prima di aver consumato tutto l'input.
\end{itemize}
\end{definition}

\begin{definition}{DFA che Decide un Linguaggio}{dfa-che-decide-un-li}
Un DFA $D$ \emph{decide} un linguaggio $L$ se per ogni stringa $w \in L$, $D$ accetta $w$, e per ogni stringa $w \notin L$, $D$ rifiuta $w$.
\end{definition}

\section{Automi a Stati Finiti Non-Deterministici (NFA)}
Gli automi non-deterministici introducono il concetto di scelte multiple durante la computazione.

\subsection{Introduzione al Non-Determinismo}
Consideriamo il linguaggio delle stringhe binarie che terminano con $000$, espresso dalla espressione regolare $(0|1)^*000$.

\begin{example}{NFA per $(0|1)^*000$}{nfa-per}
Un tentativo intuitivo di DFA potrebbe essere:
\begin{itemize}
    \item $Q_0$: Stato iniziale e per leggere $(0|1)^*$.
    \item $Q_1$: Dopo aver letto il primo $0$ dei $000$ finali.
    \item $Q_2$: Dopo aver letto il secondo $0$ dei $000$ finali.
    \item $Q_3$: Stato finale, dopo aver letto il terzo $0$ dei $000$ finali.
\end{itemize}
Le transizioni sarebbero:
\begin{itemize}
    \item Da $Q_0$:
        \begin{itemize}
            \item Se legge $0$: può rimanere in $Q_0$ (per $(0|1)^*$) \emph{oppure} andare in $Q_1$ (se è il primo $0$ della sequenza finale). Questo è un punto di non-determinismo.
            \item Se legge $1$: rimane in $Q_0$.
        \end{itemize}
    \item Da $Q_1$:
        \begin{itemize}
            \item Se legge $0$: va in $Q_2$.
            \item Se legge $1$: torna in $Q_0$ (la sequenza $000$ è interrotta).
        \end{itemize}
    \item Da $Q_2$:
        \begin{itemize}
            \item Se legge $0$: va in $Q_3$.
            \item Se legge $1$: torna in $Q_0$.
        \end{itemize}
    \item Da $Q_3$:
        \begin{itemize}
            \item Se legge $0$: rimane in $Q_3$ (ma in realtà non è così semplice per $(0|1)^*000$ per via dei $000$ finali). Per la semplicità dell'esempio del professore, si può immaginare che da $Q_3$ si accettino solo stringhe che finiscono esattamente con $000$ e nessun altro carattere dopo. Se la stringa è $010001$ dopo $Q_3$ dovrebbe andare in $Q_0$.
        \end{itemize}
\end{itemize}
Il problema evidenziato dal professore è il \emph{non-determinismo} in $Q_0$ quando legge $0$: può andare in $Q_0$ o $Q_1$.
\end{example}

\begin{definition}{Automa a Stati Finiti Non-Deterministico (NFA)}{automa-a-stati-finit}
Un automa a stati finiti non-deterministico $N$ è una quintupla $( \Sigma, Q, q_0, F, \delta )$, dove:
\begin{itemize}
    \item $\Sigma$, $Q$, $q_0$, $F$ sono definiti come per i DFA.
    \item $\delta: Q \times \Sigma \to \mathcal{P}(Q)$ è la funzione di transizione (o relazione di transizione). Per ogni coppia (stato corrente, simbolo letto), $\delta$ determina un \textbf{insieme} di possibili prossimi stati ($\mathcal{P}(Q)$ denota l'insieme delle parti di $Q$).
\end{itemize}
\end{definition}

\subsection{Computazione di un NFA}
A causa delle transizioni che possono portare a più stati, per un dato input un NFA può avere \emph{più computazioni differenti}.

\begin{definition}{Computazione Parziale per NFA}{computazione-parzial}
Una \emph{computazione parziale} di un NFA $N$ su una stringa $w = c_1 c_2 \dots c_n$ è una sequenza di $m+1$ configurazioni:
\[ (R_0, c_1 \dots c_n) \xrightarrow{} (R_1, c_2 \dots c_n) \xrightarrow{} \dots \xrightarrow{} (R_m, c_{m+1} \dots c_n) \]
tale che:
\begin{itemize}
    \item $R_0 = q_0$ (la computazione inizia dallo stato iniziale con l'intera stringa in input).
    \item Per ogni $0 \leq i < m$, si ha $R_{i+1} \in \delta(R_i, c_{i+1})$. (Il prossimo stato è \textbf{uno qualsiasi} tra quelli ammessi dalla relazione di transizione).
\end{itemize}
\end{definition}

\begin{example}{Traccia di Computazione per $0100$ su NFA $(0|1)^*000$}{traccia-di-computazi}
Consideriamo l'NFA precedente e la stringa $w = 0100$.
\begin{itemize}
    \item \textbf{Percorso Accettante (se esistesse Q3 come accettante):}
    \item $(Q_0, 0100)$
    \item $\xrightarrow{\text{leggi 0, transisci in } Q_0}$ $(Q_0, 100)$
    \item $\xrightarrow{\text{leggi 1, transisci in } Q_0}$ $(Q_0, 00)$
    \item $\xrightarrow{\text{leggi 0, transisci in } Q_1}$ $(Q_1, 0)$
    \item $\xrightarrow{\text{leggi 0, transisci in } Q_2}$ $(Q_2, \epsilon)$
    \item $\xrightarrow{\text{leggi } \epsilon \text{, transisci in } Q_3}$ $(Q_3, \epsilon)$
    Questo percorso, se $Q_3$ è accettante, porta all'accettazione.

    \item \textbf{Percorso Rifiutante:}
    \item $(Q_0, 0100)$
    \item $\xrightarrow{\text{leggi 0, transisci in } Q_1}$ $(Q_1, 100)$
    \item Ora, da $Q_1$ leggendo $1$, si torna in $Q_0$.
    \item $\xrightarrow{\text{leggi 1, transisci in } Q_0}$ $(Q_0, 00)$
    *A questo punto l'automa non è riuscito a finire la sequenza $000$ nel modo sperato, e dovrebbe continuare a processare per vedere se trova altri $000$.*
    *Esempio più semplice: se da $Q_1$ non ci fosse una transizione per $1$, la macchina si bloccherebbe.*
\end{itemize}
\end{example}

\begin{definition}{Accettazione di una Stringa (NFA)}{accettazione-di-una-}
Un NFA $N$ \emph{accetta} una stringa $w$ se \textbf{esiste almeno una} computazione completa per $w$ che sia accettante (ovvero, termina in uno stato finale $q_f \in F$ con input esaurito $\epsilon$).
\end{definition}
\noindent \textbf{Nota}: Un NFA rifiuta una stringa $w$ se \emph{tutte} le possibili computazioni per $w$ sono rifiutanti.

\subsubsection{La Natura del Non-Determinismo}
È cruciale comprendere che gli NFA sono \emph{modelli astratti} di calcolo. Non implicano che una macchina fisica "provi tutte le strade contemporaneamente", "tiri a indovinare" o "faccia backtracking". La loro utilità risiede nella \textbf{semplicità di progettazione} e analisi concettuale.

\begin{theorem}{Equivalenza DFA-NFA}{equivalenza-dfanfa}
Per ogni Automa a Stati Finiti Non-Deterministico (NFA), esiste un Automa a Stati Finiti Deterministico (DFA) equivalente che riconosce lo stesso linguaggio.
\end{theorem}
Questo teorema, sebbene non dimostrato qui, è fondamentale: significa che DFA e NFA hanno la stessa \emph{potenza espressiva}. Qualsiasi linguaggio riconoscibile da un NFA è riconoscibile anche da un DFA, e viceversa. Questo giustifica l'uso degli NFA per la loro semplicità di programmazione, sapendo che possono essere sempre convertiti in DFA per l'implementazione pratica.

\section{Limitazioni degli Automi a Stati Finiti}
Nonostante la loro utilità, gli automi a stati finiti (sia deterministici che non-deterministici) non sono modelli di calcolo universali. Hanno delle limitazioni intrinseche.

\begin{example}{Linguaggio $L = \{a^m b^m \mid m \geq 0\}$}{linguaggio}
Consideriamo il linguaggio $L$ sull'alfabeto $\Sigma = \{a, b\}$ definito come:
\[ L = \{a^m b^m \mid m \geq 0\} \]
Questo linguaggio include stringhe come $\epsilon$ (per $m=0$), $ab$, $aabb$, $aaabbb$, e così via. Ogni stringa in $L$ è composta da un certo numero di $a$ seguito esattamente dallo stesso numero di $b$.
Un programma Python o Java saprebbe facilmente verificare se una stringa è di questo tipo, ad esempio contando il numero di $a$ e $b$.
\end{example}

\subsection{Pumping Lemma}
Gli automi a stati finiti non sono in grado di riconoscere il linguaggio $L = \{a^m b^m \mid m \geq 0\}$. L'intuizione dietro questa limitazione è che gli automi a stati finiti \textbf{non sanno contare} in modo arbitrario e \textbf{non hanno memoria esterna} per ricordare conteggi indefiniti.

\begin{example}{DFA che riconosce solo $L_{limitato} = \{a^m b^m \mid 0 \leq m \leq 3\}$}{dfa-che-riconosce-so}
Per comprendere meglio la limitazione, consideriamo un DFA che può riconoscere solo una versione \emph{limitata} del linguaggio, ovvero $L_{limitato} = \{\epsilon, ab, aabb, aaabbb\}$:

\begin{center}
\begin{tikzpicture}[
    ->,
    >=stealth,
    node distance=2cm,
    semithick,
    every state/.style={draw,circle,minimum size=20pt}
]
    % Stati
    \node[state,initial,accepting] (q0) {$q_0$};
    \node[state,right of=q0] (q1) {$q_1$};
    \node[state,accepting,below of=q1] (q2) {$q_2$};
    \node[state,right of=q1] (q3) {$q_3$};
    \node[state,accepting,below of=q3] (q4) {$q_4$};
    \node[state,right of=q3] (q5) {$q_5$};
    \node[state,accepting,below of=q5] (q6) {$q_6$};
    \node[state,below of=q2] (trap) {$\text{trap}$};
    
    % Transizioni per le a
    \path (q0) edge[above] node{$a$} (q1);
    \path (q1) edge[above] node{$a$} (q3);
    \path (q3) edge[above] node{$a$} (q5);
    
    % Transizioni per le b
    \path (q1) edge[left] node{$b$} (q2);
    \path (q3) edge[left] node{$b$} (q4);
    \path (q5) edge[right] node{$b$} (q6);
    
    % Transizioni verso il pozzo
    \path (q0) edge[bend right=45,below left] node{$b$} (trap);
    \path (q2) edge[left] node{$a,b$} (trap);
    \path (q4) edge[bend left=30,above left] node{$a,b$} (trap);
    \path (q6) edge[bend left=45,below] node{$a,b$} (trap);
    \path (q5) edge[bend left=30,right] node{$a$} (trap);
    \path (trap) edge[loop below] node{$a,b$} (trap);
\end{tikzpicture}
\end{center}

Questo automa ha 8 stati e può riconoscere solo stringhe con al massimo 3 $a$ seguite da 3 $b$. Per riconoscere il linguaggio completo $L = \{a^m b^m \mid m \geq 0\}$, avremmo bisogno di infiniti stati!
\end{example}

\begin{itemize}
    \item \textbf{Assunzione per contraddizione}: Supponiamo che esista un DFA $D$ in grado di decidere il linguaggio $L = \{a^m b^m \mid m \geq 0\}$.
    \item Sia $P$ il numero di stati del DFA $D$. $P$ è un numero finito e costante.
    \item \textbf{Consideriamo una stringa lunga}: Scegliamo una stringa $s = a^P b^P$. Questa stringa appartiene a $L$ ed ha lunghezza $2P$. Poiché $P \geq 1$ (assumendo almeno uno stato), $2P > P$.
    \item \textbf{Il Principio dei Cassetti (Pigeonhole Principle)}: Quando il DFA $D$ processa i primi $P$ simboli $a$ della stringa $s$, esso visiterà $P+1$ configurazioni (partendo da $q_0$ e leggendo $P$ $a$). Poiché il DFA ha solo $P$ stati, per il principio dei cassetti, l'automa deve \textbf{per forza rivisitare almeno uno stato} durante il processo delle prime $P$ $a$.
    \item Questo significa che esiste un \textbf{ciclo} nel percorso di computazione dell'automa mentre sta leggendo la porzione $a^P$ della stringa. Possiamo scomporre la stringa $s$ in tre parti: $s = uvw$, dove:
        \begin{itemize}
            \item $u$ è la parte iniziale della stringa prima che si entri nel ciclo (potrebbe essere vuota).
            \item $v$ è la porzione di stringa che corrisponde al ciclo (contiene solo $a$ e non è vuota).
            \item $w$ è la parte restante della stringa dopo aver completato il ciclo una volta, che include le $a$ rimanenti (se ci sono) e tutte le $b$.
        \end{itemize}
        Graficamente: $q_0 \xrightarrow{u} q_i \xrightarrow{v} q_i \xrightarrow{w} q_f \in F$.

\begin{example}{Visualizzazione del Ciclo Forzato}{visualizzazione-del-}
Supponiamo di avere un DFA con $P = 3$ stati che cerca di decidere $L = \{a^m b^m \mid m \geq 0\}$. Consideriamo la stringa $s = a^3 b^3 = aaabbb$:

\begin{center}
\begin{tikzpicture}[
    ->,
    >=stealth,
    node distance=2cm,
    semithick,
    every state/.style={draw,circle,minimum size=25pt}
]
    % Stati del DFA ipotetico
    \node[state,initial] (q0) {$q_0$};
    \node[state,right of=q0] (q1) {$q_1$};
    \node[state,right of=q1] (q2) {$q_2$};
    
    % Percorso forzato per aaa
    \path (q0) edge[above] node{$a$} (q1);
    \path (q1) edge[above] node{$a$} (q2);
    \path (q2) edge[below,bend left,red,thick] node{$a$} (q0);
    
    % Evidenziamo il ciclo
    \node[above=0.5cm of q1] {\textcolor{red}{Ciclo forzato!}};
    \node[below=1cm of q1] {$u = \epsilon$, $v = aaa$, $w = bbb$};
\end{tikzpicture}
\end{center}

Leggendo $aaa$, l'automa passa attraverso $q_0 \to q_1 \to q_2 \to q_0$. Il ciclo $v = aaa$ può essere ripetuto, generando stringhe come $a^6 b^3$, $a^9 b^3$, etc., che non appartengono a $L$.
\end{example}
    \item \textbf{Implicazione del ciclo}: Se il DFA accetta $uvw$, dato che $v$ corrisponde a un ciclo, l'automa accetterà anche $uvvw$, $uvvvw$, e in generale $uv^i w$ per qualsiasi $i \geq 0$. Questo perché percorrere il ciclo ($v$) più volte riporta sempre l'automa nello stesso stato $q_i$, e da lì la computazione prosegue identica.
    \item \textbf{Contraddizione}: Poiché $v$ contiene solo $a$ (essendo parte dei primi $P$ simboli $a$), se consideriamo la stringa $uv^2 w$, essa avrà più $a$ di $uvw$ ma lo stesso numero di $b$. Ad esempio, se $s = a^P b^P$, allora $uv^2 w$ sarà della forma $a^{P+|v|} b^P$, dove $|v| > 0$. Questa nuova stringa \textbf{non appartiene} al linguaggio $L$ perché il numero di $a$ non è uguale al numero di $b$.
    \item Tuttavia, per la proprietà dei cicli, il DFA $D$ dovrebbe accettare $uv^2 w$. Questo è una contraddizione, in quanto $D$ dovrebbe accettare solo stringhe in $L$ e rifiutare quelle non in $L$.
\end{itemize}
\textbf{Conclusione}: L'assunzione iniziale che un DFA possa decidere il linguaggio $L = \{a^m b^m \mid m \geq 0\}$ è falsa. Gli automi a stati finiti non hanno la capacità di "contare" in modo arbitrario e confrontare quantità.

Questa limitazione fondamentale degli automi a stati finiti ci spinge a cercare modelli di calcolo più potenti, in grado di svolgere compiti come il conteggio e la memorizzazione di informazioni arbitrarie. Il prossimo passo è l'introduzione delle \emph{Macchine di Turing}, che si dimostreranno essere un modello di calcolo universale, in grado di simulare qualsiasi algoritmo.

\end{document}