\documentclass[a4paper]{article}
\usepackage{amsmath, amssymb, amsfonts}
\usepackage[italian]{babel}
\usepackage[utf8]{inputenc}
\usepackage{graphicx}
\usepackage{hyperref}
\usepackage[a4paper, total={6in, 8in}]{geometry}
\usepackage{minted}
\usepackage{mathpazo} % For a nicer font
\usepackage{fancyhdr}
\usepackage{amsthm}
\usepackage{tikz}
\usetikzlibrary{automata,positioning, arrows.meta} % Added arrows.meta for better arrow styles

% Definizione degli ambienti
\newtheorem{theorem}{Teorema}[section] % Number theorems by section
\newtheorem{definition}{Definizione}[section] % Number definitions by section
\newtheorem{example}{Esempio}[section]
\newtheorem{lemma}{Lemma}[section]
\newtheorem{proposition}{Proposizione}[section]

% Custom proof environment (optional, amsthm provides one)
\makeatletter
\renewenvironment{proof}[1][\proofname]{\par
  \pushQED{\qed}%
  \normalfont \topsep6\p@\@plus6\p@\relax
  \trivlist
  \item[\hskip\labelsep
        \bfseries
    #1\@addpunct{.}]\ignorespaces
}{%
  \popQED\endtrivlist\@endpefalse
}
\makeatother

\hypersetup{
    pdftitle={Lezione di Informatica Teorica},
    pdfauthor={Appunti da Trascrizione AI},
}

\pagestyle{fancy}
\fancyhf{}
\fancyhead[L]{\textit{Lezione di Informatica Teorica}}
\fancyfoot[C]{\thepage}

\title{Lezione di Informatica Teorica: Decidibilità e Indecidibilità}
\author{Appunti da Trascrizione Automatica}
\date{\today}

\begin{document}
\maketitle
\tableofcontents
\newpage

\section{Linguaggio Universale (LU)}

Introduciamo il Linguaggio Universale.
\begin{definition}[Linguaggio Universale (LU)]
Il linguaggio universale $LU$ è l'insieme delle coppie $\langle M, w \rangle$, dove $M$ è una Macchina di Turing e $w$ è una stringa, tali che $M$ accetta $w$. L'operatore "accetta" (indicato anche con $\models$) implica che $M$ si arresta in uno stato accettante su $w$.
\[ LU = \{ \langle M, w \rangle \mid M \text{ accetta } w \} \]
\end{definition}

\begin{theorem}
Il linguaggio universale $LU$ è ricorsivamente enumerabile (ovvero, $LU \in RE$).
\end{theorem}
\begin{proof}
Per dimostrare che $LU \in RE$, dobbiamo mostrare l'esistenza di una Macchina di Turing che lo accetti. Tale macchina è la Macchina di Turing Universale (UTM).
\begin{enumerate}
    \item La UTM prende in input la codifica $\langle M, w \rangle$.
    \item La UTM simula passo-passo il comportamento di $M$ su $w$.
    \item Se $M$ si arresta in uno stato accettante su $w$, la UTM si arresta e accetta (risponde "sì").
    \item Se $M$ si arresta in uno stato non accettante su $w$, la UTM si arresta e rifiuta (risponde "no").
    \item Se $M$ non si arresta su $w$ (entra in un loop infinito), la UTM simulerà all'infinito e non si arresterà (non darà risposta "sì").
\end{enumerate}
Poiché la UTM accetta tutte e sole le istanze di $LU$, il linguaggio $LU$ è ricorsivamente enumerabile.
\end{proof}

Ora, la domanda cruciale: $LU \in R$? Intuitivamente, se $M$ va in loop su $w$, la UTM non si arresterà per dare una risposta "no". Questo suggerisce che $LU$ non sia decidibile.

\begin{theorem}
Il linguaggio universale $LU$ non è ricorsivo (ovvero, $LU \notin R$).
\end{theorem}
\begin{proof}
La dimostrazione procede per assurdo.

\textbf{Assunzione per assurdo:} Supponiamo che $LU \in R$.
\begin{enumerate}
    \item Se $LU \in R$, allora per le proprietà di chiusura delle classi di linguaggi, anche il suo complemento $\overline{LU}$ deve essere in $R$.
    \item Se $\overline{LU} \in R$, allora esiste una Macchina di Turing $M_{\overline{LU}}$ che \textbf{decide} $\overline{LU}$. Ciò significa che $M_{\overline{LU}}$ si arresta sempre (su ogni input) e dà una risposta corretta (sì/no).
    \item Costruiamo una nuova Macchina di Turing $M'$ che prende in input una codifica di macchina $\langle M_i \rangle$. Il comportamento di $M'$ è il seguente:
    \begin{enumerate}
        \item Riceve $\langle M_i \rangle$ come input.
        \item Crea una copia di $\langle M_i \rangle$ e la usa come stringa $w$. Forma la coppia $\langle M_i, \langle M_i \rangle \rangle$. (Quindi l'input per la fase successiva è $\langle M_i, w \rangle$ dove $w = \langle M_i \rangle$).
        \item Dà in input la coppia $\langle M_i, \langle M_i \rangle \rangle$ alla macchina $M_{\overline{LU}}$ (la cui esistenza è garantita dalla nostra assunzione).
        \item $M'$ adotta la risposta di $M_{\overline{LU}}$:
        \begin{itemize}
            \item Se $M_{\overline{LU}}$ risponde "sì", allora $M'$ risponde "sì".
            \item Se $M_{\overline{LU}}$ risponde "no", allora $M'$ risponde "no".
        \end{itemize}
    \end{enumerate}
\end{enumerate}
Ora analizziamo il linguaggio deciso da $M'$, $L(M')$:
\begin{itemize}
    \item \textbf{Se $M'$ risponde "sì":}
    Ciò significa che $M_{\overline{LU}}$ ha risposto "sì" sull'input $\langle M_i, \langle M_i \rangle \rangle$.
    Poiché $M_{\overline{LU}}$ decide $\overline{LU}$, la risposta "sì" implica che $\langle M_i, \langle M_i \rangle \rangle \in \overline{LU}$.
    Per definizione di $\overline{LU}$, ciò significa che $M_i$ \textbf{non accetta} $\langle M_i \rangle$.
    \item \textbf{Se $M'$ risponde "no":}
    Ciò significa che $M_{\overline{LU}}$ ha risposto "no" sull'input $\langle M_i, \langle M_i \rangle \rangle$.
    Poiché $M_{\overline{LU}}$ decide $\overline{LU}$, la risposta "no" implica che $\langle M_i, \langle M_i \rangle \rangle \notin \overline{LU}$, ovvero $\langle M_i, \langle M_i \rangle \rangle \in LU$.
    Per definizione di $LU$, ciò significa che $M_i$ \textbf{accetta} $\langle M_i \rangle$.
\end{itemize}
Il comportamento di $M'$ è esattamente quello di una macchina che decide il linguaggio diagonale $L_D$. Infatti, $M'$ risponde "sì" se $M_i$ non accetta $\langle M_i \rangle$, e "no" se $M_i$ accetta $\langle M_i \rangle$. Dunque, $L(M') = L_D$.

Poiché $M'$ è costruita usando $M_{\overline{LU}}$ (che è un decider e si arresta sempre), $M'$ è anch'essa una Macchina di Turing che si arresta sempre, ovvero un decider. Questo implicherebbe che $L_D \in R$.

\textbf{Contraddizione:} Sappiamo che $L_D \notin RE$, e quindi a maggior ragione $L_D \notin R$.
L'assunzione iniziale ($LU \in R$) deve essere falsa.

\textbf{Conclusione:} $LU \notin R$.
\end{proof}

\subsection{Il Complemento del Linguaggio Diagonale ($\overline{L_D}$)}
Consideriamo ora il complemento di $L_D$:
\begin{definition}[Complemento del Linguaggio Diagonale ($\overline{L_D}$)]
Il complemento del linguaggio diagonale $\overline{L_D}$ è l'insieme delle codifiche di Macchine di Turing $M_i$ tali che $M_i$ accetta la propria codifica $\langle M_i \rangle$.
\[ \overline{L_D} = \{ \langle M_i \rangle \mid M_i \text{ accetta } \langle M_i \rangle \} \]
\end{definition}

\begin{theorem}
Il complemento del linguaggio diagonale, $\overline{L_D}$, è ricorsivamente enumerabile (ovvero, $\overline{L_D} \in RE$).
\end{theorem}
\begin{proof}
Per dimostrare che $\overline{L_D} \in RE$, dobbiamo costruire una Macchina di Turing $M_{\overline{L_D}}$ che accetti $\overline{L_D}$.

\textbf{Costruzione di $M_{\overline{L_D}}$:}
\begin{enumerate}
    \item $M_{\overline{L_D}}$ prende in input la codifica di una Macchina di Turing $\langle M_i \rangle$.
    \item $M_{\overline{L_D}}$ crea una copia di $\langle M_i \rangle$ per usarla come stringa $w$. Forma quindi la coppia $\langle M_i, \langle M_i \rangle \rangle$.
    \item $M_{\overline{L_D}}$ simula $M_i$ su $\langle M_i \rangle$ usando una UTM.
    \item Se la simulazione di $M_i$ su $\langle M_i \rangle$ si arresta e accetta, allora $M_{\overline{L_D}}$ accetta (risponde "sì").
    \item Se la simulazione di $M_i$ su $\langle M_i \rangle$ si arresta e rifiuta, o entra in loop, allora $M_{\overline{L_D}}$ non accetta (non risponde "sì").
\end{enumerate}
\textbf{Analisi del comportamento di $M_{\overline{L_D}}$:}
\begin{itemize}
    \item \textbf{Se $\langle M_i \rangle \in \overline{L_D}$:}
    Per definizione, $M_i$ accetta $\langle M_i \rangle$. La simulazione della UTM si arresterà e accetterà. Di conseguenza, $M_{\overline{L_D}}$ accetterà.
    \item \textbf{Se $\langle M_i \rangle \notin \overline{L_D}$:}
    Per definizione, $M_i$ non accetta $\langle M_i \rangle$. Ciò significa che $M_i$ o rifiuta o va in loop su $\langle M_i \rangle$. In entrambi i casi, $M_{\overline{L_D}}$ non si arresterà in uno stato accettante.
\end{itemize}
Poiché $M_{\overline{L_D}}$ accetta esattamente le stringhe che appartengono a $\overline{L_D}$, si conclude che $\overline{L_D} \in RE$.
\end{proof}

\begin{proposition}
Il linguaggio $\overline{L_D}$ non è ricorsivo (ovvero, $\overline{L_D} \notin R$).
\end{proposition}
\begin{proof}
Se per assurdo $\overline{L_D}$ fosse in $R$, allora per la proprietà che $R$ è chiusa rispetto al complemento, anche il suo complemento $L_D = \overline{\overline{L_D}}$ sarebbe in $R$. Ma sappiamo che $L_D \notin RE$, e quindi $L_D \notin R$. Questo è una contraddizione.
\end{proof}

\section{Problema dell'Arresto (HALT)}
Introduciamo il famoso Problema dell'Arresto.
\begin{definition}[Problema dell'Arresto (HALT)]
Il linguaggio $HALT$ è l'insieme delle coppie $\langle M, w \rangle$, dove $M$ è una Macchina di Turing e $w$ è una stringa, tali che $M$ si arresta su $w$ (indipendentemente dal fatto che accetti o rifiuti).
\[ HALT = \{ \langle M, w \rangle \mid M \text{ si arresta su } w \} \]
\end{definition}
La differenza con $LU$ è sottile ma cruciale: $LU$ richiede l'accettazione, $HALT$ richiede solo l'arresto.

\begin{theorem}
Il linguaggio dell'arresto $HALT$ è ricorsivamente enumerabile (ovvero, $HALT \in RE$).
\end{theorem}
\begin{proof}
Per dimostrare che $HALT \in RE$, dobbiamo costruire una Macchina di Turing $M_{HALT}$ che accetti $HALT$.

\textbf{Costruzione di $M_{HALT}$:}
\begin{enumerate}
    \item $M_{HALT}$ prende in input la coppia $\langle M, w \rangle$.
    \item $M_{HALT}$ simula $M$ su $w$ usando una UTM.
    \item Se la simulazione di $M$ su $w$ si arresta (sia in uno stato accettante che non accettante), allora $M_{HALT}$ accetta (risponde "sì").
    \item Se la simulazione di $M$ su $w$ entra in loop infinito, allora $M_{HALT}$ entra in loop (non risponde "sì").
\end{enumerate}
\textbf{Analisi del comportamento di $M_{HALT}$:}
\begin{itemize}
    \item \textbf{Se $\langle M, w \rangle \in HALT$:}
    Per definizione, $M$ si arresta su $w$. La simulazione della UTM si arresterà. Di conseguenza, $M_{HALT}$ accetterà.
    \item \textbf{Se $\langle M, w \rangle \notin HALT$:}
    Per definizione, $M$ non si arresta su $w$ (va in loop). La simulazione della UTM andrà in loop. Di conseguenza, $M_{HALT}$ non accetterà.
\end{itemize}
Poiché $M_{HALT}$ accetta esattamente le istanze di $HALT$, si conclude che $HALT \in RE$.
\end{proof}

\begin{theorem}
Il linguaggio dell'arresto $HALT$ non è ricorsivo (ovvero, $HALT \notin R$).
\end{theorem}
\begin{proof}
La dimostrazione procede per assurdo, utilizzando una riduzione da $LU$ (di cui sappiamo la non appartenenza a $R$).

\textbf{Assunzione per assurdo:} Supponiamo che $HALT \in R$.
\begin{enumerate}
    \item Se $HALT \in R$, allora esiste una Macchina di Turing $M_{HALT}^*$ che \textbf{decide} $HALT$. Ciò significa che $M_{HALT}^*$ si arresta sempre e dà una risposta corretta (sì/no).
    \item Costruiamo una nuova Macchina di Turing $M'$ che prende in input una coppia $\langle M, w \rangle$ e intende decidere $LU$. Il suo comportamento è il seguente:
    \begin{enumerate}
        \item Riceve $\langle M, w \rangle$ come input.
        \item Dà in input $\langle M, w \rangle$ alla macchina $M_{HALT}^*$ (la cui esistenza è garantita dalla nostra assunzione).
        \item $M'$ verifica la risposta di $M_{HALT}^*$:
        \begin{itemize}
            \item \textbf{Se $M_{HALT}^*$ risponde "no"}: (Significa che $M$ non si arresta su $w$). Allora $M'$ risponde "no". (In questo caso, $M$ non può accettare $w$, quindi $\langle M, w \rangle \notin LU$).
            \item \textbf{Se $M_{HALT}^*$ risponde "sì"}: (Significa che $M$ si arresta su $w$). A questo punto, sapendo che la computazione terminerà, $M'$ simula $M$ su $w$ usando una UTM.
            \begin{itemize}
                \item Se la simulazione di $M$ su $w$ si arresta e accetta, allora $M'$ risponde "sì".
                \item Se la simulazione di $M$ su $w$ si arresta e rifiuta, allora $M'$ risponde "no".
            \end{itemize}
        \end{itemize}
    \end{enumerate}
\end{enumerate}
Analizziamo il linguaggio deciso da $M'$, $L(M')$:
$M'$ risponde "sì" se e solo se $M_{HALT}^*$ risponde "sì" (cioè $M$ si arresta su $w$) \textbf{e} la successiva simulazione di $M$ su $w$ accetta. Questo è vero se e solo se $M$ accetta $w$, ovvero $\langle M, w \rangle \in LU$. In tutti gli altri casi, $M'$ risponde "no".

Il comportamento di $M'$ è esattamente quello di una macchina che decide $LU$. Poiché $M'$ è costruita usando $M_{HALT}^*$ (che è un decider) e una simulazione che è garantita terminare, $M'$ è anch'essa un decider. Questo implicherebbe che $LU \in R$.

\textbf{Contraddizione:} Sappiamo che $LU \notin R$.
L'assunzione iniziale ($HALT \in R$) deve essere falsa.

\textbf{Conclusione:} $HALT \notin R$.
\end{proof}

\begin{proposition}
Il complemento di $HALT$, $\overline{HALT}$, non appartiene a $RE$.
\end{proposition}
\begin{proof}
Sappiamo che $HALT \in RE$ ma $HALT \notin R$. Se per assurdo il suo complemento $\overline{HALT}$ fosse in $RE$, allora per il teorema sulla decidibilità, dato che sia un linguaggio che il suo complemento sono ricorsivamente enumerabili, il linguaggio stesso ($HALT$) sarebbe ricorsivo ($R$). Ma abbiamo appena dimostrato che $HALT \notin R$. Questo è una contraddizione.
\end{proof}

\section{Problema dell'Arresto su Stringa Vuota (HALT$_\epsilon$)}
Questa è una variante specifica del problema dell'arresto.
\begin{definition}[Problema dell'Arresto su Stringa Vuota (HALT$_\epsilon$)]
Il linguaggio $HALT_\epsilon$ è l'insieme delle codifiche di Macchine di Turing $M$ tali che $M$ si arresta quando le viene data in input la stringa vuota $\epsilon$.
\[ HALT_\epsilon = \{ \langle M \rangle \mid M \text{ si arresta su } \epsilon \} \]
\end{definition}

\begin{theorem}
Il linguaggio dell'arresto su stringa vuota, $HALT_\epsilon$, è ricorsivamente enumerabile (ovvero, $HALT_\epsilon \in RE$).
\end{theorem}
\begin{proof}
Per dimostrare che $HALT_\epsilon \in RE$, dobbiamo costruire una Macchina di Turing $M_{HALT_\epsilon}$ che accetti $HALT_\epsilon$.

\textbf{Costruzione di $M_{HALT_\epsilon}$:}
\begin{enumerate}
    \item $M_{HALT_\epsilon}$ prende in input la codifica $\langle M \rangle$.
    \item $M_{HALT_\epsilon}$ simula $M$ sulla stringa vuota $\epsilon$ usando una UTM.
    \item Se la simulazione di $M$ su $\epsilon$ si arresta (sia accettando che rifiutando), allora $M_{HALT_\epsilon}$ accetta (risponde "sì").
    \item Se la simulazione di $M$ su $\epsilon$ entra in loop infinito, allora $M_{HALT_\epsilon}$ entra in loop (non risponde "sì").
\end{enumerate}
Poiché $M_{HALT_\epsilon}$ accetta esattamente le istanze di $HALT_\epsilon$, si conclude che $HALT_\epsilon \in RE$.
\end{proof}

\begin{theorem}
Il linguaggio dell'arresto su stringa vuota, $HALT_\epsilon$, non è ricorsivo (ovvero, $HALT_\epsilon \notin R$).
\end{theorem}
\begin{proof}
La dimostrazione procede per assurdo, utilizzando una riduzione da $HALT$ (di cui sappiamo la non appartenenza a $R$).

\textbf{Assunzione per assurdo:} Supponiamo che $HALT_\epsilon \in R$.
\begin{enumerate}
    \item Se $HALT_\epsilon \in R$, allora esiste una Macchina di Turing $M_{HALT_\epsilon}^*$ che \textbf{decide} $HALT_\epsilon$.
    \item Costruiamo una nuova Macchina di Turing $M'$ che prende in input una coppia $\langle M, w \rangle$ (un'istanza del problema $HALT$) e intende deciderla.
    \begin{enumerate}
        \item Riceve $\langle M, w \rangle$ come input.
        \item $M'$ costruisce la descrizione di una nuova Macchina di Turing, $M_{w}$.
        \item La macchina $M_w$ è definita come segue:
        \begin{itemize}
            \item Quando $M_w$ viene avviata su un qualsiasi input, per prima cosa ignora l'input.
            \item Scrive la stringa $w$ (ottenuta dalla coppia $\langle M, w \rangle$ iniziale) sul proprio nastro.
            \item Successivamente, simula la Macchina di Turing $M$ sul contenuto attuale del nastro, ovvero su $w$.
            \item $M_w$ si arresta se e solo se la simulazione di $M$ su $w$ si arresta.
        \end{itemize}
        \item L'idea chiave è che $M_w$ (che ignora il suo input) si arresta sulla stringa vuota $\epsilon$ se e solo se $M$ si arresta su $w$.
        \item $M'$ dà in input la codifica $\langle M_w \rangle$ alla macchina $M_{HALT_\epsilon}^*$.
        \item $M'$ adotta la risposta di $M_{HALT_\epsilon}^*$: se risponde "sì", $M'$ risponde "sì"; se risponde "no", $M'$ risponde "no".
    \end{enumerate}
\end{enumerate}
Analizziamo il linguaggio deciso da $M'$, $L(M')$:
$M'$ risponde "sì" all'input $\langle M, w \rangle$ se e solo se $M_{HALT_\epsilon}^*$ risponde "sì" a $\langle M_w \rangle$. Questo accade se e solo se $M_w$ si arresta su $\epsilon$, che a sua volta accade se e solo se $M$ si arresta su $w$.
Dunque, $M'$ decide correttamente se $\langle M, w \rangle \in HALT$.

Il comportamento di $M'$ è esattamente quello di una macchina che decide $HALT$. Poiché $M'$ è costruita usando un decider, è essa stessa un decider. Questo implicherebbe che $HALT \in R$.

\textbf{Contraddizione:} Sappiamo che $HALT \notin R$.
L'assunzione iniziale ($HALT_\epsilon \in R$) deve essere falsa.

\textbf{Conclusione:} $HALT_\epsilon \notin R$.
\end{proof}

\section{Riepilogo e Conclusioni}
Abbiamo esplorato la decidibilità di diversi linguaggi fondamentali per la teoria della computazione, stabilendo le seguenti proprietà:
\begin{itemize}
    \item $L_D$: Non è ricorsivamente enumerabile ($L_D \notin RE$).
    \item $\overline{L_D}$: È ricorsivamente enumerabile ma non ricorsivo ($\overline{L_D} \in RE \setminus R$).
    \item $LU$: È ricorsivamente enumerabile ma non ricorsivo ($LU \in RE \setminus R$).
    \item $HALT$: È ricorsivamente enumerabile ma non ricorsivo ($HALT \in RE \setminus R$).
    \item $HALT_\epsilon$: È ricorsivamente enumerabile ma non ricorsivo ($HALT_\epsilon \in RE \setminus R$).
\end{itemize}
Tutti i problemi di non appartenenza a $R$ sono stati dimostrati mediante riduzione, una tecnica fondamentale in teoria della computazione per classificare la difficoltà dei problemi.

\begin{figure}[h!]
    \centering
    \begin{tikzpicture}[node distance=1.8cm, >=Stealth]
        % RE boundary
        \draw[thick, blue!60] (0,0) circle (3.5cm) node[below right=0.1cm, blue!60] {$RE$};
        
        % R boundary
        \draw[thick, green!60!black] (0,0) circle (1.8cm) node[above right=0.1cm, green!60!black] {$R$};
        
        % Languages inside RE but outside R (Recursively Enumerable but not Recursive)
        \node[red] at (135:2.6cm) {$\overline{L_D}$};
        \node[red] at (45:2.6cm) {$LU$};
        \node[red] at (225:2.6cm) {$HALT_\epsilon$};
        \node[red] at (315:2.6cm) {$HALT$};
        
        % Languages outside RE (and therefore outside R)
        \node[purple] at (0:4.2cm) {$L_D$};
        \node[purple] at (180:4.2cm) {$\overline{HALT}$};
        \node[purple] at (90:4.2cm) {$\overline{LU}$};
        \node[purple] at (270:4.2cm) {$\overline{HALT_\epsilon}$};
    \end{tikzpicture}
    \caption{Mappa delle classi di linguaggi $R$ e $RE$ con la posizione dei problemi discussi.}
    \label{fig:r_re_map_updated}
\end{figure}

\end{document}