\documentclass[a4paper]{article}
\usepackage{amsmath, amssymb, amsfonts}
\usepackage[italian]{babel}
\usepackage[utf8]{inputenc}
\usepackage{graphicx}
\usepackage{hyperref}
\usepackage[a4paper, total={6in, 8in}]{geometry}
\usepackage{minted}
\usepackage{mathpazo}
\usepackage{fancyhdr}
\usepackage{amsthm}
\usepackage{tikz}
\usetikzlibrary{automata,positioning,arrows,shapes.geometric}

% Definizione degli ambienti
\theoremstyle{definition} % Use definition style for normal (non-italic) text
\newtheorem{theorem}{Teorema}
\newtheorem{definition}{Definizione}
\newtheorem{example}{Esempio}
\newtheorem{lemma}{Lemma}
\newtheorem{proposition}{Proposizione}
\newcommand{\blankS}{\ensuremath{\raisebox{-0.15ex}{\scalebox{1.3}[0.7]{$\sqcup$}}\mkern2mu}}

\hypersetup{
    pdftitle={Lezione di Informatica Teorica - Macchine di Turing Non Deterministica e Classi di Computabilità},
    pdfauthor={Appunti da Trascrizione AI}
}

\pagestyle{fancy}
\fancyhf{}
\fancyhead[L]{\textit{Lezione di Informatica Teorica}}
\fancyfoot[C]{\thepage}

\title{Lezione di Informatica Teorica: Macchine di Turing Non Deterministiche e Classi di Computabilità}
\author{Appunti da Trascrizione Automatica}
\date{\today}

\begin{document}
\maketitle
\tableofcontents
\newpage

\section{Introduzione alle Macchine di Turing Non Deterministiche}

Nella lezione precedente abbiamo esplorato le macchine di Turing multinastro, constatando che, sebbene più pratiche da programmare, sono equivalenti in termini di potenza computazionale alle macchine di Turing mononastro. La simulazione di una macchina multinastro su una mononastro ha un costo polinomiale (quadratico), il che significa che non cambia l'ordine di complessità degli algoritmi. Questo ci permette di utilizzare le macchine multinastro per semplicità, sapendo che il risultato in termini di calcolabilità (e classi di complessità polinomiale) rimane invariato.

Oggi introduciamo un nuovo modello di macchina di Turing: la \textbf{Macchina di Turing Non Deterministica (NDTM)}. Questo modello sarà fondamentale per il resto del corso per la sua praticità.

\begin{example}[Una Macchina di Turing Non Deterministica]
Consideriamo la seguente macchina di Turing (parzialmente descritta):
\begin{center}
\begin{tikzpicture}[->,>=stealth',shorten >=1pt,auto,node distance=2.8cm,
                    semithick,state/.style={circle,draw,initial text=}]

  \node[state] (q0) {$q_0$};
  \node[state, right of=q0] (q1) {$q_1$};
  \node[state, right of=q1, accepting] (q2) {$q_2$};

  \path (q0) edge [loop above] node {$0/1, R$} (q0);
  \path (q0) edge [bend left] node {$1/0, R$} (q1);

  % Non-determinismo
  \path (q1) edge [loop above] node {$0/0, L$} (q1);
  \path (q1) edge [bend left] node {$0/0, L$} (q0); % Questo è il punto di non-determinismo

  \path (q1) edge [loop below] node {$1/1, R$} (q1);
  \path (q1) edge node {$\blankS/\blankS, R$} (q2);

  \path (q2) edge [loop above] node {$0/0, R$} (q2); % Aggiunta per completezza
  \path (q2) edge [loop below] node {$1/1, R$} (q2); % Aggiunta per completezza
\end{tikzpicture}
\end{center}
Questa macchina presenta una peculiarità nello stato $q_1$: se legge il simbolo $0$, ha due possibili transizioni:
\begin{enumerate}
    \item Rimanere in $q_1$, scrivendo $0$ e spostandosi a sinistra ($q_1 \xrightarrow{0/0, L} q_1$).
    \item Spostarsi in $q_0$, scrivendo $0$ e spostandosi a sinistra ($q_1 \xrightarrow{0/0, L} q_0$).
\end{enumerate}
In un dato stato e con un dato simbolo letto, la macchina ha più di una scelta per il prossimo passo. Questo comportamento è ciò che definisce il \textbf{non-determinismo}.
\end{example}

Le macchine viste finora (mononastro, multinastro, multitraccia) erano tutte \textbf{deterministich}, ovvero per ogni coppia (stato, simbolo letto) esiste un solo passo successivo possibile.

\section{Definizione Formale di Macchina di Turing Non Deterministica}

Una \textbf{Macchina di Turing Non Deterministica (NDTM)} $N$ è una tupla $N = \langle \Sigma, \Gamma, \blankS, Q, q_0, F, \delta \rangle$, dove:
\begin{itemize}
    \item $\Sigma$: l'alfabeto di input (simboli che possono essere letti sull'input iniziale del nastro).
    \item $\Gamma$: l'alfabeto di nastro (tutti i simboli che la macchina può maneggiare sul nastro, con $\Sigma \subset \Gamma$).
    \item $\blankS$: il simbolo di blank, $\blankS \in \Gamma \setminus \Sigma$.
    \item $Q$: l'insieme finito degli stati della macchina.
    \item $q_0$: lo stato iniziale, $q_0 \in Q$.
    \item $F$: l'insieme degli stati finali (o accettanti), $F \subseteq Q$.
    \item $\delta$: la \textbf{funzione di transizione} (o relazione di transizione). Per una NDTM, $\delta$ mappa una coppia (stato, simbolo letto) a un \textbf{insieme} di possibili prossimi passi:
    \[ \delta: Q \times \Gamma \to \mathcal{P}(Q \times \Gamma \times \{L, R\}) \]
    dove $\mathcal{P}(S)$ denota l'insieme delle parti di $S$ (l'insieme di tutti i possibili sottoinsiemi di $S$). Questo significa che per una data coppia $(q, a)$, $\delta(q, a)$ restituisce un insieme di triple $(q', b, D)$, dove $q'$ è il nuovo stato, $b$ il simbolo da scrivere, e $D$ la direzione di movimento della testina. La cardinalità di questo insieme può essere maggiore di uno.

    \begin{example}
    Per l'esempio precedente, la transizione non deterministica è formalmente descritta come:
    \[ \delta(q_1, 0) = \{ (q_0, 0, L), (q_1, 0, R) \} \]
    \end{example}
    Non è necessario che la funzione di transizione sia non deterministica per tutte le coppie $(q,a)$; basta che lo sia per almeno una coppia.
\end{itemize}

\section{Computazione di una Macchina di Turing Non Deterministica}

\subsection{Configurazioni e Albero di Computazione}
Le \textbf{configurazioni} (o descrizioni istantanee) di una NDTM sono definite in modo analogo a quelle delle DTM: una stringa che cattura il contenuto corrente del nastro, lo stato della macchina e la posizione della testina.
Una configurazione iniziale è $q_0W$, dove $W$ è la stringa di input.
Una configurazione è finale se non ammette configurazioni successive. Una configurazione finale è accettante se lo stato in cui si trova la macchina è uno stato accettante ($q \in F$).

La principale differenza è che una data configurazione può avere più "successori legali". Questo porta a organizzare la sequenza delle configurazioni non più come una lista, ma come un \textbf{albero di computazione (Computation Tree)}.

\begin{definition}[Computation Tree]
Il \textbf{computation tree} di una macchina di Turing non deterministica $M$ su una stringa di input $W$ è un albero i cui nodi sono tutte le possibili configurazioni in cui la macchina $M$ può trovarsi processando $W$.
\begin{itemize}
    \item La radice dell'albero è la configurazione iniziale $q_0W$.
    \item C'è un arco da una configurazione $\alpha$ a una configurazione $\beta$ se $\beta$ è uno dei successori legali di $\alpha$.
\end{itemize}
\end{definition}

\begin{example}[Albero di Computazione di una NDTM]
    \begin{tikzpicture}[
    node distance=2.5cm,
    auto,
    state/.style={circle, draw=black, thick, minimum size=1cm},
    config/.style={ellipse, draw=black, thick, minimum width=2cm, minimum height=0.6cm, inner sep=2pt},
    arrow/.style={->, thick}
]

% Left side - Nondeterministic TM
\node[state] (Q) at (-2,1.5) {$q_0$};
\node[state] (P) at (1,2.5) {$q_1$};
\node[state] (R) at (1,0.5) {$q_2$};
\node[state] (S) at (-2,-1) {$q_3$};
\node[state] (T) at (-3.5,-2.5) {$q_4$};
\node[state] (U) at (-0.5,-2.5) {$q_5$};

% Start arrow pointing to initial state Q
\draw[arrow] (-3.5,1.5) -- (Q) node[midway, above] {\textbf{...}};

% Transitions for TM
\draw[arrow] (Q) -- (P) node[midway, above] {$a \to x, R$};
\draw[arrow] (Q) -- (S) node[midway, left] {$a \to y, L$};
\draw[arrow] (S) -- (T) node[midway, left] {$c \to z, L$};
\draw[arrow] (S) -- (U) node[midway, right] {$c \to w, R$};
\draw[arrow] (P) -- (R) node[midway, right] {$b \to y, R$};

\draw[arrow] (P) to[loop above] (P);
\draw[arrow] (R) to[loop right] (R);

% Title for left side
\node at (-0.5, 4) {\Large \textbf{NDTM}};

% Right side - Computation History
\node[config] (c1) at (5.5,2) {$c$ $c$ $\underline{q_0}$ $a$ $b$ $b$ $c$};
\node[config] (c2) at (4,0) {$c$ $\underline{q_3}$ $c$ $y$ $b$ $b$ $c$};
\node[config] (c3) at (7,0) {$c$ $c$ $x$ $\underline{q_1}$ $b$ $b$ $c$};
\node[config] (c4) at (2.5,-2) {$\underline{q_4}$ $c$ $z$ $y$ $b$ $b$ $c$};
\node[config] (c5) at (5.5,-2) {$c$ $w$ $\underline{q_5}$ $y$ $b$ $b$ $c$};
\node[config] (c6) at (8.5,-2) {$c$ $c$ $x$ $y$ $\underline{q_2}$ $b$ $c$};

% Start arrow pointing to initial configuration c1
\draw[arrow] (5.5,3) -- (c1) node[midway, above=8pt] {\textbf{...}};

% Arrows for computation history
\draw[arrow] (c1) -- (c2);
\draw[arrow] (c1) -- (c3);
\draw[arrow] (c2) -- (c4);
\draw[arrow] (c2) -- (c5);
\draw[arrow] (c3) -- (c6);

% Title for right side
\node at (5.5, 4) {\Large \textbf{Computation Tree}};

\end{tikzpicture}
\end{example}

\begin{example}[Computazione di $N$ su input $01$]
Usiamo l'esempio di NDTM di cui sopra e la stringa di input $W = 01$.
La configurazione iniziale è $q_001$.

\begin{figure}[h!]
    \centering
    \begin{tikzpicture}[
        node distance=2.5cm,
        every node/.style={align=center},
        level/.style={sibling distance=60mm/#1},
        config/.style={circle, draw, minimum size=1.2cm, font=\small},
        accept/.style={circle, draw, double, minimum size=1.2cm, font=\small, fill=green!20},
        reject/.style={circle, draw, minimum size=1.2cm, font=\small, fill=red!20}
    ]
    
    % Root node
    \node[config] (root) {$q_0 01$};
    
    % Level 1
    \node[config, below left of=root] (n1) {$1 q_0 1$};
    
    % Level 2  
    \node[config, below of=n1] (n2) {$10 q_1 \blankS$};
    
    % Level 3 - Non-deterministic branching
    \node[accept, below left of=n2] (accept1) {$10\blankS q_2 \blankS$};
    
    % Add edges
    \draw[->] (root) -- (n1) node[midway, left] {\small $0/1, R$};
    \draw[->] (n1) -- (n2) node[midway, left] {\small $1/0, R$};
    \draw[->] (n2) -- (accept1) node[midway, left] {\small $\blankS/\blankS, R$};
    
    % Add labels
    \node[below=0.3cm of accept1, font=\small] {Accetta};
    
    \end{tikzpicture}
    \caption{Albero di Computazione della NDTM su input $01$ - Caso semplice}
\end{figure}

Per illustrare meglio il non-determinismo, consideriamo un input che attivi effettivamente la scelta multipla. Supponiamo di avere la configurazione $10q_10$ dove la macchina deve scegliere tra due transizioni per $\delta(q_1, 0)$:

\begin{figure}[h!]
    \centering
    \begin{tikzpicture}[
        node distance=2cm,
        every node/.style={align=center},
        level/.style={sibling distance=50mm/#1},
        config/.style={rectangle, draw, minimum width=1.5cm, minimum height=0.8cm, font=\scriptsize},
        accept/.style={rectangle, draw, minimum width=1.5cm, minimum height=0.8cm, font=\scriptsize, fill=green!20},
        reject/.style={rectangle, draw, minimum width=1.5cm, minimum height=0.8cm, font=\scriptsize, fill=red!20}
    ]
    
    % Root - the non-deterministic choice point
    \node[config] (root) {$10q_10$};
    
    % Left branch - choice 1: (q_0, 0, L)
    \node[config, below left of=root, xshift=-1cm] (left1) {$1q_000$};
    \node[config, below of=left1] (left2) {$1q_100$};
    \node[reject, below of=left2] (leftend) {Blocco};
    
    % Right branch - choice 2: (q_1, 0, R)  
    \node[config, below right of=root, xshift=1cm] (right1) {$100q_1\blankS$};
    \node[accept, below of=right1] (rightend) {$100\blankS q_2\blankS$};
    
    % Add edges with labels
    \draw[->] (root) -- (left1) node[midway, above left, font=\small] {$0/0, L \to q_0$};
    \draw[->] (root) -- (right1) node[midway, above right, font=\small] {$0/0, R \to q_1$};
    
    \draw[->] (left1) -- (left2) node[midway, left, font=\small] {$0/1, R$};
    \draw[->] (left2) -- (leftend) node[midway, left, font=\small] {Nessuna transizione};
    
    \draw[->] (right1) -- (rightend) node[midway, right, font=\small] {$\blankS/\blankS, R$};
    
    % Add branch labels
    \node[left=0.5cm of left1, font=\small, text=red] {Ramo 1: Rifiuta};
    \node[right=0.5cm of right1, font=\small, text=green!70!black] {Ramo 2: Accetta};
    
    \end{tikzpicture}
    \caption{Albero di Computazione con Non-determinismo - La macchina accetta perché esiste almeno un ramo accettante}
\end{figure}
Analizziamo il processo passo-passo:
\begin{enumerate}
    \item Iniziamo da $q_001$.
    \item Da $q_0$ leggendo $0$: si va a $q_0$, si scrive $1$, si sposta a destra. Nuova configurazione: $1q_01$.
    \item Da $q_0$ leggendo $1$: si va a $q_1$, si scrive $0$, si sposta a destra. Nuova configurazione: $10q_1\blankS$.
    \item Da $q_1$ leggendo $\blankS$: c'è il non-determinismo (in realtà l'esempio della lezione aveva $q_1$ che leggeva $0$, ma nel grafico è $\blankS$, seguiamo il grafico per la computazione e l'esempio della $\delta(q_1,0)$). Il professore ha disegnato la freccia per $(q_1, \blankS)$ in $q_2$. Rivediamo l'esempio del disegno, $q_1$ legge $0$, sposta a destra. C'è un arco da $q_1$ con $1/1,R$ che cicla in $q_1$. Poi da $q_1$ con $\blankS/\blankS,R$ va a $q_2$.
    
    \begin{itemize}
        \item \textbf{Rivediamo l'esempio del professore basandoci sul disegno e l'input 01:}
        \begin{itemize}
            \item Iniziale: $\mathbf{q_0}01$
            \item $q_0,0 \to (q_0,1,R)$: $1\mathbf{q_0}1$
            \item $q_0,1 \to (q_1,0,R)$: $10\mathbf{q_1}\blankS$
            \item Ora siamo in $10q_1\blankS$. La testina è sul blank. Dal diagramma, $q_1$ leggendo blank porta a $q_2$ scrivendo blank e spostandosi a destra.
            \item $q_1,\blankS \to (q_2,\blankS,R)$: $10\blankS\mathbf{q_2}\blankS$ (questa è una configurazione accettante, in quanto $q_2 \in F$).
        \end{itemize}
        
        \item \textbf{Rivediamo invece l'esempio di non-determinismo che il professore ha esplicitamente discusso $\delta(q_1,0)$:}
        \begin{itemize}
            \item Configurazione $q_001$
            \item $q_0,0 \to (q_0,1,R)$: $1q_01$
            \item $q_0,1 \to (q_1,0,R)$: $10q_1\blankS$ (Questo è il punto di divergenza tra il diagramma mostrato e l'input $01$ effettivo, in quanto dopo $10q_1\blankS$ la testina è sul blank, non sul $0$ per attivare il non-determinismo).
            \item \textbf{Per allineare l'esempio alla discussione del professore, dobbiamo immaginare un input diverso che porti a $\mathbf{q_1}0...$}: Il professore usa \texttt{10q10} nell'esempio sull'albero, non \texttt{10q1\_blank\_}. Questo implica che l'input doveva essere più lungo.
            \item \textbf{Seguiamo l'esempio disegnato dal professore, che mostra la configurazione $10q_10$ come punto di non-determinismo:}
            \begin{itemize}
                \item Siamo in $10\mathbf{q_1}0$. Testina su $0$, stato $q_1$.
                \item $\delta(q_1, 0) = \{ (q_0, 0, L), (q_1, 0, R) \}$
                \item \textbf{Primo branch:} $(q_0, 0, L)$. Scriviamo $0$, spostiamo a sinistra, andiamo in $q_0$.\\
                Configurazione successiva: $1\mathbf{q_0}00$.\\
                Da $1\mathbf{q_0}00$: $q_0,0 \to (q_0,1,R)$. Scriviamo $1$, spostiamo a destra. $11\mathbf{q_0}0$.\\
                Da $11\mathbf{q_0}0$: $q_0,0 \to (q_0,1,R)$. Scriviamo $1$, spostiamo a destra. $111\mathbf{q_0}\blankS$.\\
                Da $111\mathbf{q_0}\blankS$: $q_0,\blankS \to (q_2,\blankS,R)$. (Non specificato nel disegno iniziale, assumiamo). Se non c'è transizione, si blocca e rifiuta. Nell'esempio disegnato, questo ramo è bloccato e non accettante (X).
                \item \textbf{Secondo branch:} $(q_1, 0, R)$. Scriviamo $0$, spostiamo a destra, andiamo in $q_1$.\\
                Configurazione successiva: $100\mathbf{q_1}\blankS$.\\
                Da $100\mathbf{q_1}\blankS$: $q_1,\blankS \to (q_2,\blankS,R)$. Scriviamo $\blankS$, spostiamo a destra.\\
                Configurazione successiva: $100\blankS\mathbf{q_2}\blankS$. (Questa è accettante, $q_2 \in F$).
            \end{itemize}
            \item L'esempio nel disegno ha un ramo ulteriore: $10\mathbf{q_1}0 \to 10\mathbf{q_1}0$ (ciclo su $0/0,L$) $\to \dots$. Questo dimostra che possono esserci più rami e che un ramo può anche andare in loop.
        \end{itemize}
    \end{itemize}

\end{enumerate}
\end{example}

\subsection{Condizione di Accettazione per NDTM}

Una macchina di Turing non deterministica $M$ \textbf{accetta} un input $W$ se e solo se all'interno del computation tree di $M$ su $W$ \textbf{esiste almeno una configurazione accettante}.

Ciò significa che la macchina non deve trovare \emph{tutti} i cammini accettanti, ne basta uno. Se la macchina ha un modo per accettare, allora accetta.
Perché una NDTM \textbf{rifiuti} un input, deve succedere che \emph{tutte} le computazioni all'interno del computation tree o terminano in una configurazione non accettante, o non terminano mai (loop).

\section{Simulazione di NDTM tramite DTM}

Una domanda fondamentale è: le Macchine di Turing Non Deterministiche sono più potenti delle Macchine di Turing Deterministiche (DTM) in termini di capacità di calcolo? Ovvero, possono calcolare funzioni o accettare linguaggi che le DTM non possono? La risposta è \textbf{No}.

\begin{theorem}
Per ogni linguaggio $L$ accettato da una macchina di Turing non deterministica $N$, esiste una macchina di Turing deterministica $M$ che accetta $L$.
\end{theorem}
Questo teorema è cruciale, perché significa che, anche se le NDTM non sono fisicamente realizzabili (non "sanno" quale ramo scegliere o non si "sdoppiano"), possiamo usarle come modello di calcolo astratto perché tutto ciò che possono fare può essere fatto anche da una DTM.

\subsection{Strategia di Simulazione (BFS)}
Per dimostrare il teorema, è necessario mostrare come una DTM possa simulare una NDTM. La strategia comune è una \textbf{ricerca in ampiezza (Breadth-First Search - BFS)} sull'albero di computazione della NDTM.

Supponiamo di avere una NDTM $N$ (che possiamo assumere mononastro, poiché sappiamo convertire multinastro in mononastro). Vogliamo costruire una DTM $M$ (multinastro per facilità, poi riconvertibile a mononastro) che simuli $N$.

% \begin{figure}[h!]
%     \centering
%     \includegraphics[width=0.8\textwidth]{pag2_DTM_simulation.jpeg}
%     \caption{Simulazione BFS di NDTM su DTM}
% \end{figure}

La macchina $M$ utilizzerà più nastri. Una configurazione tipica per la simulazione è l'uso di tre nastri:
\begin{enumerate}
    \item \textbf{Nastro 1 (Input/Originale)}: Contiene l'input originale $W$ e non viene modificato.
    \item \textbf{Nastro 2 (Simulazione)}: Contiene la configurazione corrente di $N$ che $M$ sta simulando.
    \item \textbf{Nastro 3 (Coda delle configurazioni)}: Contiene una coda di configurazioni di $N$ che devono ancora essere esplorate, separate da un simbolo speciale (es. \texttt{*}).
\end{enumerate}

\textbf{Algoritmo di Simulazione (BFS):}
\begin{enumerate}
    \item $M$ inizializza il Nastro 3 scrivendo la configurazione iniziale di $N$ su $W$ ($q_0W$).
    \item $M$ entra in un ciclo di esplorazione:
        \begin{enumerate}
            \item Prende la prima configurazione $ID_k$ dal Nastro 3. Se il Nastro 3 è vuoto, $M$ rifiuta (non ha trovato alcun percorso accettante) e si ferma.
            \item Copia $ID_k$ sul Nastro 2.
            \item $M$ simula un singolo passo della NDTM $N$ a partire da $ID_k$ (Nastro 2).
                \begin{itemize}
                    \item Se $ID_k$ è una configurazione finale accettante (lo stato è in $F$), $M$ accetta $W$ e si ferma.
                    \item Se $ID_k$ è una configurazione finale non accettante (si blocca o lo stato non è in $F$), $M$ scarta questo ramo e continua con il passo (a).
                    \item Se $ID_k$ ammette $k'$ successori legali (data la $\delta$ di $N$, $M$ sa quanti e quali sono), $M$ genera questi $k'$ nuovi $ID$ e li aggiunge in coda al Nastro 3, separati da asterischi o altri delimitatori.
                \end{itemize}
        \end{enumerate}
\end{enumerate}

Questa strategia garantisce che $M$ accetti $W$ se e solo se $N$ accetta $W$. La BFS è cruciale perché impedisce a $M$ di bloccarsi in un ramo infinito non accettante, garantendo che se un cammino accettante esiste, $M$ lo troverà prima o poi (a meno che l'albero intero non sia infinito e senza cammini accettanti).

\subsection{Costo della Simulazione}

Il costo di questa simulazione è \textbf{esponenziale}. Se la NDTM $N$ compie $K$ passi per accettare (ovvero il cammino accettante più breve ha lunghezza $K$), e il fattore di branching massimo della NDTM è $C$ (cioè, ogni configurazione può avere al massimo $C$ successori), allora:
\begin{itemize}
    \item Al livello 0: 1 configurazione.
    \item Al livello 1: $C$ configurazioni.
    \item Al livello 2: $C^2$ configurazioni.
    \item Al livello $K$: $C^K$ configurazioni.
\end{itemize}
La DTM $M$ deve esplorare tutti i nodi fino al livello $K$ per trovare il cammino accettante (nella BFS), o tutti i nodi fino a un certo punto per esaurire le opzioni. Il numero di configurazioni da esplorare cresce esponenzialmente con la lunghezza del cammino più breve ($K$). Quindi, il tempo di esecuzione di $M$ è $O(C^K)$.
Questo significa che c'è un \textbf{gap esponenziale} tra la velocità di una NDTM e quella di una DTM che la simula. Se un problema può essere risolto da una NDTM in tempo polinomiale, la DTM che lo simula potrebbe richiedere tempo esponenziale. Questo è il cuore del famoso problema \textbf{P vs NP}.

Non si sa se esista una simulazione più efficiente (es. polinomiale) delle NDTM su DTM. Nessuno è riuscito a trovarla, né a dimostrare che non esista.

\section{La Tesi di Church-Turing}

Abbiamo esaminato vari modelli di calcolo: DTM mononastro, DTM multitraccia, DTM multinastro, NDTM. Tutti questi modelli, pur variando in efficienza, hanno la stessa potenza computazionale: possono calcolare lo stesso insieme di funzioni e accettare lo stesso insieme di linguaggi.

Negli anni '30, quando Alan Turing e altri svilupparono i loro modelli di calcolo (es. $\lambda$-calcolo di Alonzo Church, sistemi di Post), si scoprì che tutti i modelli conosciuti erano computazionalmente equivalenti alle Macchine di Turing. Questo portò alla formulazione della \textbf{Tesi di Church-Turing}:

\begin{theorem}[Tesi di Church-Turing]
\emph{Tutto ciò che è calcolabile è calcolabile da una macchina di Turing.}
\end{theorem}
È chiamata "tesi" e non "teorema" perché non è una dimostrazione formale, ma un'affermazione riguardante la natura della computazione. Non si definisce la calcolabilità a priori e poi si dimostra che la TM la raggiunge, ma si propone che la TM catturi il concetto intuitivo di calcolabilità. Ad oggi, nessun modello di calcolo più potente è stato scoperto o inventato, il che rafforza la fiducia in questa tesi.

\section{Classi di Computabilità}

Sulla base della Tesi di Church-Turing, possiamo definire le principali classi di problemi (o linguaggi) in base alla loro calcolabilità da parte di una Macchina di Turing.

% \begin{figure}[h!]
%     \centering
%     \includegraphics[width=0.6\textwidth]{pag2_computability_classes.jpeg}
%     \caption{Classi di Computabilità: R e RE}
% \end{figure}

\subsection{Classe R (Linguaggi Ricorsivi / Problemi Decidibili)}
\begin{definition}[Classe R]
La classe $R$ (linguaggi \textbf{ricorsivi}) contiene tutti i linguaggi $L$ per i quali esiste una Macchina di Turing $M$ che \textbf{decide} $L$.
\end{definition}
Una Macchina di Turing $M$ \textbf{decide} un linguaggio $L$ se per ogni input $W \in \Sigma^*$:
\begin{itemize}
    \item Se $W \in L$, $M$ si arresta in uno stato accettante.
    \item Se $W \notin L$, $M$ si arresta in uno stato non accettante.
\end{itemize}
In altre parole, una macchina che decide un linguaggio \textbf{termina sempre} per ogni input, dando una risposta definitiva (sì o no). I problemi che corrispondono ai linguaggi in $R$ sono chiamati \textbf{problemi decidibili}.

\subsection{Classe RE (Linguaggi Ricorsivamente Enumerabili / Problemi Accettabili)}
\begin{definition}[Classe RE]
La classe $RE$ (linguaggi \textbf{ricorsivamente enumerabili}) contiene tutti i linguaggi $L$ per i quali esiste una Macchina di Turing $M$ che \textbf{accetta} $L$.
\end{definition}
Una Macchina di Turing $M$ \textbf{accetta} un linguaggio $L$ se per ogni input $W \in \Sigma^*$:
\begin{itemize}
    \item Se $W \in L$, $M$ si arresta in uno stato accettante.
    \item Se $W \notin L$, $M$ \emph{potrebbe non arrestarsi mai} o arrestarsi in uno stato non accettante.
\end{itemize}
I problemi che corrispondono ai linguaggi in $RE$ sono chiamati \textbf{problemi accettabili} o \textbf{problemi semi-decidibili}.

\subsection{Relazione tra R e RE}
È chiaro dalla definizione che ogni linguaggio che può essere deciso può anche essere accettato (una macchina che termina sempre e dà una risposta è anche una macchina che accetta). Quindi, $R \subseteq RE$.
In realtà, $R \subset RE$, ovvero esistono linguaggi che sono accettabili ma non decidibili. Questi problemi sono chiamati \textbf{problemi indecidibili}.
A volte, i linguaggi in $RE \setminus R$ (cioè, i linguaggi accettabili ma non decidibili) sono chiamati \textbf{semidecidibili}. Per questi problemi, se la risposta è "sì", l'algoritmo si ferma e lo comunica. Ma se la risposta è "no", l'algoritmo potrebbe non fermarsi mai, lasciandoci nell'incertezza sulla risposta. Questo li rende problematici per l'uso pratico.

Con queste definizioni, abbiamo stabilito un quadro per classificare i problemi in base alla loro calcolabilità da parte delle Macchine di Turing.

\end{document}