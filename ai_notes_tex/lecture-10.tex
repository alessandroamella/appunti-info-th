\documentclass[a4paper]{article}
\usepackage{amsmath, amssymb, amsfonts}
\usepackage[italian]{babel}
\usepackage[utf8]{inputenc}
\usepackage{graphicx}
\usepackage{hyperref}
\usepackage[a4paper, total={6.5in, 9in}]{geometry} % Aumentato lo spazio verticale
\usepackage{minted} % For pseudocode, if needed. Requires --shell-escape
\usepackage{mathpazo} % For a nicer font
\usepackage{fancyhdr}
\usepackage{amsthm}
\usepackage{tikz}
\usetikzlibrary{automata, positioning, arrows.meta, calc}

% Definizione degli ambienti
\theoremstyle{definition} % Use definition style for normal (non-italic) text
\newtheorem{theorem}{Teorema}
\newtheorem{definition}{Definizione}
\newtheorem{example}{Esempio}
\newtheorem{lemma}{Lemma}
\newtheorem{proposition}{Proposizione}
\newcommand{\blankS}{\ensuremath{\raisebox{-0.15ex}{\scalebox{1.3}[0.7]{$\sqcup$}}\mkern2mu}}
\newtheorem{remark}{Osservazione}


\hypersetup{
    pdftitle={TM Non Deterministiche: Esercitazioni},
    pdfauthor={Appunti da Trascrizione AI}
}

\pagestyle{fancy}
\fancyhf{}
\fancyhead[L]{\textit{Lezione di Informatica Teorica}}
\fancyfoot[C]{\thepage}

\title{Macchine di Turing Non Deterministiche: Esercitazioni}
\author{Appunti da Trascrizione Automatica}
\date{\today}

% Comandi personalizzati per simboli TM
\newcommand{\B}{\blankS} % Blank symbol
\newcommand{\Sh}{\texttt{\#}} % Sharp symbol
\newcommand{\X}{\texttt{X}} % Counter symbol
\newcommand{\alphaSym}{\alpha} % Generic symbol from input alphabet (non-#)
\newcommand{\betaSym}{\beta}  % Another generic symbol
\newcommand{\any}{\_} % Wildcard symbol

\begin{document}


% Stili per i diagrammi TikZ
\tikzset{
    ->, % Stile delle frecce
    >=stealth, % Punta della freccia
    node distance=3.5cm, % Distanza tra i nodi
    every state/.style={
        thick,
        fill=blue!10,
        draw=blue!60
    },
    accepting/.style={
        double,
        double distance=2pt
    },
    edge label/.style={
        below,
        align=center,
        font=\scriptsize
    }
}


\maketitle
\tableofcontents
\newpage

\section{Introduzione alle Macchine di Turing Non Deterministiche (NDTM)}

Riprendiamo dalla discussione sulle Macchine di Turing Non Deterministiche (NDTM).
Una caratteristica fondamentale delle NDTM è la loro capacità di \textit{indovinare} o \textit{guessare} porzioni di stringa o decisioni computazionali. Questa non è una capacità intrinseca della macchina di indovinare nel senso umano, né implica una computazione parallela di tutte le possibilità. È una metafora per descrivere il fatto che se esiste almeno una sequenza di scelte che porta all'accettazione, allora la macchina accetterà.

Tuttavia, per garantire che una NDTM accetti solo le stringhe che appartengono al linguaggio desiderato, la fase di \textit{guess} deve essere sempre seguita da una fase di \textit{check} (controllo). La fase di \textit{check} è cruciale per filtrare le \textit{scelte sbagliate} fatte dalla macchina non deterministica, assicurando che solo le computazioni corrette portino a uno stato di accettazione.

Un esempio pratico di questa capacità delle NDTM è la possibilità di \textit{scrivere in anticipo} su un nastro ausiliario delle stringhe che saranno necessarie in un momento successivo della computazione. Questo comportamento è particolarmente utile quando si analizzano le classi di complessità computazionale (es. NP).

\section{Esercizi}

Analizziamo alcuni esercizi per applicare i concetti delle NDTM. Le etichette sulle transizioni sono nel formato:
\texttt{Nastro 1 / Nastro 2 / Nastro 3 / Nastro 4 / Nastro 5}
con la notazione \texttt{leggi; scrivi, mossa}. Il simbolo \texttt{\_} indica "qualsiasi simbolo".

\subsection{Esercizio 1: Sottostringa Palindroma di Lunghezza Variabile}

Sia $L_1$ il linguaggio definito come:
\[
L_1 = \{ X^N \Sh W_1 \Sh W_2 \Sh \dots \Sh W_N \mid N > 0, W_i \in \{a,b,c,d\}^+, \forall i \in [1, N], \exists S_i \subseteq W_i \text{ t.c. } |S_i| = i \text{ e } S_i = S_i^R \}
\]
dove $X^N$ indica \texttt{X} ripetuto $N$ volte.

\begin{remark}
La definizione di $L_1$ descrive stringhe che iniziano con un numero $N$ di simboli \texttt{X}, seguiti da $N$ blocchi $W_i$ separati da \texttt{\#}. La condizione chiave è che per ogni blocco $W_i$ (con $i$ che va da 1 a $N$), deve esistere una sottostringa $S_i$ di lunghezza \textbf{esattamente $i$} che sia palindroma. Questo significa che il primo blocco $W_1$ deve contenere almeno un carattere palindromo (di lunghezza 1), il secondo blocco $W_2$ deve contenere una sottostringa palindroma di lunghezza 2, e così via. La difficoltà computazionale sta nel fatto che la macchina deve "indovinare" quale sottostringa di $W_i$ rappresenta il palindromo $S_i$ di lunghezza $i$, per poi verificare che sia effettivamente un palindromo e una sottostringa di $W_i$.
\end{remark}

\begin{example}
Ecco alcuni esempi di stringhe che appartengono a $L_1$ per chiarire la definizione.
\begin{itemize}
    \item Se $N=1$, la stringa \texttt{X\#daca} è valida. Infatti, $W_1 = \texttt{daca}$ e deve contenere una sottostringa palindroma di lunghezza $i=1$. Qualsiasi carattere è un palindromo di lunghezza 1, quindi la condizione è soddisfatta.
    \item Se $N=3$, una stringa valida potrebbe essere \texttt{XXX\#a\#dabb\#cdabac}. Analizziamola:
    \begin{itemize}
        \item Per $i=1$, $W_1 = \texttt{a}$. La sottostringa palindroma di lunghezza 1 è \texttt{a} stessa.
        \item Per $i=2$, $W_2 = \texttt{dabb}$. Contiene la sottostringa \texttt{bb}, che è un palindromo di lunghezza 2.
        \item Per $i=3$, $W_3 = \texttt{cdabac}$. Contiene la sottostringa \texttt{aba}, che è un palindromo di lunghezza 3.
    \end{itemize}
    \item Una stringa come \texttt{XX\#c\#bda} non appartiene al linguaggio. Per $i=2$, $W_2 = \texttt{bda}$ non contiene alcuna sottostringa palindroma di lunghezza 2 (le sottostringhe sono \texttt{bd} e \texttt{da}, nessuna è palindroma).
\end{itemize}
\end{example}

\subsubsection{Strategia della Macchina di Turing}
La macchina utilizzerà cinque nastri:
\begin{itemize}
    \item \textbf{Nastro 1 (Input):} Contiene la stringa di input.
    \item \textbf{Nastro 2 (Conteggio $N$):} Per memorizzare il numero di blocchi $W_i$ (come una pila di $\X$).
    \item \textbf{Nastro 3 (Conteggio $i$):} Per memorizzare il valore corrente dell'indice $i$ (come $\X^i$).
    \item \textbf{Nastro 4 (\textit{Guess} $S_i$):} Per scrivere la stringa $S_i$ \textit{indovinata}.
    \item \textbf{Nastro 5 (\textit{Guess} $S_i$ per $S_i^R$):} Una copia di $S_i$ per il controllo di palindromia.
\end{itemize}
L'idea chiave, come suggerito dal professore, è \textbf{indovinare} $S_i$ prima, scrivendola sui nastri 4 e 5, e solo dopo \textbf{verificare} che sia una sottostringa palindroma di $W_i$ della lunghezza corretta.

\subsubsection{Diagramma della Macchina di Turing per $L_1$}

\begin{figure}[h!]
\centering
\begin{tikzpicture}[scale=0.7, transform shape, node distance=1.5cm and 2.5cm]
    \node[state, initial] (q0) {$Q_0$};
    \node[state, right=of q0] (q1) {$Q_1$};
    \node[state, below=of q1, yshift=-0.8cm] (q2) {$Q_2$};
    \node[state, right=of q2] (q3) {$Q_3$};
    \node[state, right=of q3] (q4) {$Q_4$};
    \node[state, below=of q4, yshift=-0.8cm] (q5) {$Q_5$};
    \node[state, right=of q5] (q6) {$Q_6$};
    \node[state, below=of q6, yshift=-0.8cm] (q7) {$Q_7$};
    \node[state, accepting, below=of q7, yshift=-0.8cm] (q_acc) {$Q_{acc}$};
    \node[state, left=of q7] (q8) {$Q_8$};
    
    \path (q0) edge[loop above] node[edge label] {$\X;\X,R$ / $\B;\X,R$ \\ / $\any;\any,S$ / $\any;\any,S$ / $\any;\any,S$} (q0);
    \path (q0) edge node[edge label] {$\Sh;\Sh,S$} (q1);
    \path (q1) edge[bend right=20] node[edge label, right] {$\Sh;\Sh,R$ / $\X;\B,L$ / $\B;\X,R$ \\ / $\any;\any,S$ / $\any;\any,S$} (q2);
    
    \path (q2) edge node[edge label] {$\any;\any,S$ / $\any;\any,S$ / $\B;\X,L$ \\ / $\B;\alphaSym,R$ / $\B;\alphaSym,R$} (q3);
    
    \path (q3) edge[loop above] node[edge label] {$\any;\any,S$ / $\any;\any,S$ / $\X;\X,R$ \\ / $\B;\alphaSym,R$ / $\B;\alphaSym,R$} (q3);
    \path (q3) edge node[edge label] {$\any;\any,S$ / $\any;\any,S$ / $\B;\B,L$ \\ / $\B;\B,L$ / $\B;\B,S$} (q4);
    
    \path (q4) edge[loop above] node[edge label] {$\any;\any,S$ / $\any;\any,S$ / $\any;\any,S$ \\ / $\alphaSym;\alphaSym,L$ / $\any;\any,S$} (q4);
    \path (q4) edge node[edge label] {$\any;\any,S$ / $\any;\any,S$ / $\any;\any,S$ \\ / $\B;\B,R$ / $\any;\any,S$} (q5);
    
    \path (q5) edge[loop above] node[edge label] {$\alphaSym;\alphaSym,R$ / $\dots$} (q5);
    \path (q5) edge node[edge label] {$\alphaSym;\alphaSym,R$ / $\any;\any,S$ / $\any;\any,S$ \\ / $\alphaSym;\B,R$ / $\alphaSym;\B,L$} (q6);
    
    \path (q6) edge[loop above] node[edge label] {$\alphaSym;\alphaSym,R$ / $\any;\any,S$ / $\any;\any,S$ \\ / $\alphaSym;\B,R$ / $\alphaSym;\B,L$} (q6);
    \path (q6) edge node[edge label] {$\any;\any,S$ / $\any;\any,S$ / $\any;\any,S$ \\ / $\B;\B,S$ / $\B;\B,S$} (q7);
    
    \path (q7) edge node[edge label] {$\any;\any,S$ / $\X;\X,S$ / $\any;\any,S$ \\ / $\any;\any,S$ / $\any;\any,S$} (q8);
    
    % Trucco del riavvolgimento non deterministico
    \path (q8) edge[loop above] node[edge label] {Riavvolgi nastri \\ non deterministicamente} (q8);
    \path (q8) edge[bend left=35] node[edge label, right] {$\Sh;\Sh,S$ / $\any;\any,S$ / $\B;\B,S$ \\ / $\B;\B,S$ / $\B;\B,S$} (q1);
    
    \path (q7) edge node[edge label] {Fine controllo} (q_acc);
\end{tikzpicture}
\caption{Diagramma della NDTM per il linguaggio $L_1$. La notazione $\{\dots\}$ indica scelte non deterministiche per il riavvolgimento dei nastri.}
\end{figure}

\subsubsection{Descrizione degli Stati e delle Transizioni}
Sia $\Sigma_I = \{\X, \Sh, a,b,c,d\}$ e $\Gamma_T = \Sigma_I \cup \{\B\}$. Usiamo $\alphaSym$ per un simbolo in $\{a,b,c,d\}$.
\begin{itemize}
    \item \textbf{$Q_0 \to Q_1$:} Copia le $\X$ iniziali dal nastro 1 al nastro 2 per contare $N$. La testina del nastro 1 si ferma su $\Sh$.
    \item \textbf{$Q_1 \to Q_2$:} Inizia il ciclo per una $W_i$. Supera $\Sh$ su Nastro 1. Cancella una $\X$ da Nastro 2 (decrementa $N$ rimanente). Scrive la prima $\X$ su Nastro 3 per $i=1$ (o incrementa $i$ per cicli successivi).
    \item \textbf{$Q_2 \to Q_3 \to Q_3$ (\textit{Guess} $S_i$):} Questa è la fase non deterministica chiave. Per ogni $\X$ sul Nastro 3 (che conta la lunghezza target $i$), la macchina scrive un simbolo $\alphaSym$ \textit{indovinato} sui nastri 4 e 5. La scelta di $\alphaSym$ è non deterministica.
    \item \textbf{$Q_3 \to Q_4$ (Fine \textit{Guess}):} Quando il contatore $i$ (Nastro 3) è esaurito, la macchina ha scritto una stringa $S_i$ di lunghezza $i$ sui nastri 4 e 5. Ora riavvolge il Nastro 4 all'inizio di $S_i$ (Nastro 5 è già pronto per essere letto al contrario).
    \item \textbf{$Q_4 \to Q_5$:} Riavvolgimento del nastro 4.
    \item \textbf{$Q_5 \to Q_6$ (\textit{Check} $S_i$):} Inizia la fase di verifica.
        \begin{itemize}
            \item \textbf{Loop su $Q_5$:} Non deterministicamente, salta un numero arbitrario di caratteri su $W_i$ (Nastro 1) per "indovinare" il punto di inizio di $S_i$.
            \item \textbf{Transizione a $Q_6$:} Inizia il confronto. Verifica che il carattere su Nastro 1 corrisponda a quello su Nastro 4 (verifica di sottostringa) e che il carattere su Nastro 4 corrisponda a quello su Nastro 5 (verifica di palindromia, leggendo Nastro 4 in avanti e Nastro 5 all'indietro). I caratteri sui nastri 4 e 5 vengono cancellati.
        \end{itemize}
    \item \textbf{$Q_6 \to Q_7$ (Fine \textit{Check}):} Se il confronto ha successo (i nastri 4 e 5 sono vuoti), la verifica per la corrente $W_i$ è completata.
    \item \textbf{$Q_7$ (Controllo Ciclo/Fine):}
        \begin{itemize}
            \item \textbf{$\to Q_8$ (Prossima $W_i$):} Se ci sono ancora $\X$ sul Nastro 2, c'è un'altra $W_i$ da processare. Va a $Q_8$ per prepararsi al prossimo ciclo.
            \item \textbf{$\to Q_{acc}$ (Accettazione):} Se il Nastro 2 è vuoto (tutte le $N$ stringhe sono state controllate) e il Nastro 1 è alla fine ($\B$), la macchina accetta.
        \end{itemize}
    \item \textbf{$Q_8 \to Q_1$ (Riavvolgimento e Ciclo):} Questo stato riavvolge in modo non deterministico il Nastro 1 fino al prossimo $\Sh$ e il Nastro 3 all'inizio. Una volta posizionate le testine, torna a $Q_1$ per iniziare il ciclo per $W_{i+1}$.
\end{itemize}


\subsection{Esercizio 2: Coppie di $W_i$}

Sia $L_2$ il linguaggio definito come:
\[
L_2 = \{ A\Sh B\Sh W_1 W_1 W_2 W_2 \dots W_N W_N \mid A, B, W_i \in \{0,1\}^+, |A| > |B|, N = |A| - |B|, |W_i| \ge |B| \}
\]

\subsubsection{Strategia della Macchina di Turing}
\begin{itemize}
    \item \textbf{Nastro 1 (Input):} Contiene $A\Sh B\Sh W_1 W_1 \dots$.
    \item \textbf{Nastro 2 (Conteggio $N$):} Per memorizzare $N = |A| - |B|$.
    \item \textbf{Nastro 3 (Copia $B$):} Per memorizzare la stringa $B$ e usarla per il check $|W_i| \ge |B|$.
    \item \textbf{Nastro 4 (Copia $W_i$):} Per memorizzare la prima occorrenza di $W_i$ e confrontarla con la seconda.
\end{itemize}

\subsubsection{Diagramma della Macchina di Turing per $L_2$}
\begin{figure}[h!]
\centering
\begin{tikzpicture}[transform shape, node distance=1.5cm and 2.5cm]
    \node[state, initial] (q0) {$Q_0$};
    \node[state, right=of q0] (q1) {$Q_1$};
    \node[state, right=of q1] (q2) {$Q_2$};
    \node[state, right=of q2] (q3) {$Q_3$};
    \node[state, below=of q3, yshift=-1cm] (q4) {$Q_4$};
    \node[state, left=of q4] (q5) {$Q_5$};
    \node[state, left=of q5] (q6) {$Q_6$};
    \node[state, accepting, below=of q6, yshift=-1cm] (q_acc) {$Q_{acc}$};
    
    \path (q0) edge[loop above] node[edge label] {$\alphaSym;\alphaSym,R$ / $\B;\X,R$} (q0);
    \path (q0) edge node[edge label] {$\Sh;\Sh,R$ / $\B;\B,L$} (q1);
    
    \path (q1) edge[loop above] node[edge label] {$\alphaSym;\alphaSym,R$ / $\X;\B,L$ / $\B;\alphaSym,R$} (q1);
    \path (q1) edge node[edge label] {$\Sh;\Sh,R$ / $\X;\X,S$ / $\B;\B,L$} (q2);
    
    \path (q2) edge node[edge label] {Inizio $W_i$} (q3);
    
    \path (q3) edge[loop above] node[edge label] {$\alphaSym;\alphaSym,R$ / $\any;\any,S$ / $\X;\X,L$ / $\B;\alphaSym,R$} (q3); % Check |W_i| >= |B|
    \path (q3) edge[loop right] node[edge label] {$\alphaSym;\alphaSym,R$ / $\any;\any,S$ / $\B;\B,S$ / $\B;\alphaSym,R$} (q3); % Copia resto di W_i
    
    \path (q3) edge node[edge label] {Fine $W_i$ \\ $\any;\any,S$ / $\X;\B,L$ / $\B;\B,R$ / $\B;\B,L$} (q4);
    
    \path (q4) edge[loop below] node[edge label] {Riavvolgi T3, T4} (q4);
    \path (q4) edge node[edge label] {Pronto} (q5);
    
    \path (q5) edge[loop below] node[edge label] {$\alphaSym;\alphaSym,R$ / $\any;\any,S$ / $\any;\any,S$ / $\alphaSym;\B,R$} (q5);
    \path (q5) edge node[edge label] {Fine $W_iW_i$ \\ $\any;\any,S$ / $\any;\any,S$ / $\any;\any,S$ / $\B;\B,S$} (q6);
    
    \path (q6) edge[bend right=45] node[edge label, above] {$\X$ su T2} (q2);
    \path (q6) edge node[edge label] {$\B$ su T2, $\B$ su T1} (q_acc);
    
\end{tikzpicture}
\caption{Schema concettuale della NDTM per il linguaggio $L_2$. Alcuni stati sono aggregati per chiarezza.}
\end{figure}

\subsubsection{Descrizione degli Stati e delle Transizioni}
\begin{itemize}
    \item \textbf{$Q_0 \to Q_1$ (Copia $|A|$):} Legge $A$ dal Nastro 1 e scrive $|A|$ \texttt{X} sul Nastro 2.
    \item \textbf{$Q_1 \to Q_2$ (Calcolo $N$ e Copia $B$):} Legge $B$. Per ogni simbolo di $B$, cancella una $\X$ da Nastro 2 e copia il simbolo sul Nastro 3. Alla fine, il Nastro 2 conterrà $N=|A|-|B|$ \texttt{X}. Il controllo $|A|>|B|$ è implicito: se il nastro 2 si svuotasse, la macchina si bloccherebbe.
    \item \textbf{$Q_2 \to Q_3 \to \dots$ (Ciclo per $W_i$):}
    \begin{enumerate}
        \item \textbf{Copia $W_i$ e verifica lunghezza:} Inizia a leggere $W_i$. Copia $W_i$ sul Nastro 4. Contemporaneamente, per ogni simbolo copiato, sposta la testina del Nastro 3 (che contiene $B$) a sinistra. Se si raggiunge la fine di $B$ mentre si sta ancora leggendo $W_i$, la condizione $|W_i| \ge |B|$ è soddisfatta. La copia di $W_i$ continua non deterministicamente fino alla fine della prima occorrenza.
        \item \textbf{Prepara confronto:} Alla fine della prima $W_i$, cancella una $\X$ dal Nastro 2, riavvolge i nastri 3 e 4.
        \item \textbf{Verifica $W_i=W_i$:} Confronta la $W_i$ copiata sul Nastro 4 con la seconda occorrenza di $W_i$ sul Nastro 1.
        \item \textbf{Ciclo o Accetta:} Se Nastro 2 ha ancora $\X$, torna all'inizio del ciclo per la prossima coppia. Altrimenti, se Nastro 1 è alla fine, accetta.
    \end{enumerate}
\end{itemize}


\subsection{Esercizio 3: Parità di Lunghezza di $W_i$}

Sia $L_3$ il linguaggio definito come:
\[
L_3 = \{A\Sh B\Sh W_1 \Sh \dots \Sh W_N \mid A, B, W_i \in \{a,b,c,d\}^+, |A| > |B|, N = |A| - |B|, 
\]
$\qquad (\text{se } |W_i| \text{ è pari, allora } B \subseteq W_i) \land (\text{se } |W_i| \text{ è dispari, allora } B^R \subseteq W_i) \}$

\subsubsection{Strategia della Macchina di Turing}
La strategia chiave qui è determinare la parità della lunghezza di $W_i$ e poi biforcare la logica. Come spiegato dal professore, questo si può fare elegantemente usando due stati che si alternano, uno per le lunghezze parziali dispari e uno per quelle pari, senza bisogno di un nastro aggiuntivo.
\begin{itemize}
    \item \textbf{Nastro 1 (Input):} Contiene la stringa di input.
    \item \textbf{Nastro 2 (Conteggio $N$):} Per memorizzare $N = |A| - |B|$.
    \item \textbf{Nastro 3 (Copia $B$):} Memorizza $B$. $B^R$ viene verificato leggendo questo nastro all'indietro.
\end{itemize}

\subsubsection{Diagramma della Macchina di Turing per $L_3$}
\begin{figure}[h!]
\centering
\begin{tikzpicture}[scale=0.7, transform shape, node distance=2.5cm]
    \node[state, initial] (q_start) {$Q_{0-2}$};
    \node[state, right=of q_start] (q3) {$Q_3$};
    \node[state, below right=of q3, yshift=-0.5cm] (q4) {\begin{tabular}{c}$Q_4$ \\ (dispari)\end{tabular}};
    \node[state, above right=of q3, yshift=0.5cm] (q5) {\begin{tabular}{c}$Q_5$ \\ (pari)\end{tabular}};
    \node[state, right=of q4, xshift=1.5cm] (q6) {\begin{tabular}{c}Cerca \\ $B^R$\end{tabular}};
    \node[state, right=of q5, xshift=1.5cm] (q9) {\begin{tabular}{c}Cerca \\ $B$\end{tabular}};
    \node[state, right=of q6, xshift=1.5cm] (q_loop) {$Q_{loop}$};
    \node[state, accepting, below=of q_loop] (q_acc) {$Q_{acc}$};

    \path (q_start) edge node[edge label] {Inizio $W_i$} (q3);
    
    \path (q3) edge node[edge label, sloped] {$\alphaSym;\alphaSym,R$} (q4);
    
    \path (q4) edge[bend left=30] node[edge label] {$\alphaSym;\alphaSym,R$} (q5);
    \path (q5) edge[bend left=30] node[edge label] {$\alphaSym;\alphaSym,R$} (q4);
    
    \path (q4) edge node[edge label] {$\Sh$ o $\B$} (q6);
    \path (q5) edge node[edge label] {$\Sh$ o $\B$} (q9);
    
    \path (q6) edge node[edge label] {Check OK} (q_loop);
    \path (q9) edge node[edge label] {Check OK} (q_loop);
    
    \path (q_loop) edge[bend left=45] node[edge label, left] {Altra $W_i$} (q3);
    \path (q_loop) edge node[edge label] {Fine} (q_acc);
    
\end{tikzpicture}
\caption{Schema concettuale della NDTM per $L_3$. $Q_{0-2}$ rappresenta la fase iniziale di conteggio. L'oscillazione tra $Q_4$ e $Q_5$ determina la parità della lunghezza di $W_i$.}
\end{figure}

\subsubsection{Descrizione degli Stati e delle Transizioni}
\begin{itemize}
    \item \textbf{$Q_{0-2}$ (Fase Iniziale):} Come nell'esercizio precedente, calcola $N=|A|-|B|$ sul Nastro 2 e copia $B$ sul Nastro 3.
    \item \textbf{$Q_3$ (Inizio Controllo Parità per $W_i$):} Inizia il ciclo per una $W_i$. Decrementa $N$ su Nastro 2. Legge il primo carattere di $W_i$ e passa a $Q_4$, lo stato che rappresenta una lunghezza parziale letta dispari (1 carattere).
    \item \textbf{$Q_4 \leftrightarrow Q_5$ (Oscillazione di Parità):}
        \begin{itemize}
            \item Da $Q_4$ (lunghezza dispari), leggere un altro carattere porta a $Q_5$ (lunghezza pari).
            \item Da $Q_5$ (lunghezza pari), leggere un altro carattere porta a $Q_4$ (lunghezza dispari).
            \item Questo continua finché non si incontra $\Sh$ o $\B$, che segna la fine di $W_i$.
        \end{itemize}
    \item \textbf{Biforcazione della Logica:}
        \begin{itemize}
            \item Se la fine di $W_i$ viene raggiunta da $Q_4$, la lunghezza è \textbf{dispari}. La macchina procede a cercare non deterministicamente $B^R$ come sottostringa di $W_i$. Questo viene fatto confrontando il Nastro 1 (letto in avanti) con il Nastro 3 (letto all'indietro).
            \item Se la fine di $W_i$ viene raggiunta da $Q_5$, la lunghezza è \textbf{pari}. La macchina procede a cercare non deterministicamente $B$ come sottostringa di $W_i$ (confrontando Nastro 1 e Nastro 3, entrambi letti in avanti).
        \end{itemize}
    \item \textbf{$Q_{loop}$ (Controllo Ciclo/Fine):} Dopo una verifica riuscita, la macchina si sposta in uno stato di controllo. Se ci sono ancora $\X$ su Nastro 2, torna a $Q_3$ per la prossima $W_i$. Altrimenti, se Nastro 1 è alla fine, accetta.
\end{itemize}

\subsection{Esercizio 4: $A$ o $B^R$ in $W_i$ in base alla Parità dell'Indice $i$}

Sia $L_4$ il linguaggio definito come:
\[
L_4 = \{A\Sh B\Sh W_1 \Sh \dots \Sh W_N \mid A, B, W_i \in \{a,b,c,d\}^+, |A|>|B|>0, N=|A|+|B|, 
\]
$\qquad (\text{se } i \text{ è dispari, allora } A \subseteq W_i) \land (\text{se } i \text{ è pari, allora } B^R \subseteq W_i) \}$

\subsubsection{Strategia della Macchina di Turing}
La logica qui è un ciclo alternato: un ramo della macchina gestisce gli indici dispari (verificando $A$) e un altro ramo gestisce gli indici pari (verificando $B^R$).
\begin{itemize}
    \item \textbf{Nastro 1 (Input):} Stringa di input.
    \item \textbf{Nastro 2 (Conteggio $N$):} Memorizza $N = |A| + |B|$ come una pila di $\X$.
    \item \textbf{Nastro 3 (Copia $A$):} Memorizza la stringa $A$.
    \item \textbf{Nastro 4 (Copia $B$):} Memorizza la stringa $B$.
\end{itemize}

\subsubsection{Diagramma della Macchina di Turing per $L_4$}
\begin{figure}[h!]
\centering
\begin{tikzpicture}
    \node[state, initial] (q_start) {$Q_{0-2}$};
    \node[state, right=of q_start, xshift=1cm] (q_dispari) {$Q_{dispari}$};
    \node[state, below=of q_dispari, yshift=-1cm] (q_pari) {$Q_{pari}$};
    \node[state, accepting, right=of q_dispari, xshift=4cm] (q_acc) {$Q_{acc}$};

    \path (q_start) edge node[edge label] {Inizio $W_1$} (q_dispari);
    
    \path (q_dispari) edge[bend left] node[edge label, left] {Check $A \subseteq W_i$ OK \\ Prossimo $\Sh$} (q_pari);
    \path (q_pari) edge[bend left] node[edge label, right] {Check $B^R \subseteq W_{i+1}$ OK \\ Prossimo $\Sh$} (q_dispari);
    
    % Acceptance paths
    \path (q_dispari) edge[dashed] node[edge label, above, sloped] {Fine input} (q_acc);
    \path (q_pari) edge[dashed] node[edge label, below, sloped] {Fine input} (q_acc);

\end{tikzpicture}
\caption{Schema concettuale della NDTM per $L_4$. $Q_{0-2}$ rappresenta la fase iniziale. La macchina alterna tra uno stato per indici dispari ($Q_{dispari}$) e uno per indici pari ($Q_{pari}$).}
\end{figure}

\subsubsection{Descrizione degli Stati e delle Transizioni}
\begin{itemize}
    \item \textbf{$Q_{0-2}$ (Fase Iniziale):}
        \begin{enumerate}
            \item Copia $A$ sul Nastro 3 e $|A|$ \texttt{X} sul Nastro 2.
            \item Legge $B$, copia $B$ sul Nastro 4 e aggiunge $|B|$ \texttt{X} sul Nastro 2. Ora Nastro 2 contiene $N=|A|+|B|$ \texttt{X}.
            \item Il controllo $|A|>|B|>0$ viene fatto durante questo processo (es. verificando che la parte di $B$ sia non vuota e che la parte di $A$ sul nastro 3 sia più lunga di $B$ sul nastro 4).
            \item Riavvolge i nastri 2, 3, 4 e si prepara per il ciclo principale.
        \end{enumerate}
    \item \textbf{$Q_{dispari}$ (Processa $W_i$ con $i$ dispari):}
        \begin{itemize}
            \item Decrementa il contatore $N$ sul Nastro 2.
            \item Cerca non deterministicamente $A$ (da Nastro 3) come sottostringa di $W_i$ (su Nastro 1).
            \item Se la trova, avanza fino al prossimo $\Sh$ e passa a $Q_{pari}$.
            \item Se il contatore $N$ è esaurito e l'input è terminato, accetta.
        \end{itemize}
    \item \textbf{$Q_{pari}$ (Processa $W_i$ con $i$ pari):}
        \begin{itemize}
            \item Decrementa il contatore $N$ sul Nastro 2.
            \item Cerca non deterministicamente $B^R$ (leggendo Nastro 4 all'indietro) come sottostringa di $W_i$ (su Nastro 1).
            \item Se la trova, avanza fino al prossimo $\Sh$ e torna a $Q_{dispari}$.
            \item Se il contatore $N$ è esaurito e l'input è terminato, accetta.
        \end{itemize}
\end{itemize}
L'alternanza tra i due stati garantisce che la condizione venga verificata correttamente per indici pari e dispari.

\section{Conclusioni}
Le esercitazioni di oggi hanno permesso di approfondire la progettazione di Macchine di Turing Non Deterministiche, in particolare mostrando come la capacità di \textit{guess} (indovinare) combinata con un robusto \textit{check} (controllo) possa semplificare la logica di alcune verifiche complesse. È stato evidenziato come le NDTM possano \textit{scrivere in anticipo} su nastri ausiliari stringhe che verranno poi validate.
Nella prossima lezione, si inizierà a esplorare i concetti di calcolabilità, inclusi problemi indecidibili e il concetto fondamentale di riduzione, che sarà un pilastro per comprendere la complessità dei problemi.

\end{document}