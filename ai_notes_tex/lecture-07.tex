\documentclass[a4paper]{article}
\usepackage{amsmath, amssymb, amsfonts}
\usepackage[italian]{babel}
\usepackage[utf8]{inputenc}
\usepackage{graphicx}
\usepackage{hyperref}
\usepackage[scaled]{helvet}
\renewcommand{\familydefault}{\sfdefault}
\usepackage[T1]{fontenc}
\usepackage[a4paper, total={6in, 8in}]{geometry}
\usepackage{minted}
\usepackage{mathpazo}
\usepackage{fancyhdr}
\usepackage{amsthm}
\usepackage{tikz}
\usetikzlibrary{
    automata,
    positioning,
    arrows,
    arrows.meta,
    calc,
    fit,
    patterns,
    graphs,
    graphdrawing,
    matrix,
    quotes,
    shapes.geometric,
    snakes,
    shapes,
    decorations.pathreplacing,
    calligraphy,
    chains,
}

% Definizione degli ambienti
\theoremstyle{definition} % Use definition style for normal (non-italic) text
\newtheorem{theorem}{Teorema}
\newtheorem{definition}{Definizione}
\newtheorem{example}{Esempio}
\newtheorem{lemma}{Lemma}
\newtheorem{proposition}{Proposizione}
\newcommand{\blankS}{\ensuremath{\raisebox{-0.15ex}{\scalebox{1.3}[0.7]{$\sqcup$}}\mkern2mu}}

\hypersetup{
    pdftitle={Lezione di Informatica Teorica - Macchine di Turing Multinastro},
    pdfauthor={Appunti da Trascrizione Automatica}
}

\pagestyle{fancy}
\fancyhf{}
\fancyhead[L]{\textit{Lezione di Informatica Teorica}}
\fancyfoot[C]{\thepage}

\title{Lezione di Informatica Teorica\\{\Large Macchine di Turing Multinastro}}
\author{Appunti da Trascrizione Automatica}
\date{\today}

\begin{document}
\maketitle
\tableofcontents
\newpage

\section{Macchine di Turing Multinastro}

\subsection{Definizione Formale}

\begin{definition}[Macchina di Turing Multinastro]
Una macchina di Turing multinastro è definita come una tupla $(Q, \Sigma, \Gamma, \delta, q_0, q_{acc}, q_{rej})$, dove:
\begin{itemize}
    \item $Q$ è un insieme finito di stati.
    \item $\Sigma$ è l'alfabeto di input (non contiene il simbolo di blank $\blankS$).
    \item $\Gamma$ è l'alfabeto del nastro (contiene $\Sigma$ e $\blankS$).
    \item $\delta: Q \times \Gamma^k \to Q \times (\Gamma \times \{L, R, S\})^k$ è la funzione di transizione, dove $k$ è il numero di nastri. Per ogni stato e per ogni tupla di simboli letti dalle $k$ testine, la funzione specifica il prossimo stato, i simboli da scrivere su ogni nastro e le direzioni in cui muovere ogni testina (Left, Right, Stay).
    \item $q_0 \in Q$ è lo stato iniziale.
    \item $q_{acc} \in Q$ è lo stato di accettazione.
    \item $q_{rej} \in Q$ è lo stato di rifiuto.
\end{itemize}
\end{definition}

\subsection{Convenzioni di Notazione nelle Transizioni}
Nei diagrammi di stato (e nella loro descrizione testuale), le transizioni sono etichettate con una notazione compatta per rappresentare le operazioni su più nastri. Una transizione da uno stato $q_i$ a $q_j$ etichettata con, ad esempio:
\[ \text{nastro 1: } (\text{leggi}_1/\text{scrivi}_1, \text{muovi}_1), \quad \text{nastro 2: } (\text{leggi}_2/\text{scrivi}_2, \text{muovi}_2) \]
significa che per eseguire la transizione, la TM deve leggere $\text{leggi}_1$ dal nastro 1 e $\text{leggi}_2$ dal nastro 2. Se le condizioni di lettura sono soddisfatte, la TM scrive $\text{scrivi}_1$ su nastro 1 e $\text{scrivi}_2$ su nastro 2, e muove le testine come indicato.

\begin{itemize}
    \item \textbf{Simboli Variabili:} Spesso si usano simboli come $\alpha, \beta, \gamma$ per indicare caratteri generici dall'alfabeto del nastro ($\Gamma$).
    \begin{itemize}
        \item Se lo \textbf{stesso simbolo variabile} (es. $\alpha$) compare nella lettura di \textbf{più nastri} nella stessa etichetta di transizione, significa che su quei nastri deve essere letto lo \textbf{stesso identico carattere}. Es. \texttt{nastro 1: ($\alpha$/$\alpha$, R), nastro 2: ($\alpha$/$\alpha$, R)} implica che se si legge $0$ su nastro 1, si deve leggere $0$ anche su nastro 2. Se si legge $1$ su nastro 1, si deve leggere $1$ anche su nastro 2. Se si legge $0$ su nastro 1 e $1$ su nastro 2, questa transizione non è valida.
        \item Se \textbf{simboli variabili diversi} (es. $\alpha, \beta$) compaiono nella lettura di \textbf{più nastri} nella stessa etichetta di transizione, significa che i caratteri letti possono essere \textbf{diversi o uguali}. Es. \texttt{nastro 1: ($\alpha$/$\alpha$, R), nastro 2: ($\beta$/$\beta$, R)} implica che $\alpha$ può essere $0$ e $\beta$ $1$ (o viceversa, o uguali). Non viene imposto alcun confronto.
    \end{itemize}
    \item \textbf{Omissione di operazioni:} Se per un nastro non viene specificata alcuna operazione di scrittura o movimento, si intende che la testina rimane ferma (\texttt{S}) e il simbolo non viene modificato.
    \item \textbf{Simbolo di blank:} Il simbolo $\blankS$ indica una cella vuota.
\end{itemize}

\section{Esercizi di Riconoscimento di Linguaggi}
Illustriamo il funzionamento delle TM multinastro attraverso esempi pratici di riconoscimento di linguaggi.

\subsection{Linguaggio $L_1$}
Il linguaggio $L_1 = \{A\#B\#AB \mid A,B \in \{0,1\}^+\}$ è l'insieme di tutte le stringhe composte da tre parti $A$, $B$ e $AB$ (concatenazione di $A$ e $B$), separate da simboli \texttt{\#}. Le stringhe $A$ e $B$ sono composte da caratteri $0$ o $1$ e devono avere lunghezza di almeno 1.
\begin{example}
Esempi di stringhe in $L_1$: $0\texttt{\#}1\texttt{\#}01$, $01\texttt{\#}0\texttt{\#}010$.
Esempi di stringhe non in $L_1$: $0\texttt{\#}1\texttt{\#}10$ (AB sbagliato), $0\texttt{\#}\texttt{\#}0$ (B vuoto), $\texttt{\#}1\texttt{\#}1$ (A vuoto).
\end{example}

\subsubsection{Strategia}
Utilizziamo una TM con 2 nastri.
\begin{itemize}
    \item \textbf{Nastro 1 (Input):} Contiene la stringa di input $A\#B\#AB$.
    \item \textbf{Nastro 2 (Lavoro):} Utilizzato per copiare la stringa $A$ e poi $B$ sequenzialmente, ottenendo $AB$. Successivamente, il contenuto di Nastro 2 (che è $AB$) verrà confrontato con l'ultima parte della stringa sul Nastro 1 (che dovrebbe essere $AB$).
\end{itemize}

\begin{center}
\begin{tikzpicture}[
    tape/.style={draw, rectangle, minimum size=9mm},
    state/.style={draw, circle, thick, font=\Large},
    head/.style={-Stealth, thick, red},
]
% Stato attuale della Macchina
\node[state] (q) at (2.5, -1.5) {$q_3$};

% --- Nastro 1 (Input) ---
\node (label1) at (-2.5,0) {\textbf{Nastro 1:}};
\begin{scope}[start chain=t1 going right, node distance=0pt]
    \node[tape, on chain] at (0,0) {0};
    \node[tape, on chain] {1};
    \node[tape, on chain] {\#};
    \node[tape, on chain] {1};
    \node[tape, on chain] {0};
    \node[tape, on chain] {\#};
    \node[tape, on chain, name=head1] {0}; % Testina qui
    \node[tape, on chain] {1};
    \node[tape, on chain] {1};
    \node[tape, on chain] {0};
    \node[tape, on chain] {\blankS};
\end{scope}
\draw[head] (q) to [out=90, in=-90] (head1.south);

% --- Nastro 2 (Lavoro) ---
\node (label2) at (-2.5,-1.5) {\textbf{Nastro 2:}};
\begin{scope}[yshift=-1.5cm, start chain=t2 going right, node distance=0pt]
    \node[tape, on chain, name=head2] at (0,-1.5) {0}; % Testina qui
    \node[tape, on chain] {1};
    \node[tape, on chain] {1};
    \node[tape, on chain] {0};
    \node[tape, on chain] {\blankS};
    \node[tape, on chain] {\blankS};
    \node[tape, on chain] {\blankS};
    \node[tape, on chain] {\blankS};
    \node[tape, on chain] {\blankS};
    \node[tape, on chain] {\blankS};
    \node[tape, on chain] {\blankS};
\end{scope}
\draw[head] (q) to [out=-90, in=90] (head2.north);

\end{tikzpicture}
\end{center}

\subsubsection{Implementazione (Stati e Transizioni)}
\begin{description}
    \item[$q_0$ (Copia A):] Legge i caratteri di $A$ dal Nastro 1 e li copia sul Nastro 2.
    \begin{itemize}
        \item \textbf{Condizione:} Leggi $\alpha \in \{0,1\}$ su Nastro 1, $\blankS$ su Nastro 2.
        \item \textbf{Azione:} Scrivi $\alpha$ su Nastro 2. Sposta entrambe le testine a destra. Rimani in $q_0$.
        \item \textbf{Uscita:} Quando Nastro 1 legge \texttt{\#}.
        \item \textbf{Transizione:} Da $q_0$ a $q_1$: Nastro 1: $(\texttt{\#}/\texttt{\#}, R)$, Nastro 2: $(\blankS/\blankS, S)$.
    \end{itemize}
    \item[$q_1$ (Copia B):] Legge i caratteri di $B$ dal Nastro 1 e li copia sul Nastro 2 (dopo la copia di $A$).
    \begin{itemize}
        \item \textbf{Condizione:} Leggi $\alpha \in \{0,1\}$ su Nastro 1, $\blankS$ su Nastro 2.
        \item \textbf{Azione:} Scrivi $\alpha$ su Nastro 2. Sposta entrambe le testine a destra. Rimani in $q_1$.
        \item \textbf{Uscita:} Quando Nastro 1 legge \texttt{\#}.
        \item \textbf{Transizione:} Da $q_1$ a $q_2$: Nastro 1: $(\texttt{\#}/\texttt{\#}, R)$, Nastro 2: $(\blankS/\blankS, S)$.
    \end{itemize}
    \item[$q_2$ (Riavvolgi Nastro 2):] Riporta la testina del Nastro 2 all'inizio della stringa $AB$ (cioè all'inizio del Nastro 2).
    \begin{itemize}
        \item \textbf{Condizione:} Leggi $\alpha \in \{0,1\}$ su Nastro 2.
        \item \textbf{Azione:} Scrivi $\alpha$ su Nastro 2. Sposta testina Nastro 2 a sinistra. Rimani in $q_2$.
        \item \textbf{Uscita:} Quando Nastro 2 legge $\blankS$.
        \item \textbf{Transizione:} Da $q_2$ a $q_3$: Nastro 2: $(\blankS/\blankS, R)$. (Nastro 1 rimane fermo).
    \end{itemize}
    \item[$q_3$ (Confronto AB):] Confronta la stringa $AB$ sul Nastro 2 con l'ultima parte della stringa sul Nastro 1.
    \begin{itemize}
        \item \textbf{Condizione:} Leggi $\alpha \in \{0,1\}$ su Nastro 1, $\alpha$ su Nastro 2. (Stesso $\alpha$ indica che devono essere uguali).
        \item \textbf{Azione:} Scrivi $\alpha$ su Nastro 1 e $\alpha$ su Nastro 2. Sposta entrambe le testine a destra. Rimani in $q_3$.
        \item \textbf{Uscita:} Quando entrambe le testine leggono $\blankS$.
        \item \textbf{Transizione:} Da $q_3$ a $q_{acc}$: Nastro 1: $(\blankS/\blankS, S)$, Nastro 2: $(\blankS/\blankS, S)$.
    \end{itemize}
    \item[Rifiuto:] Se in qualsiasi stato la TM tenta una transizione non definita (es. simboli non corrispondenti durante il confronto, \texttt{\#} al posto di un carattere, stringa non conforme al formato $A\texttt{\#}B\texttt{\#}AB$), la macchina si blocca e rifiuta.
\end{description}

\subsection{Linguaggio $L_2$}
Il linguaggio $L_2 = \{A\#B\#C \mid A,B,C \in \{0,1\}^+, |A|>|B|>|C|, |C|=|A|-|B|\}$ consiste in stringhe $A\#B\#C$ dove $A,B,C$ sono stringhe non vuote di $0$ o $1$. Devono valere le condizioni sulle lunghezze:
\begin{itemize}
    \item $|A| > |B| > |C|$ (lunghezze strettamente decrescenti)
    \item $|C| = |A| - |B|$
\end{itemize}

\subsubsection{Strategia}
Utilizziamo 3 nastri.
\begin{itemize}
    \item \textbf{Nastro 1 (Input):} Contiene la stringa di input.
    \item \textbf{Nastro 2 (Lavoro):} Usato per marcare la lunghezza di $A$ (con 'X') e poi sottrarre la lunghezza di $B$. Al termine, conterrà $|A|-|B|$.
    \item \textbf{Nastro 3 (Lavoro):} Usato per marcare la lunghezza di $B$ (con 'X').
\end{itemize}

\subsubsection{Implementazione (Stati e Transizioni)}
\begin{description}
    \item[$q_0$ (Copia $|A|$ su Nastro 2):] Legge i caratteri di $A$ dal Nastro 1 e marca 'X' sul Nastro 2 per ogni carattere letto.
    \begin{itemize}
        \item \textbf{Condizione:} Nastro 1: $\alpha \in \{0,1\}$, Nastro 2: $\blankS$.
        \item \textbf{Azione:} Nastro 1: $(\alpha/\alpha, R)$, Nastro 2: $(\blankS/X, R)$. Rimani in $q_0$.
        \item \textbf{Uscita:} Quando Nastro 1 legge \texttt{\#}.
        \item \textbf{Transizione:} Da $q_0$ a $q_1$: Nastro 1: $(\texttt{\#}/\texttt{\#}, R)$, Nastro 2: $(X/X, L)$ (sposta indietro per iniziare la \textbf{sottrazione}).
    \end{itemize}
    \item[$q_1$ (Copia $|B|$ su Nastro 3 e calcola $|A|-|B|$):] Legge i caratteri di $B$ dal Nastro 1, marca 'X' sul Nastro 3 per ogni carattere letto, e cancella una 'X' dal Nastro 2 per ogni carattere letto (simulando $|A|-|B|$).
    \begin{itemize}
        \item \textbf{Condizione:} Nastro 1: $\alpha \in \{0,1\}$, Nastro 2: $X$, Nastro 3: $\blankS$.
        \item \textbf{Azione:} Nastro 1: $(\alpha/\alpha, R)$, Nastro 2: $(X/\blankS, L)$, Nastro 3: $(\blankS/X, R)$. Rimani in $q_1$.
        \item \textbf{Controllo $|A|>|B|$:} Se Nastro 2 diventa $\blankS$ prima che Nastro 1 trovi \texttt{\#}, significa che $|A| \le |B|$, quindi la condizione $|A|>|B|$ non è rispettata. La TM si blocca e rifiuta. Altrimenti, se Nastro 1 trova \texttt{\#} e Nastro 2 ha ancora X, la condizione è rispettata.
        \item \textbf{Uscita:} Quando Nastro 1 legge \texttt{\#}.
        \item \textbf{Transizione:} Da $q_1$ a $q_2$: Nastro 1: $(\texttt{\#}/\texttt{\#}, R)$, Nastro 2: $(X/X, S)$, Nastro 3: $(\blankS/\blankS, L)$. (Nastro 2 rimane fermo sulla X residua, Nastro 3 si riavvolge per il confronto con C).
    \end{itemize}
    \item[$q_2$ (Confronto $|C|$ con $|A|-|B|$ e $|B|>|C|$):] Legge i caratteri di $C$ dal Nastro 1, cancella una 'X' dal Nastro 2 (verificando $|C|=|A|-|B|$), e cancella una 'X' dal Nastro 3 (verificando $|B|>|C|$).
    \begin{itemize}
        \item \textbf{Condizione:} Nastro 1: $\alpha \in \{0,1\}$, Nastro 2: $X$, Nastro 3: $X$.
        \item \textbf{Azione:} Nastro 1: $(\alpha/\alpha, R)$, Nastro 2: $(X/\blankS, L)$, Nastro 3: $(X/\blankS, L)$. Rimani in $q_2$.
        \item \textbf{Controllo $|C|=|A|-|B|$:} Se Nastro 2 diventa $\blankS$ e Nastro 1 non è finito, significa $|C|>|A|-|B|$. Se Nastro 1 finisce e Nastro 2 ha ancora X, significa $|C|<|A|-|B|$. In entrambi i casi, rifiuta. Se Nastro 1 e Nastro 2 finiscono assieme (Nastro 2 legge $\blankS$ quando Nastro 1 legge $\blankS$), allora $|C|=|A|-|B|$.
        \item \textbf{Controllo $|B|>|C|$:} Durante il processo, Nastro 3 deve ancora avere 'X' quando Nastro 1 finisce (cioè Nastro 3 deve diventare $\blankS$ \textbf{dopo} Nastro 1). Se Nastro 3 diventa $\blankS$ prima o insieme a Nastro 1, significa $|B| \le |C|$, quindi rifiuta.
        \item \textbf{Uscita:} Quando Nastro 1 legge $\blankS$.
        \item \textbf{Transizione:} Da $q_2$ a $q_{acc}$: Nastro 1: $(\blankS/\blankS, S)$, Nastro 2: $(\blankS/\blankS, S)$, Nastro 3: $(X/X, S)$ (Nastro 3 deve ancora avere X per $|B|>|C|$).
    \end{itemize}
    \item[Rifiuto:] Tutte le condizioni non soddisfatte portano al rifiuto (es. testina Nastro 2 su $\blankS$ troppo presto, testina Nastro 3 su $\blankS$ troppo presto, etc.).
\end{description}

\subsection{Linguaggio $L_3$}
Il linguaggio $L_3 = \{W\#W\#W \mid W \in \{0,1\}^*\}$ consiste in stringhe $W\#W\#W$, dove $W$ è una stringa (anche vuota) di $0$ o $1$. Le tre occorrenze di $W$ devono essere identiche.
\begin{example}
Esempi: $01\#01\#01$, $\#\#$ (per $W=\epsilon$).
\end{example}

\subsubsection{Strategia}
Utilizziamo 3 nastri.
\begin{itemize}
    \item \textbf{Nastro 1 (Input):} Contiene la stringa di input.
    \item \textbf{Nastro 2 (Lavoro):} Copia la prima $W$.
    \item \textbf{Nastro 3 (Lavoro):} Copia la prima $W$.
\end{itemize}
Dopo aver copiato la prima $W$, le testine di Nastro 2 e 3 vengono riavvolte. Successivamente, la seconda $W$ sul Nastro 1 viene confrontata con la copia su Nastro 2. Infine, la terza $W$ sul Nastro 1 viene confrontata con la copia su Nastro 3.

\subsubsection{Implementazione (Stati e Transizioni)}
\begin{description}
    \item[$q_0$ (Copia $W_1$ su Nastro 2 e 3):] Legge i caratteri della prima $W$ dal Nastro 1 e li copia su Nastro 2 e Nastro 3.
    \begin{itemize}
        \item \textbf{Condizione:} Nastro 1: $\alpha \in \{0,1\}$, Nastro 2: $\blankS$, Nastro 3: $\blankS$.
        \item \textbf{Azione:} Nastro 1: $(\alpha/\alpha, R)$, Nastro 2: $(\blankS/\alpha, R)$, Nastro 3: $(\blankS/\alpha, R)$. Rimani in $q_0$.
        \item \textbf{Uscita:} Quando Nastro 1 legge \texttt{\#}.
        \item \textbf{Transizione:} Da $q_0$ a $q_1$: Nastro 1: $(\#/\#, R)$, Nastro 2: $(\blankS/\blankS, L)$, Nastro 3: $(\blankS/\blankS, L)$. (Per $W=\epsilon$, la TM va direttamente a $q_1$ leggendo \texttt{\#} sul Nastro 1 e $\blankS$ su Nastro 2 e 3. Le testine 2 e 3 si muovono a sinistra sul $\blankS$ e poi, nel successivo riavvolgimento, si riposizionano correttamente).
    \end{itemize}
    \item[$q_1$ (Riavvolgi Nastro 2 e 3):] Riporta le testine di Nastro 2 e 3 all'inizio delle loro rispettive stringhe $W$.
    \begin{itemize}
        \item \textbf{Condizione:} Nastro 2: $\alpha \in \{0,1\}$, Nastro 3: $\alpha \in \{0,1\}$.
        \item \textbf{Azione:} Nastro 2: $(\alpha/\alpha, L)$, Nastro 3: $(\alpha/\alpha, L)$. Rimani in $q_1$.
        \item \textbf{Uscita:} Quando Nastro 2 e Nastro 3 leggono $\blankS$.
        \item \textbf{Transizione:} Da $q_1$ a $q_2$: Nastro 2: $(\blankS/\blankS, R)$, Nastro 3: $(\blankS/\blankS, R)$. (Nastro 1 rimane fermo).
    \end{itemize}
    \item[$q_2$ (Confronta $W_2$ con Nastro 2):] Confronta la seconda $W$ sul Nastro 1 con la copia su Nastro 2.
    \begin{itemize}
        \item \textbf{Condizione:} Nastro 1: $\alpha \in \{0,1\}$, Nastro 2: $\alpha \in \{0,1\}$.
        \item \textbf{Azione:} Nastro 1: $(\alpha/\alpha, R)$, Nastro 2: $(\alpha/\alpha, R)$. Rimani in $q_2$.
        \item \textbf{Uscita:} Quando Nastro 1 legge \texttt{\#} e Nastro 2 legge $\blankS$.
        \item \textbf{Transizione:} Da $q_2$ a $q_3$: Nastro 1: $(\#/\#, R)$, Nastro 2: $(\blankS/\blankS, S)$.
    \end{itemize}
    \item[$q_3$ (Confronta $W_3$ con Nastro 3):] Confronta la terza $W$ sul Nastro 1 con la copia su Nastro 3.
    \begin{itemize}
        \item \textbf{Condizione:} Nastro 1: $\alpha \in \{0,1\}$, Nastro 3: $\alpha \in \{0,1\}$.
        \item \textbf{Azione:} Nastro 1: $(\alpha/\alpha, R)$, Nastro 3: $(\alpha/\alpha, R)$. Rimani in $q_3$.
        \item \textbf{Uscita:} Quando Nastro 1 e Nastro 3 leggono $\blankS$.
        \item \textbf{Transizione:} Da $q_3$ a $q_{acc}$: Nastro 1: $(\blankS/\blankS, S)$, Nastro 3: $(\blankS/\blankS, S)$.
    \end{itemize}
    \item[Rifiuto:] Qualsiasi discrepanza tra i simboli durante i confronti o formato non corretto della stringa porta al rifiuto.
\end{description}

\subsection{Linguaggio $L_4$}
Il linguaggio $L_4 = \{A\#B \mid A,B \in \{a,b\}^*, A \subseteq B \}$ è l'insieme di stringhe $A\#B$ dove $A$ e $B$ sono stringhe sull'alfabeto $\{a,b\}$, e $A$ è una sottostringa di $B$. La stringa vuota ($\epsilon$) è una sottostringa di ogni stringa.
\begin{example}
Esempi: $a\#bab$, $ab\#ababa$, $\#ab$ (A è vuota).
Esempi non in $L_4$: $ab\#ba$, $a\#b$.
\end{example}

\subsubsection{Strategia}
Questa è più complessa a causa della natura della sottostringa, che può apparire in qualsiasi posizione all'interno di $B$.
\begin{itemize}
    \item \textbf{Nastro 1 (Input):} Contiene la stringa di input.
    \item \textbf{Nastro 2 (Lavoro):} Copia la stringa $A$.
\end{itemize}
La strategia consiste nel provare a far corrispondere la stringa $A$ (copiata sul Nastro 2) con ogni possibile sottostringa di $B$ sul Nastro 1. Se un tentativo fallisce, si sposta la posizione di partenza su $B$ di un carattere e si riprova.

\subsubsection{Implementazione (Stati e Transizioni)}
\begin{description}
    \item[$q_0$ (Copia A):] Legge i caratteri di $A$ dal Nastro 1 e li copia sul Nastro 2.
    \begin{itemize}
        \item \textbf{Condizione:} Nastro 1: $\alpha \in \{a,b\}$, Nastro 2: $\blankS$.
        \item \textbf{Azione:} Nastro 1: $(\alpha/\alpha, R)$, Nastro 2: $(\blankS/\alpha, R)$. Rimani in $q_0$.
        \item \textbf{Uscita:} Quando Nastro 1 legge \texttt{\#}.
        \item \textbf{Transizione:} Da $q_0$ a $q_1$: Nastro 1: $(\#/\#, R)$, Nastro 2: $(\blankS/\blankS, L)$.
    \end{itemize}
    \item[$q_1$ (Riavvolgi Nastro 2):] Riporta la testina del Nastro 2 all'inizio della stringa $A$.
    \begin{itemize}
        \item \textbf{Condizione:} Nastro 2: $\alpha \in \{a,b\}$.
        \item \textbf{Azione:} Nastro 2: $(\alpha/\alpha, L)$. Rimani in $q_1$.
        \item \textbf{Uscita:} Quando Nastro 2 legge $\blankS$.
        \item \textbf{Transizione:} Da $q_1$ a $q_2$: Nastro 2: $(\blankS/\blankS, R)$. (Nastro 1 rimane fermo).
    \end{itemize}
    \item[$q_2$ (Tentativo di Match Iniziale):] Questo stato è il punto di partenza per ogni tentativo di confronto di $A$ con una sottostringa di $B$. Gestisce anche il caso di $A = \epsilon$.
    \begin{itemize}
        \item \textbf{Condizione (A è $\epsilon$):} Nastro 2: $\blankS$.
        \item \textbf{Azione:} Nastro 2: $(\blankS/\blankS, S)$. Spostati a $q_5$ e accetta. (Se $A$ è vuota, è sottostringa di qualsiasi $B$).
        \item \textbf{Condizione (Inizio Match):} Nastro 1: $\alpha \in \{a,b\}$, Nastro 2: $\alpha \in \{a,b\}$.
        \item \textbf{Azione:} Nastro 1: $(\alpha/\alpha, R)$, Nastro 2: $(\alpha/\alpha, R)$. Spostati a $q_3$. (Qui inizia il vero confronto).
        \item \textbf{Condizione (Nastro 1 finisce, A non vuota):} Nastro 1: $\blankS$, Nastro 2: $\alpha \in \{a,b\}$. (Se B finisce ma A non è stata trovata)
        \item \textbf{Azione:} Rifiuta. (Non c'è una transizione definita per questo caso, quindi la TM si blocca e rifiuta).
    \end{itemize}
    \item[$q_3$ (Match di $A$):] Continua il confronto tra $A$ (Nastro 2) e la sottostringa di $B$ (Nastro 1).
    \begin{itemize}
        \item \textbf{Condizione (Match):} Nastro 1: $\alpha \in \{a,b\}$, Nastro 2: $\alpha \in \{a,b\}$.
        \item \textbf{Azione:} Nastro 1: $(\alpha/\alpha, R)$, Nastro 2: $(\alpha/\alpha, R)$. Rimani in $q_3$.
        \item \textbf{Uscita (Match di $A$ completato):} Nastro 2 legge $\blankS$.
        \item \textbf{Transizione:} Da $q_3$ a $q_4$: Nastro 1: $(\alpha/\alpha, S)$, Nastro 2: $(\blankS/\blankS, S)$. (Nastro 1 è posizionato dopo la sottostringa, ora controlliamo il resto di B).
        \item \textbf{Uscita (Mismatch):} Nastro 1: $\alpha \in \{a,b\}$, Nastro 2: $\beta \in \{a,b\}$, con $\alpha \ne \beta$.
        \item \textbf{Transizione:} Da $q_3$ a $q_6$: Nastro 1: $(\alpha/\alpha, L)$, Nastro 2: $(\beta/\beta, L)$. (La testina del nastro 1 torna indietro di 1, poi entrambe si riavvolgeranno al punto di inizio per il prossimo tentativo).
    \end{itemize}
    \item[$q_4$ (Verifica fine di B):] Dopo aver trovato $A$ come sottostringa, si assicura che non ci siano simboli \texttt{\#} o altro sul resto di $B$.
    \begin{itemize}
        \item \textbf{Condizione:} Nastro 1: $\alpha \in \{a,b\}$.
        \item \textbf{Azione:} Nastro 1: $(\alpha/\alpha, R)$. Rimani in $q_4$.
        \item \textbf{Uscita:} Nastro 1 legge $\blankS$.
        \item \textbf{Transizione:} Da $q_4$ a $q_5$: Nastro 1: $(\blankS/\blankS, S)$.
    \end{itemize}
    \item[$q_5$ (Accetta):] Stato di accettazione.
    \item[$q_6$ (Riavvolgi Nastro 1 e 2 dopo Mismatch):] Riporta Nastro 1 e Nastro 2 all'inizio della stringa $B$ e $A$ rispettivamente.
    \begin{itemize}
        \item \textbf{Condizione:} Nastro 1: $\gamma \in \{a,b,\#\}$, Nastro 2: $\gamma \in \{a,b,\blankS\}$.
        \item \textbf{Azione:} Nastro 1: $(\gamma/\gamma, L)$, Nastro 2: $(\gamma/\gamma, L)$. Rimani in $q_6$. (Questo porta entrambe le testine indietro fino all'inizio di A e all'inizio di B).
        \item \textbf{Uscita:} Nastro 1 legge \texttt{\#}.
        \item \textbf{Transizione:} Da $q_6$ a $q_2$: Nastro 1: $(\#/\#, R)$, Nastro 2: $(\blankS/\blankS, R)$. (La testina del Nastro 1 si sposta a destra di un posto, la testina del Nastro 2 si posiziona all'inizio di A. Ora Nastro 1 è pronto per un nuovo tentativo di match in $B$ un carattere più avanti).
    \end{itemize}
\end{description}

\subsection{Linguaggio $L_5$}
Il linguaggio $L_5 = \{WW \mid W \in \{0,1\}^+\}$ è l'insieme di stringhe che sono la concatenazione di una stringa $W$ con sé stessa, dove $W$ è non vuota e composta da $0$ o $1$.
\begin{example}
Esempi: $00$, $0101$, $1111$.
Esempi non in $L_5$: $0110$ (non $WW$), $0$ (lunghezza dispari), $\epsilon$ (W non vuota).
\end{example}

\subsubsection{Strategia}
Il problema principale è identificare il punto centrale della stringa senza un delimitatore.
\begin{itemize}
    \item \textbf{Nastro 1 (Input):} Contiene la stringa di input.
    \item \textbf{Nastro 2 (Lavoro):} Utilizzato per determinare la lunghezza di $W$. Si marca un carattere sì e uno no per trovare la metà. Se la lunghezza totale è dispari, rifiuta.
    \item \textbf{Nastro 3 (Lavoro):} Copia la prima metà (il primo $W$).
\end{itemize}

\subsubsection{Implementazione (Stati e Transizioni)}
\begin{description}
    \item[$q_0$ (Determina metà lunghezza):] Scorre la stringa di input, scrivendo una 'X' sul Nastro 2 ogni due caratteri del Nastro 1. Questo identifica la metà della stringa.
    \begin{itemize}
        \item \textbf{Condizione (Pari):} Nastro 1: $\alpha \in \{0,1\}$, Nastro 2: $\blankS$.
        \item \textbf{Azione:} Nastro 1: $(\alpha/\alpha, R)$, Nastro 2: $(\blankS/X, R)$. Spostati a $q_1$. (Marca il primo carattere di una coppia).
        \item \textbf{Uscita (Nastro 1 finito, dispari):} Se Nastro 1 legge $\blankS$ mentre Nastro 2 legge $\blankS$, significa che la stringa era di lunghezza dispari. Rifiuta. (Non c'è transizione definita per Nastro 1: $\blankS$, Nastro 2: $\blankS$).
    \end{itemize}
    \item[$q_1$ (Salta un carattere):] Salta un carattere sul Nastro 1 senza scrivere sul Nastro 2.
    \begin{itemize}
        \item \textbf{Condizione:} Nastro 1: $\alpha \in \{0,1\}$, Nastro 2: $\blankS$.
        \item \textbf{Azione:} Nastro 1: $(\alpha/\alpha, R)$, Nastro 2: $(\blankS/\blankS, R)$. Rimani in $q_0$. (Va a marcare il prossimo carattere della coppia, o si ferma se la stringa finisce).
        \item \textbf{Uscita (Nastro 1 finito, pari):} Se Nastro 1 legge $\blankS$ e Nastro 2 legge $X$, significa che la stringa era di lunghezza pari.
        \item \textbf{Transizione:} Da $q_1$ a $q_2$: Nastro 1: $(\blankS/\blankS, S)$, Nastro 2: $(X/X, S)$. (Nastro 1 si è fermato alla fine, Nastro 2 sulla sua ultima X).
    \end{itemize}
    \item[$q_2$ (Riavvolgi Nastro 1 e 2):] Riporta le testine del Nastro 1 e 2 all'inizio delle stringhe.
    \begin{itemize}
        \item \textbf{Condizione (Nastro 2):} Nastro 2: $X$.
        \item \textbf{Azione:} Nastro 2: $(X/X, L)$. Rimani in $q_2$.
        \item \textbf{Uscita (Nastro 2 finito):} Nastro 2 legge $\blankS$.
        \item \textbf{Transizione:} Da $q_2$ a $q_3$: Nastro 2: $(\blankS/\blankS, R)$.
        \item \textbf{Condizione (Nastro 1):} Nastro 1: $\alpha \in \{0,1\}$.
        \item \textbf{Azione:} Nastro 1: $(\alpha/\alpha, L)$. Rimani in $q_3$.
        \item \textbf{Uscita (Nastro 1 finito):} Nastro 1 legge $\blankS$.
        \item \textbf{Transizione:} Da $q_3$ a $q_4$: Nastro 1: $(\blankS/\blankS, R)$.
    \end{itemize}
    \item[$q_4$ (Copia $W_1$ su Nastro 3):] Copia la prima metà della stringa (i primi $|W|$ caratteri) dal Nastro 1 al Nastro 3, cancellando le 'X' dal Nastro 2.
    \begin{itemize}
        \item \textbf{Condizione:} Nastro 1: $\alpha \in \{0,1\}$, Nastro 2: $X$, Nastro 3: $\blankS$.
        \item \textbf{Azione:} Nastro 1: $(\alpha/\alpha, R)$, Nastro 2: $(X/\blankS, R)$, Nastro 3: $(\blankS/\alpha, R)$. Rimani in $q_4$.
        \item \textbf{Uscita:} Nastro 2 legge $\blankS$. (Abbiamo copiato esattamente metà della stringa).
        \item \textbf{Transizione:} Da $q_4$ a $q_5$: Nastro 2: $(\blankS/\blankS, S)$.
    \end{itemize}
    \item[$q_5$ (Riavvolgi Nastro 3):] Riporta la testina del Nastro 3 all'inizio della stringa $W_1$.
    \begin{itemize}
        \item \textbf{Condizione:} Nastro 3: $\alpha \in \{0,1\}$.
        \item \textbf{Azione:} Nastro 3: $(\alpha/\alpha, L)$. Rimani in $q_5$.
        \item \textbf{Uscita:} Nastro 3 legge $\blankS$.
        \item \textbf{Transizione:} Da $q_5$ a $q_6$: Nastro 3: $(\blankS/\blankS, R)$.
    \end{itemize}
    \item[$q_6$ (Confronta $W_2$ con Nastro 3):] Confronta la seconda metà della stringa sul Nastro 1 con la copia della prima metà sul Nastro 3.
    \begin{itemize}
        \item \textbf{Condizione:} Nastro 1: $\alpha \in \{0,1\}$, Nastro 3: $\alpha \in \{0,1\}$.
        \item \textbf{Azione:} Nastro 1: $(\alpha/\alpha, R)$, Nastro 3: $(\alpha/\alpha, R)$. Rimani in $q_6$.
        \item \textbf{Uscita:} Nastro 1 e Nastro 3 leggono $\blankS$.
        \item \textbf{Transizione:} Da $q_6$ a $q_{acc}$: Nastro 1: $(\blankS/\blankS, S)$, Nastro 3: $(\blankS/\blankS, S)$.
    \end{itemize}
    \item[Rifiuto:] Qualsiasi incongruenza (lunghezza dispari, simboli non corrispondenti durante il confronto) porta al rifiuto.
\end{description}

\subsection{Esercizio Aggiuntivo per Casa}
Sia $L_6 = \{W\#A \mid W,A \in \{0,1\}^+, A=W \text{ OR } A=W^R\}$.
Il linguaggio $L_6$ include stringhe $W\#A$ dove $W$ e $A$ sono stringhe non vuote di $0$ o $1$. La stringa $A$ deve essere identica a $W$ oppure essere la sua inversa ($W^R$).

\end{document}